

Uncertain data comes in many  forms.  For example, sensor readings with
a  margin  of  error
%,  sensitive information  released  with  injected errors,
or  databases containing  predictions describing  a range  of possible
outcomes.    Traditional  database   management  systems   (DBMS)  are
ill-equipped  to manage  uncertain data.
For  example, consider  a predictive model  for customer
orders based  on trends extrapolated  from historical data.   While a
traditional  DBMS can process  queries on  a sampled  order data\-base
(one possible  concrete database without  uncertainty in it) if  it is
somehow instantiated  from the  model, it is  not equipped  to provide
statistical information about quality of the estimates it generates as
a result.


Probabilistic  database  management  systems \cite{dalvi07efficient,
WidomTrio2008, KochMayBMS2008, SD2007, ORION, MCDB, BayesStore}
aim at  addressing  this
issue; the  goal is to  provide better support for  querying uncertain
data.  The core problem in query
processing in probabilistic databases is computing
moments and histograms for uncertain values. This includes as special cases
the computation of expectations of aggregates (for example,
the expected aggregate revenues of a company for the next quarter) and
the computation of probabilities of events definable by queries (e.g.,
the probability that the total sales will exceed a certain amount).

Technically, computing moments 
in the  continuous case calls for the numerical integration of
rather complicated
functions. Assume that uncertainty is encoded via a number of random variables
$X_1 \dots X_k$ for which a probability density function (PDF) $p$ is given.
Computing the expectation of a query given by a function $q(\vec{X})$
is equivalent to computing the integral
%
\begin{equation}
\label{eq:expectation}
E[q] = \int_{x_1=-\infty}^\infty \cdots \int_{x_k=-\infty}^\infty
    p(\vec{x}) \cdot q(\vec{x}) \; d\vec{x}.
\end{equation}
%
Computing the probability of an event defined by a Boolean query can be
thought of as a special case of the above where the query expressing
events of interest defines a function $q$ that maps to
values 0 and 1 only.

In  most practical  cases,  PDFs defined  by  queries are  complicated
functions for which no closed-form  integrals are known.  This is true
even if very  strong simplifying assumptions are made  about the input
data  --  for  example,  that  the  individual  uncertain  values  are
independent from each other and use well-studied distributions such as
normal   distributions:   the   queries  create   complex
statistical dependencies in their own right.

{\em Monte Carlo integration}\/.
There is one conceptually simple technique,
however, that  allows for the (approximate) numerical integration of even  the most
general  functions, including those occurring in  probabilistic data  processing: Monte
Carlo integration \cite{montecarlo}. Conceptually, to compute an expectation, one simply
approximates   the  integral  by  taking $n$ samples
$\vec{x}_1, \dots, \vec{x}_n$ for
$\vec{X}$ from $p$  and  taking  the  average of the $q$ values,
%
\begin{equation}\label{eq:mc_expectation}
\frac{1}{n} \cdot \sum_{i=1}^n q(\vec{x}_i).
\end{equation}

In general, however, even just taking a sample from
a complicated PDF is difficult.  Markov-chain Monte Carlo (MCMC, cf.\ e.g.,
\cite{GRS1995}) is a class of extremely powerful
techniques for doing this of which the
Metropolis-Hastings technique is possibly the best-known example
\cite{metropolis,GRS1995}.
%
MCMC is used pervasively in science, for instance in
simulating  physical processes  or reconstructing  genomes  from short
sequence data.  It is also extensively used
in a new generation of Artificial Intelligence probabilistic inference
systems whose aims are somewhat related to those of probabilistic
database systems \cite{RD2006, BLOG}.




\begin{figure}[!]
\begin{center}
\begin{tabular}{rc}
& Query
\\
%\\
%\hline
\\
\begin{tabular}{r}
query \\
evaluation
\end{tabular}
&
\framebox{
\begin{tabular}{c}
\framebox{
\begin{tabular}{p{4.2cm}}
computing probabilities, \\
moments, and statistical tests
\end{tabular}
}
\\[3ex]
\framebox{
\begin{tabular}{p{4.2cm}}
query plans on c-tables
\end{tabular}
}
\end{tabular}
}
\\[7ex]
\begin{tabular}{l}
data \\
store
\end{tabular}
&
\framebox{
\begin{tabular}{c}
\framebox{
\begin{tabular}{p{4.2cm}}
(probabilistic) c-tables
\end{tabular}
}
\\[2ex]
\framebox{
\begin{tabular}{p{4.2cm}}
succinct representation of \\
joint distribution of random \\
variables (exchangeable)
\end{tabular}
}
\end{tabular}
}
\end{tabular}
\end{center}
\caption{Pip Query Engine Architecture.}
\label{fig:arch}
\end{figure}





{\em Monte Carlo in the discrete case}\/.
Computing probabilities and expectations is by no means only difficult
in  the  continuous case.   In  the  discrete  and finite  case  (each
uncertain value can only take  a finite number of possible values),
the difficulty of computing  the probabilities of events (e.g., that a particular tuple
occurs in  the query result) follows  from the fact that  the probability
mass  in  discrete  distributions  is concentrated  in  {\em  several}
distinct   places   --   similarly   as   in   multimodal   continuous
distributions.   As   a  consequence,  Monte   Carlo  simulation  with
relatively few  samples leads  to low-quality approximations  of probabilities
and moments. In fact, it follows from
\cite{KL1983, KLM1989, GGH1998}
that approximating  the probability of a tuple  in a select-from-where
query with duplicate elimination to  within an error bound relative to
the size of the probability value {\em must}\/ take exponentially many
Monte Carlo  iterations already on a  so-called {\em tuple-independent
probabilistic database}\/
\cite{dalvi07efficient} (arguably the simplest form of a discrete probabilistic
database).
In the discrete case, the exact  computation of
probabilities  and moments  is just  a finite  summation  problem.  However, in the general case 
there  are exponentially  many (inclusion-exclusion)  terms to  sum up. The  problem  is  \#P-hard   and  thus  computationally
infeasible \cite{GGH1998,dalvi07efficient}.

{\em Importance sampling and error bounds}\/.
The exponentiality result just mentioned has been overcome
by a nontrivial form of {\em  importance sampling}\/
originally due to Karp and Luby
\cite{KL1983}, which achieves relative error bounds in polynomial time
(actually, with linearly many Monte-Carlo iterations). Intuitively,
the sampling here is from a derived distribution over an interesting
subspace of the original probability space.
This technique, in more refined form \cite{KLM1989,DKLR2000,Vazirani2001},
is implemented in the MystiQ and MayBMS systems
\cite{RDS07,KO2008,KochMayBMS2008}. It is only known to work for discrete
probabilistic data.



{\em Challenges  of sampling-based query processing}\/.  To  summarize, a
Monte Carlo simulation/sampling-based approach
is  a  natural  design  choice  for  probabilistic  database  systems.
However,  a  Monte Carlo  approach  leaves  us  with the  problems  of
efficiently creating samples (the  techniques we have discussed so far
call for samples that are  entire databases, not just atomic values or
tuples) and computing sufficiently  many samples to yield satisfactory
results.  Additionally, we would like to know how many samples we need to
get certain error bounds.

Recently,  the paper  \cite{MCDB} on  the MCDB  system  has promoted
an integrated  sampling-based  approach to  probabilistic
databases.  Conceptually,  MCDB uses a {\em sample-first}\/ approach:
it   first  computes  samples  of  entire
databases and then processes queries  on these samples.  This is a very
general and flexible approach, largely due to its modular approach to
probability distributions via so called {\em VG-functions}\/.  However,
in its basic form, parallel evaluation over many sampled databases is
very inefficient; as an optimization, samples are computed lazily and
bundled by tuple.  

{\em  Conditional tables}\/  (c-tables, \cite{IL1984})  are relational
tables in which tuples have associated conditions expressed as boolean
expressions over  comparisons of random variables  and constants.
C-tables are a natural way
to  represent  the  {\em  deterministic skeleton}\/  of a probabilistic
relational  database in  a succinct  and tabular  form.  That  is, complete
information  about uncertain data is encoded using random  variables, excluding
only  specifications  of the  joint  probability  distribution of  the
random  variables   themselves.   This  model   allows 
representation of  input databases  with  nontrivial statistical  dependencies
that are normally associated with graphical models.

For discrete probabilistic  databases, a canon of systems has been
developed that
essentially uses  c-tables, without referring to them as such.
MystiQ  \cite{dalvi07efficient}  uses  c-tables internally  for  query
processing  but  uses  a  simpler  model for  input  databases.   Trio
\cite{WidomTrio2008}  uses  c-tables with  additional  syntactic sugar  and
calls conditions {\em lineage}\/.  MayBMS
\cite{AJKO2008}  uses a  form of  c-tables called  U-relations that define
how relational algebra representations of queries can encode the corresponding condition
transformations.

Using
c-tables  for  query evaluation  allows  compositional  transformation of
 probabilistic  database representations, thus allowing a single c-table to sumultaneously represent  all
possible worlds  (sample databases) at once. Samples can then
be drawn as late in the query evaluation process as possible,  after all
relational algebra work is finished.
This has  many advantages over  sample-first
approaches.
\begin{itemize}
\addtolength{\topsep}{-0.3ex}
\addtolength{\labelsep}{-0.3ex}
\addtolength{\itemsep}{-1ex}
\item
Evaluating relational algebra on
c-tables is  essentially as efficient as processing  the same algebra
queries on a  single  MCDB  sample. By evaluating
query operations on  the c-table, we may be able  to filter out tuples
that would otherwise have to be  sampled before being dropped  in a
sample-first approach.  This
allows us  to sample more selectively  for the needs of  the query and
produce higher-quality results with  the same amount of work as
in a sample-first approach.
This  is particularly important for highly  selective queries or
aggregation queries with group-by constructs, for which relatively low
numbers   of   samples   (as   MCDB   produces   them) result in notoriously
imprecise answers.

\item
Computing further samples in the c-tables approach
is  cheaper than in  MCDB, because  we do  so only  after most  of the
querying  work  has been  done.   Thus,  using  c-tables, multiple sample sets
may be produced without incurring query processing overheads more than once.

This is particularly important in
online  sampling, where we want to  quickly produce  a first  result
based on few samples and then  improve on the result by computing more
samples until the user is satisfied with the result.
\end{itemize}


\begin{example}\em
\label{ex:intro}
Suppose a database captures customer orders expected for the next quarter,
including prices and
and destinations of shipment. The order prices are 
uncertain, but a probability distribution is assumed.
The database also stores
distributions of shipping durations for each location.
Here are two c-tables defining such a probabilistic database:
\[
\begin{tabular}{c|ccc}
Order & Cust & ShipTo & Price \\
\hline
& Joe & NY & $X_1$ \\
& Bob & LA & $X_3$ \\
\end{tabular}
%\hspace{5mm}
\]\[
\begin{tabular}{c|cc}
Shipping & Dest & Duration \\
\hline
& NY & $X_2$ \\
& LA & $X_4$ \\
\end{tabular}
\]
We assume a suitable specification of the joint distribution $p$ of the random
variables $X_1,\dots,X_4$ occurring in this database.

Now consider the query
\begin{verbatim}
select expected_sum(O.Price)
from   Order O, Shipping S
where  O.ShipTo = S.Dest
and    O.Cust = 'Joe'
and    S.Duration >= 7;
\end{verbatim}
asking for the expected loss due to late deliveries to customers named Joe,
where the product is free if not delivered within seven days.
%
This can be approximated by drawing a number of samples from $p$
and using formula~\ref{eq:mc_expectation}
to approximate $E[q]$,
where $q$ represents the result of the sum aggregate query on a sample,
here
\[
q(\vec{x}) =
\left\{
\begin{array}{lll}
x_1 & \dots & x_2 \ge 7 \\
0 & \dots & \mbox{otherwise.}
\end{array}
\right.
\]

In a naive sample-first approach, 
if $x_2 \ge 7$ is a relatively rare event, a large number of samples will be
required to compute a good approximation to the expectation.
Moreover, samples also require an effort for customer Bob,
which does not contribute to the result.

Now consider using c-tables. The result of the relational algebra part of the
above query can be easily computed without looking at $f$ as
\[
\begin{tabular}{c|c|c}
R & Price & Condition \\
\hline
& $X_1$ & $X_2 \le 7$ \\
\end{tabular}
\]
This c-table compactly represents all relevant data still relevant after the
application of the relational algebra part of the query, other than $p$,
which remains unchanged.
Sampling from R to compute
\begin{verbatim}
select expected_sum(Price) from R;
\end{verbatim}
is a much more focused effort.
First, we only have to consider the random variables relating to Joe;
but determining that random variable $X_2$ is relevant while $X_4$
is not requires
executing a query involving a join. We want to do this query first, before
we start sampling.

Second, assume that delivery times are
independent from sales volumes. Then we can approximate the
query result
by first sampling an $X_2$ value and only sampling an $X_1$ value if $X_2 \ge 7$.
Otherwise, we use $0$ as the $X_1$ value.
If $X_2 \ge 7$ is relatively rare (e.g., the average shipping times to NY are
very slow, with a low variance), this may reduce the amount of samples
for $X_1$ that are first computed and then discarded without seeing use
considerably.
If CDFs are available, we can of course do even better.

In Section \ref{sec:evaluation} we compare the c-tables approach to a very basic, 
naive sample-first approach.  Note however, that \cite{MCDB} describes a number of 
optimizations that MCDB uses to obtain some of the advantages gained by using c-tables.
%
\punto
\end{example}


The  c-tables  approach has  never  been used  to build  a
probabilistic  database  management  system that  supports  continuous
probability  distributions.
%
ORION \cite{ORION} is a probabilistic database management system for
continuous distributions that can alternate between sampling
and transforming distributions. However, their representation
system is not based on c-tables but essentially on the
world-set decompositions of \cite{AKO07WSD}, a factorization
based approach related to graphical models.
Selection queries in this model may require an exponential blow-up in the
representation size, while selections are efficient in c-tables.

In addition,  we have a  vision of  a
powerful  unified approach that  combines transformations  of succinct
representations (c-tables)  and Monte-Carlo techniques (beyond
the Karp-Luby algorithm for handling duplicate elimination in
discrete systems), which has never been put forward before.




{\em The Pip System}\/ is a probabilistic database system
based on probabilistic c-tables that combines the strong points of
recent discrete systems such as MystiQ and MayBMS with the generality of
the Monte-Carlo approach of MCDB. It supports both discrete and
continuous probability distributions, powerful correlations definable
by queries, expectations of aggregates and distinct-aggregates with or
without  group-by,  and the computation of confidences.

The detailed technical contributions of this paper are as follows.
\begin{itemize}
\addtolength{\topsep}{-0.3ex}
\addtolength{\labelsep}{-0.3ex}
\addtolength{\itemsep}{-1ex}
\item
We study query evaluation using probabilistic conditional tables, for
relational algebra, aggregates, and the computation of expectations and
probabilities.

\item
We present the architectural and language design decisions made in the
Pip system, and show how the various techniques introduced combine into
a unified whole.

\item
We study Monte-Carlo sampling and integration.
We present a Karp-Luby style importance
sampling algorithm for continuous distributions.
This technique is essential
in scenarios where very small probabilities have to be computed up to a small
relative error, as is often the case for the computation of conditional
probabilities (as ratios of probabilities).
{\em The title of this
paper in part relates to this: our system can compute expectations of 
great quality, even if small}.

\item
We provide experimental evidence for the competitiveness
of our approach, comparing PIP with a reimplementation of the
refined sample-first approach taken by MCDB. 
We use a common codebase for both systems based on Postgres to enable
fair comparison. We show that PIP is in general competitive to MCDB (it essentially never does more work) and can reduce the amount of work needed to compute the same number of samples substantially; in other cases, delaying sampling using c-tables to a stage where more is known about what samples are needed;
This added information can lead to the same number of samples giving better quality results in
PIP than in MCDB.
\end{itemize}


The       structure       of        this       paper       is       as
follows. Section~\ref{sec:background} presents probabilistic c-tables,
our  database  model, and  the  conceptual  evaluation  of queries  on
probabilistic c-tables. Section~\ref{sec:sampling}  studies  sampling,  Monte
Carlo      integration,     and      the      continuous     Karp-Luby
algorithm.  In Section~\ref{sec:design}, we present a high level view of Pip in
the  abstract.   Section~\ref{sec:implementation} discusses details  of the
implementation of  PIP and our reimplementation  of MCDB.    Finally,   in
Section~\ref{sec:evaluation},   we  present   the   outcomes  of   our
experiments with PIP and our MCDB reimplementation.



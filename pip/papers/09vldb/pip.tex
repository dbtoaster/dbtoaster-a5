\documentclass{vldb}

\usepackage{latexsym}
\usepackage{amsmath}
\usepackage{subfigure}
%\usepackage{epsfig}
%\usepackage{epic}
%\usepackage{eepic}
%\usepackage{xspace}
%\usepackage{pstricks}
%\usepackage{pst-doc}
%\usepackage{pst-func,pst-math,pst-xkey}

\def\punto{$\hspace*{\fill}\Box$}
\newcommand{\nop}[1]{}
\newcommand{\tuple}[1]{{\langle#1\rangle}}
\def\lBrack{\lbrack\!\lbrack}
\def\rBrack{\rbrack\!\rbrack}
\newcommand{\Bracks}[1]{\lBrack#1\rBrack}


\leftmargini 2.9ex


\newtheorem{theorem}{Theorem}[section]
\newtheorem{metatheorem}{Metatheorem}[section]
\newtheorem{example}[theorem]{Example}
\newtheorem{algorithm}[theorem]{Algorithm}
\newtheorem{definition}[theorem]{Definition}
\newtheorem{proposition}[theorem]{Proposition}
\newtheorem{property}[theorem]{Property}
\newtheorem{corollary}[theorem]{Corollary}
\newtheorem{lemma}[theorem]{Lemma}
\newtheorem{remark}[theorem]{Remark}
\newtheorem{conjecture}[theorem]{Conjecture}
\newtheorem{proviso}[theorem]{Proviso}
\newtheorem{todo}[theorem]{ToDo}

%\addtolength{\textwidth}{1in}
%\addtolength{\oddsidemargin}{-0.5in}
%\addtolength{\evensidemargin}{-0.5in}
%\addtolength{\textheight}{1.5in}
%\addtolength{\topmargin}{-1in}


\title{PIP: A Database System for Great and Small Expectations%
% Efficiently Computing
\thanks{The
system is named after Philip Pirrip, nicknamed Pip, the protagonist of
Charles Dickens' novel Great Expectations.}}


\author{Oliver Kennedy and Christoph Koch\\
Department of Computer Science \\
Cornell University, Ithaca, NY, USA \\
\{okennedy, koch\}@cs.cornell.edu}

\date{}


\begin{document}


\numberofauthors{2}


\maketitle



\begin{abstract}
We describe PIP, a probabilistic database system
that combines the strengths of
recent discrete systems such as MystiQ and MayBMS with the generality of
the Monte-Carlo approach of MCDB. It supports both discrete and
continuous probability distributions, powerful correlations definable
by queries, expectations of aggregates and distinct-aggregates with or
without  group-by,  and the computation of confidences.
%
PIP uses c-tables to delay and minimize the amount of sampling work required.
We study Monte-Carlo sampling and integration and
present a Karp-Luby style importance
sampling algorithm for continuous distributions.
This technique is essential
in scenarios where very small probabilities have to be approximated to a small
relative error, as is often the case for the computation of conditional
probabilities (as ratios of probabilities).
%
We also provide experimental evidence for the competitiveness
of our approach, comparing PIP with a reimplementation of the
refined sample-first approach taken by MCDB. 
\end{abstract}



\section{Introduction}
\label{sec:introduction}


Uncertain data comes in many  forms.  For example, sensor readings with
a  margin  of  error
%,  sensitive information  released  with  injected errors,
or  databases containing  predictions describing  a range  of possible
outcomes.    Traditional  database   management  systems   (DBMS)  are
ill-equipped  to manage  uncertain data.
For  example, consider  a predictive model  for customer
orders based  on trends extrapolated  from historical data.   While a
traditional  DBMS can process  queries on  a sampled  order data\-base
(one possible  concrete database without  uncertainty in it) if  it is
somehow instantiated  from the  model, it is  not equipped  to provide
statistical information about quality of the estimates it generates as
a result.


Probabilistic  database  management  systems \cite{dalvi07efficient,
WidomTrio2008, KochMayBMS2008, SD2007, ORION, MCDB, BayesStore}
aim at  addressing  this
issue; the  goal is to  provide better support for  querying uncertain
data.  The core problem in query
processing in probabilistic databases is computing
moments and histograms for uncertain values. This includes as special cases
the computation of expectations of aggregates (for example,
the expected aggregate revenues of a company for the next quarter) and
the computation of probabilities of events definable by queries (e.g.,
the probability that the total sales will exceed a certain amount).

Technically, computing moments 
in the  continuous case calls for the numerical integration of
rather complicated
functions. Assume that uncertainty is encoded via a number of random variables
$X_1 \dots X_k$ for which a probability density function (PDF) $p$ is given.
Computing the expectation of a query given by a function $q(\vec{X})$
is equivalent to computing the integral
%
\begin{equation}
\label{eq:expectation}
E[q] = \int_{x_1=-\infty}^\infty \cdots \int_{x_k=-\infty}^\infty
    p(\vec{x}) \cdot q(\vec{x}) \; d\vec{x}.
\end{equation}
%
Computing the probability of an event defined by a Boolean query can be
thought of as a special case of the above where the query expressing
events of interest defines a function $q$ that maps to
values 0 and 1 only.

In  most practical  cases,  PDFs defined  by  queries are  complicated
functions for which no closed-form  integrals are known.  This is true
even if very  strong simplifying assumptions are made  about the input
data  --  for  example,  that  the  individual  uncertain  values  are
independent from each other and use well-studied distributions such as
normal   distributions:   the   queries  create   complex
statistical dependencies in their own right.

{\em Monte Carlo integration}\/.
There is one conceptually simple technique,
however, that  allows for the (approximate) numerical integration of even  the most
general  functions, including those occurring in  probabilistic data  processing: Monte
Carlo integration \cite{montecarlo}. Conceptually, to compute an expectation, one simply
approximates   the  integral  by  taking $n$ samples
$\vec{x}_1, \dots, \vec{x}_n$ for
$\vec{X}$ from $p$  and  taking  the  average of the $q$ values,
%
\begin{equation}\label{eq:mc_expectation}
\frac{1}{n} \cdot \sum_{i=1}^n q(\vec{x}_i).
\end{equation}

In general, however, even just taking a sample from
a complicated PDF is difficult.  Markov-chain Monte Carlo (MCMC, cf.\ e.g.,
\cite{GRS1995}) is a class of extremely powerful
techniques for doing this of which the
Metropolis-Hastings technique is possibly the best-known example
\cite{metropolis,GRS1995}.
%
MCMC is used pervasively in science, for instance in
simulating  physical processes  or reconstructing  genomes  from short
sequence data.  It is also extensively used
in a new generation of Artificial Intelligence probabilistic inference
systems whose aims are somewhat related to those of probabilistic
database systems \cite{RD2006, BLOG}.




\begin{figure}[!]
\begin{center}
\begin{tabular}{rc}
& Query
\\
%\\
%\hline
\\
\begin{tabular}{r}
query \\
evaluation
\end{tabular}
&
\framebox{
\begin{tabular}{c}
\framebox{
\begin{tabular}{p{4.2cm}}
computing probabilities, \\
moments, and statistical tests
\end{tabular}
}
\\[3ex]
\framebox{
\begin{tabular}{p{4.2cm}}
query plans on c-tables
\end{tabular}
}
\end{tabular}
}
\\[7ex]
\begin{tabular}{l}
data \\
store
\end{tabular}
&
\framebox{
\begin{tabular}{c}
\framebox{
\begin{tabular}{p{4.2cm}}
(probabilistic) c-tables
\end{tabular}
}
\\[2ex]
\framebox{
\begin{tabular}{p{4.2cm}}
succinct representation of \\
joint distribution of random \\
variables (exchangeable)
\end{tabular}
}
\end{tabular}
}
\end{tabular}
\end{center}
\caption{Pip Query Engine Architecture.}
\label{fig:arch}
\end{figure}





{\em Monte Carlo in the discrete case}\/.
Computing probabilities and expectations is by no means only difficult
in  the  continuous case.   In  the  discrete  and finite  case  (each
uncertain value can only take  a finite number of possible values),
the difficulty of computing  the probabilities of events (e.g., that a particular tuple
occurs in  the query result) follows  from the fact that  the probability
mass  in  discrete  distributions  is concentrated  in  {\em  several}
distinct   places   --   similarly   as   in   multimodal   continuous
distributions.   As   a  consequence,  Monte   Carlo  simulation  with
relatively few  samples leads  to low-quality approximations  of probabilities
and moments. In fact, it follows from
\cite{KL1983, KLM1989, GGH1998}
that approximating  the probability of a tuple  in a select-from-where
query with duplicate elimination to  within an error bound relative to
the size of the probability value {\em must}\/ take exponentially many
Monte Carlo  iterations already on a  so-called {\em tuple-independent
probabilistic database}\/
\cite{dalvi07efficient} (arguably the simplest form of a discrete probabilistic
database).
In the discrete case, the exact  computation of
probabilities  and moments  is just  a finite  summation  problem.  However, in the general case 
there  are exponentially  many (inclusion-exclusion)  terms to  sum up. The  problem  is  \#P-hard   and  thus  computationally
infeasible \cite{GGH1998,dalvi07efficient}.

{\em Importance sampling and error bounds}\/.
The exponentiality result just mentioned has been overcome
by a nontrivial form of {\em  importance sampling}\/
originally due to Karp and Luby
\cite{KL1983}, which achieves relative error bounds in polynomial time
(actually, with linearly many Monte-Carlo iterations). Intuitively,
the sampling here is from a derived distribution over an interesting
subspace of the original probability space.
This technique, in more refined form \cite{KLM1989,DKLR2000,Vazirani2001},
is implemented in the MystiQ and MayBMS systems
\cite{RDS07,KO2008,KochMayBMS2008}. It is only known to work for discrete
probabilistic data.



{\em Challenges  of sampling-based query processing}\/.  To  summarize, a
Monte Carlo simulation/sampling-based approach
is  a  natural  design  choice  for  probabilistic  database  systems.
However,  a  Monte Carlo  approach  leaves  us  with the  problems  of
efficiently creating samples (the  techniques we have discussed so far
call for samples that are  entire databases, not just atomic values or
tuples) and computing sufficiently  many samples to yield satisfactory
results.  Additionally, we would like to know how many samples we need to
get certain error bounds.

Recently,  the paper  \cite{MCDB} on  the MCDB  system  has promoted
an integrated  sampling-based  approach to  probabilistic
databases.  Conceptually,  MCDB uses a {\em sample-first}\/ approach:
it   first  computes  samples  of  entire
databases and then processes queries  on these samples.  This is a very
general and flexible approach, largely due to its modular approach to
probability distributions via so called {\em VG-functions}\/.  However,
in its basic form, parallel evaluation over many sampled databases is
very inefficient; as an optimization, samples are computed lazily and
bundled by tuple.  

{\em  Conditional tables}\/  (c-tables, \cite{IL1984})  are relational
tables in which tuples have associated conditions expressed as boolean
expressions over  comparisons of random variables  and constants.
C-tables are a natural way
to  represent  the  {\em  deterministic skeleton}\/  of a probabilistic
relational  database in  a succinct  and tabular  form.  That  is, complete
information  about uncertain data is encoded using random  variables, excluding
only  specifications  of the  joint  probability  distribution of  the
random  variables   themselves.   This  model   allows 
representation of  input databases  with  nontrivial statistical  dependencies
that are normally associated with graphical models.

For discrete probabilistic  databases, a canon of systems has been
developed that
essentially uses  c-tables, without referring to them as such.
MystiQ  \cite{dalvi07efficient}  uses  c-tables internally  for  query
processing  but  uses  a  simpler  model for  input  databases.   Trio
\cite{WidomTrio2008}  uses  c-tables with  additional  syntactic sugar  and
calls conditions {\em lineage}\/.  MayBMS
\cite{AJKO2008}  uses a  form of  c-tables called  U-relations that define
how relational algebra representations of queries can encode the corresponding condition
transformations.

Using
c-tables  for  query evaluation  allows  compositional  transformation of
 probabilistic  database representations, thus allowing a single c-table to sumultaneously represent  all
possible worlds  (sample databases) at once. Samples can then
be drawn as late in the query evaluation process as possible,  after all
relational algebra work is finished.
This has  many advantages over  sample-first
approaches.
\begin{itemize}
\addtolength{\topsep}{-0.3ex}
\addtolength{\labelsep}{-0.3ex}
\addtolength{\itemsep}{-1ex}
\item
Evaluating relational algebra on
c-tables is  essentially as efficient as processing  the same algebra
queries on a  single  MCDB  sample. By evaluating
query operations on  the c-table, we may be able  to filter out tuples
that would otherwise have to be  sampled before being dropped  in a
sample-first approach.  This
allows us  to sample more selectively  for the needs of  the query and
produce higher-quality results with  the same amount of work as
in a sample-first approach.
This  is particularly important for highly  selective queries or
aggregation queries with group-by constructs, for which relatively low
numbers   of   samples   (as   MCDB   produces   them) result in notoriously
imprecise answers.

\item
Computing further samples in the c-tables approach
is  cheaper than in  MCDB, because  we do  so only  after most  of the
querying  work  has been  done.   Thus,  using  c-tables, multiple sample sets
may be produced without incurring query processing overheads more than once.

This is particularly important in
online  sampling, where we want to  quickly produce  a first  result
based on few samples and then  improve on the result by computing more
samples until the user is satisfied with the result.
\end{itemize}


\begin{example}\em
\label{ex:intro}
Suppose a database captures customer orders expected for the next quarter,
including prices and
and destinations of shipment. The order prices are 
uncertain, but a probability distribution is assumed.
The database also stores
distributions of shipping durations for each location.
Here are two c-tables defining such a probabilistic database:
\[
\begin{tabular}{c|ccc}
Order & Cust & ShipTo & Price \\
\hline
& Joe & NY & $X_1$ \\
& Bob & LA & $X_3$ \\
\end{tabular}
%\hspace{5mm}
\]\[
\begin{tabular}{c|cc}
Shipping & Dest & Duration \\
\hline
& NY & $X_2$ \\
& LA & $X_4$ \\
\end{tabular}
\]
We assume a suitable specification of the joint distribution $p$ of the random
variables $X_1,\dots,X_4$ occurring in this database.

Now consider the query
\begin{verbatim}
select expected_sum(O.Price)
from   Order O, Shipping S
where  O.ShipTo = S.Dest
and    O.Cust = 'Joe'
and    S.Duration >= 7;
\end{verbatim}
asking for the expected loss due to late deliveries to customers named Joe,
where the product is free if not delivered within seven days.
%
This can be approximated by drawing a number of samples from $p$
and using formula~\ref{eq:mc_expectation}
to approximate $E[q]$,
where $q$ represents the result of the sum aggregate query on a sample,
here
\[
q(\vec{x}) =
\left\{
\begin{array}{lll}
x_1 & \dots & x_2 \ge 7 \\
0 & \dots & \mbox{otherwise.}
\end{array}
\right.
\]

In a naive sample-first approach, 
if $x_2 \ge 7$ is a relatively rare event, a large number of samples will be
required to compute a good approximation to the expectation.
Moreover, samples also require an effort for customer Bob,
which does not contribute to the result.

Now consider using c-tables. The result of the relational algebra part of the
above query can be easily computed without looking at $f$ as
\[
\begin{tabular}{c|c|c}
R & Price & Condition \\
\hline
& $X_1$ & $X_2 \le 7$ \\
\end{tabular}
\]
This c-table compactly represents all relevant data still relevant after the
application of the relational algebra part of the query, other than $p$,
which remains unchanged.
Sampling from R to compute
\begin{verbatim}
select expected_sum(Price) from R;
\end{verbatim}
is a much more focused effort.
First, we only have to consider the random variables relating to Joe;
but determining that random variable $X_2$ is relevant while $X_4$
is not requires
executing a query involving a join. We want to do this query first, before
we start sampling.

Second, assume that delivery times are
independent from sales volumes. Then we can approximate the
query result
by first sampling an $X_2$ value and only sampling an $X_1$ value if $X_2 \ge 7$.
Otherwise, we use $0$ as the $X_1$ value.
If $X_2 \ge 7$ is relatively rare (e.g., the average shipping times to NY are
very slow, with a low variance), this may reduce the amount of samples
for $X_1$ that are first computed and then discarded without seeing use
considerably.
If CDFs are available, we can of course do even better.

In Section \ref{sec:evaluation} we compare the c-tables approach to a very basic, 
naive sample-first approach.  Note however, that \cite{MCDB} describes a number of 
optimizations that MCDB uses to obtain some of the advantages gained by using c-tables.
%
\punto
\end{example}


The  c-tables  approach has  never  been used  to build  a
probabilistic  database  management  system that  supports  continuous
probability  distributions.
%
ORION \cite{ORION} is a probabilistic database management system for
continuous distributions that can alternate between sampling
and transforming distributions. However, their representation
system is not based on c-tables but essentially on the
world-set decompositions of \cite{AKO07WSD}, a factorization
based approach related to graphical models.
Selection queries in this model may require an exponential blow-up in the
representation size, while selections are efficient in c-tables.

In addition,  we have a  vision of  a
powerful  unified approach that  combines transformations  of succinct
representations (c-tables)  and Monte-Carlo techniques (beyond
the Karp-Luby algorithm for handling duplicate elimination in
discrete systems), which has never been put forward before.




{\em The Pip System}\/ is a probabilistic database system
based on probabilistic c-tables that combines the strong points of
recent discrete systems such as MystiQ and MayBMS with the generality of
the Monte-Carlo approach of MCDB. It supports both discrete and
continuous probability distributions, powerful correlations definable
by queries, expectations of aggregates and distinct-aggregates with or
without  group-by,  and the computation of confidences.

The detailed technical contributions of this paper are as follows.
\begin{itemize}
\addtolength{\topsep}{-0.3ex}
\addtolength{\labelsep}{-0.3ex}
\addtolength{\itemsep}{-1ex}
\item
We study query evaluation using probabilistic conditional tables, for
relational algebra, aggregates, and the computation of expectations and
probabilities.

\item
We present the architectural and language design decisions made in the
Pip system, and show how the various techniques introduced combine into
a unified whole.

\item
We study Monte-Carlo sampling and integration.
We present a Karp-Luby style importance
sampling algorithm for continuous distributions.
This technique is essential
in scenarios where very small probabilities have to be computed up to a small
relative error, as is often the case for the computation of conditional
probabilities (as ratios of probabilities).
{\em The title of this
paper in part relates to this: our system can compute expectations of 
great quality, even if small}.

\item
We provide experimental evidence for the competitiveness
of our approach, comparing PIP with a reimplementation of the
refined sample-first approach taken by MCDB. 
We use a common codebase for both systems based on Postgres to enable
fair comparison. We show that PIP is in general competitive to MCDB (it essentially never does more work) and can reduce the amount of work needed to compute the same number of samples substantially; in other cases, delaying sampling using c-tables to a stage where more is known about what samples are needed;
This added information can lead to the same number of samples giving better quality results in
PIP than in MCDB.
\end{itemize}


The       structure       of        this       paper       is       as
follows. Section~\ref{sec:background} presents probabilistic c-tables,
our  database  model, and  the  conceptual  evaluation  of queries  on
probabilistic c-tables. Section~\ref{sec:sampling}  studies  sampling,  Monte
Carlo      integration,     and      the      continuous     Karp-Luby
algorithm.  In Section~\ref{sec:design}, we present a high level view of Pip in
the  abstract.   Section~\ref{sec:implementation} discusses details  of the
implementation of  PIP and our reimplementation  of MCDB.    Finally,   in
Section~\ref{sec:evaluation},   we  present   the   outcomes  of   our
experiments with PIP and our MCDB reimplementation.




\section{Probabilistic C-Tables}
\label{sec:background}

\def\bagopen{\{\!|\,}
\def\bagclose{\,|\!\}}


In the following, we use a multiset semantics for tables: Tables may
contain duplicate tuples. Using $\in$ as an iterator over multisets in
comprehension notation $\bagopen \cdot \mid \cdot \bagclose$ preserves
duplicates. We use $\uplus$ to denote bag union, which can be thought of
as list concatenation if the multisets are represented as unsorted lists.


\subsection{C-tables}


{\em Conditional tables  (c-tables)}\/ \cite{IL1984} are extensions of
relational databases to handle  uncertainty. A c-table  over
a set  of variables is  a relational table
extended by a  column for holding a \textit{local  condition} for each
tuple.   A local condition  is a  Boolean combination  (using ``and'',
``or'', and ``not'') of  atomic conditions, which are constructed from
variables  and constants  using  $=$, $<$,  $\leq$,  $\neq$, $>$,  and
$\geq$.   The fields  of the  remaining data  columns may  hold domain
values or variables.

Given  a variable  assignment $\theta$  that maps  each variable  to a
domain  value  and  a  condition $\phi$,  $\theta(\phi)$  denotes  the
condition  obtained  from  $\phi$   by  replacing  each  variable  $X$
occurring  in  it   by  $\theta(X)$.   Analogously,  $\theta(\vec{t})$
denotes  the tuple  obtained  from tuple  $\vec{t}$  by replacing  all
variables using $\theta$.

The semantics  of c-tables are defined  in terms of  possible worlds as
follows.  A  possible world is  identified with a  variable assignment
$\theta$.  A relation $R$ in  that possible world is obtained from its
c-table $C_R$ as
$$R  :=  \bagopen  \theta(\vec{t})   \mid  (\vec{t},  \phi)  \in  C_R,
   \theta(\phi) \mbox{  is true} \bagclose.$$ That is,  for each tuple
   $(\vec{t},  \phi)$  of  the  c-table,  where $\phi$  is  the  local
   condition   and  $\vec{t}$   is   the  remainder   of  the   tuple,
   $\theta(\vec{t})$ exists in the world if and only if $\theta(\phi)$
   is true.  Note  that each c-table has at  least one possible world,
   but worlds  constructed from  distinct variable assignments  do not
   necessarily represent different database instances.



\subsection{Relational algebra on c-tables}


Evaluating relational algebra on c-tables (and without the slightest difference, on probabilistic c-tables, since probabilities need not be touched at all) is surprisingly straightforward. The evaluation of the operators of relational
algebra on multiset c-tables is summarized in Figure \ref{fig:ctables-relalg}.
An explicit operator ``distinct'' is used to perform duplicate elimination.  


\begin{figure}[t!]
\begin{center}
\begin{eqnarray*}
C_{\sigma_\psi(R)} &=&
   \bagopen (\vec{r}, \phi \land \psi[\vec{r}]) \mid (\vec{r}, \phi) \in C_R
   \bagclose
\\
&\dots& \mbox{$\psi[\vec{r}]$ denotes $\psi$ with each reference to}
\\
&& \mbox{a column $A$ of $R$ replaced by $\vec{r}.A$.}
\\[1ex]
C_{\pi_{\vec{A}}(R)} &=&
   \bagopen (\vec{r}.\vec{A}, \phi) \mid (\vec{r}, \phi) \in C_R \bagclose
\\
C_{R \times S} &=& \bagopen (\vec{r}, \vec{s}, \phi \land \psi) \mid
   (\vec{r}, \phi) \in C_R, (\vec{s}, \psi) \in C_S \bagclose
\\
C_{R \cup S} &=& C_R \uplus C_S
\\
C_{\mathrm{distinct}(R)} &=&
\bagopen (\vec{r},
    \bigvee \{ \phi \mid (\vec{r}, \phi) \in C_R \} )
    \mid (\vec{r}, \cdot) \in C_R \bagclose
\\
C_{R - S} &=& \bagopen (\vec{r}, \phi \land \psi) \mid
   (\vec{r}, \phi) \in C_{\mathrm{distinct}(R)}, \\
&& \quad\quad
   \mbox{if } (\vec{r}, \pi) \in C_{\mathrm{distinct}(S)} \mbox{ then } \psi := \neg \pi \\
&& \quad\quad
   \mbox{else } \psi := \mbox{true} \bagclose
\end{eqnarray*}

\vspace{-3mm}

\caption{Relational algebra on c-tables.}
\label{fig:ctables-relalg}
\end{center}
\end{figure}


\begin{example}\em
We continue the example from the introduction. The input c-tables are
\[
C_{\mathrm{Order}} = \bagopen ((Joe, NY, X_1), \mbox{true}),
((Bob, LA, X_3), \mbox{true}) \bagclose
\]
and
\[
C_{\mathrm{Shipping}} = \bagopen ((NY, X_2), \mbox{true}),
   ((LA, X_4), \mbox{true}) \bagclose.
\]
The relational algebra query is
\begin{multline*}
\pi_{\mathrm{Price}}(\sigma_{\mathrm{ShipTo} = \mathrm{Dest}}( \\
\sigma_{\mathrm{Cust}='Joe'}(\mathrm{Order}) \times
\sigma_{\mathrm{Duration} \ge 7}(\mathrm{Shipping}))).
\end{multline*}
%
We compute
$
C_{\sigma_{\mathrm{Cust}='Joe'}(\mathrm{Order})} = \bagopen ((Joe, NY, X_1), \mbox{true})
\bagclose
$,
$
C_{\sigma_{\mathrm{Duration} \le 7}(\mathrm{Shipping})} =
\bagopen ((NY, X_2), X_2 \ge 7),
((LA, X_4)$, \\
$X_4 \le 7) \bagclose,
$
%
and
$C_{\sigma_{\mathrm{Cust}='Joe'}(\mathrm{Order}) \times
\sigma_{\mathrm{Duration} \le 7}(\mathrm{Shipping})}$ = \\
$\bagopen ((Joe, NY$, $X_1, NY, X_2), X_2 \le 7),
((Joe, NY, X_1, LA, X_4)$, \\
$X_4 \le 7) \bagclose$.
%
The c-table for the overall result is as shown in Example \ref{ex:intro}. 
%
\punto
\end{example}


Note that  a tuple can be removed  from a c-table if  its condition is
inconsistent.   Conditions  can   become  inconsistent   by  combining
contradictory conditions  using conjunction,  which may happen  in the
implementations of the operators selection, product, and difference.

A condition is consistent if there is a variable assignment that makes
the condition true. For general boolean formulas, deciding consistency
is computationally  hard. But we do  not need to decide  it during the
evaluation of relational  algebra operations.  Rather, we exploit straightforward cases of inconsistency to clean-up c-tables and reduce their sizes.
We rely on the later 
Monte Carlo simulation phase to enforce the remaining inconsistencies.
%
\begin{enumerate}
\addtolength{\topsep}{-0.3ex}
\addtolength{\labelsep}{-0.3ex}
\addtolength{\itemsep}{-1ex}
\item The consistency of conditions not involving variable values is always immediately apparent.
\item Conditions $X_i = c_1 \land X_i = c_2$ with constants $c_1 \neq c_2$ are always inconsistent.
\item Equality conditions over continuous variables $Y_j = (\cdot)$, with the exception of the identity $Y_j = Y_j$, are not inconsistent but can be treated as such (the probability mass will always be zero).  Similarly, conditions $Y_j \neq (\cdot)$, with the exception of $Y_j \neq Y_j$, can be treated as true and removed.
\item Other forms of inconsistency can also be detected where it is efficient to do so.
\end{enumerate}

With  respect to  discrete variables,  inconsistency detection  may be
further simplified.  Rather than using abstract representations, every
row containing  discrete variables  may be exploded  into one  row for
every possible valuation.  Condition atoms matching each variable to its
valuation  are used  to ensure  mutual exclusion  of each  row.  Thus,
discrete variable columns may be  treated as constants for the purpose
of  consistency checks.  As  shown in  \cite{AJKO2008}, deterministic
database  query optimizers  do  a satisfactory  job  of ensuring  that
constraints over discrete variables are filtered as soon as possible.

Given  tables  in which  all  conditions  are  conjunctions of  atomic
conditions and  the query does not employ  duplicate elimination, then
all conditions  in the output  table are conjunctions.  Thus  it makes
sense to particularly optimise this scenario \cite{AJKO2008}.
In the case of positive
relational algebra  with the duplicate elimination  operator (i.e., we
trade  duplicate   elimination  against  difference),   we  can  still
efficiently  maintain  the  conditions  in  DNF,  i.e.,  as  a  simple
disjunction of conjunctions of atomic conditions.

Without loss of generality, the model can be limited to conditions that
are conjunctions of
constraint  atoms.  Generality  is maintained by  using bag  semantics to
encode disjunctions. This  restriction   provides   several  benefits.
First,
constraint  validation is  simplified;  A pairwise  comparison of  all
atoms in the clause is  sufficient to catch the inconsistencies listed
above.   As an additional  benefit, if  all atoms  of a  clause define
convex and  contiguous regions in the space $\vec{x},\vec{y}$, these same
properties are also shared by their intersection.
%This benefits  sample generation in ways that will be discussed later.




\subsection{Probabilistic c-tables; expectations}
\label{sec:montecarlo}


A \textit{probabilistic  c-table} is a c-table in  which each variable is
simply considered a (discrete  or continuous) {\em random variable}\/,
and a joint probability distribution is given for the random variable.
As  a convention,  we will  denote  the discrete  random variables  by
$\vec{X}$ and the continuous ones by $\vec{Y}$.  Throughout the paper,
we  will  always  assume  without  saying that  {\em  discrete  random
variables have a finite domain}\/.

We    will   assume    a    suitable   function    $p(\vec{X}=\vec{x},
\vec{Y}=\vec{y})$ specifying a joint distribution which is essentially
a  PDF  on the  continuous  and a  probability  mass  function on  the
discrete variables.  To clarify  this, $p$ is such that
we can define the expectation of a function $q$ as
\[
E[q] =
\sum_{\vec{x}} \int_{y_1} \cdots \int_{y_n}
p(\vec{x}, \vec{y}) \cdot q(\vec{x}, \vec{y}) \; d \vec{y}
\]
and approximate it as
\[
\frac{1}{n} \cdot \sum_{i=1}^n q(\vec{x}_i, \vec{y}_i)
\]
given samples $(\vec{x}_1, \vec{y}_1), \dots, (\vec{x}_n, \vec{y}_n)$ from
the distribution $p$.

%ck: That's BS, moments are covered by expectations.
%Throughout this paper, we will consider expectations and their special
%cases (such as event probabilities), but we will not discuss higher moments.
%Note though that the extension of our framework is conceptually simple.


We can specify events (sets of possible worlds) via Boolean conditions
$\phi$  that  are true  on  a  possible  world (given  by  assignment)
$\theta$  iff the condition  obtained by  replacing each  variable $x$
occurring  in  $\phi$  by  $\theta(x)$ is  true.   The  characteristic
function  $\chi_\phi$  of condition  (event)  $\phi$  returns  1 on  a
variable  assignment  if  it   makes  $\phi$  true  and  returns  zero
otherwise.   The probability  $\Pr[\phi]$  of event  $\phi$ is  simply
$E[\chi_\phi]$.

The expected  sum of a function $h$  applied to the tuples  of a table
$R$,
\begin{verbatim}
select expected_sum(h(*)) from R;
\end{verbatim}
can be computed as
\[
E \Big[ \sum_{\vec{t}  \in R}  h(\vec{t}) \Big]  =
E \Big[ \sum_{(t, \phi) \in C_R} \chi_\phi \cdot h(t) \Big] =
\sum_{(t, \phi) \in C_R} E \Big[ \chi_\phi \cdot (h \circ t) \Big]
\]
(the latter by linearity of expectation).
%, or equivalently
%\[
%   \sum_{(t,  \phi) \in  C_R} \sum_{\vec{x}}  \cdot  \int_{y_1} \cdots
%   \int_{y_n}  p(\vec{x}, \vec{y})  \cdot  \chi_\phi(\vec{x}, \vec{y})
%   \cdot h(t[\vec{x}, \vec{y}]) \; d \vec{y}.
%\]
Here $t(\vec{x}, \vec{y})$ denotes
the tuple $t$, where any variable that may occur is replaced by
the value assigned to it in $(\vec{x}, \vec{y})$.


\begin{example}\em
Returning to our running example, for $C_R = \bagopen (x_1, x_2 \le 7) \bagclose$, the expected sum of prices is
\begin{multline*}
   \sum_{(t,  \phi) \in  C_R} \cdot  
   \int_{x_1} \cdots
   \int_{x_4}  p(\vec{x})  \cdot  \chi_\phi(\vec{x})
   \cdot t(\vec{x}).\mathrm{Price} \; d \vec{y}
= \\
   \int_{x_1} \cdots
   \int_{x_4}  p(\vec{x})  \cdot  \chi_{X_2 \le 7}(\vec{x})
   \cdot x_1 \; d \vec{y}.
\end{multline*}
\punto
\end{example}


{\bf Counting and group-by}\/.
Expected count aggregates are obviously special cases of expected sum
aggregates where $h$ is a constant function $1$.
We generally consider expected sum aggregates with grouping by (continuously)
uncertain columns to be of doubtful value.
Group-by on nonprobabilistic columns (i.e., which contain no random variables)
poses no difficulty in the c-tables framework: the above summation simply
proceeds within groups of tuples from $C_R$ that agree on the group columns.
In particular, by delaying any sampling process until after the relational
algebra part of the query has been evaluated on the c-table representation,
we find it easy to create as many samples as we need for each group in a 
goal-directed fashion. This is a considerable strong point of the c-tables
approach used in PIP.




\section{Monte Carlo Sampling and Integration}
\label{sec:sampling}


As has already been pointed out,
both the computation of moments and probabilities reduces to numerical
integration, and a dominant technique for doing this is Monte Carlo simulation.
The approximate computation of expectation
$E[\chi_\phi \cdot (h \circ t)]$,
\begin{equation}
\frac{1}{n} \cdot \sum_{i=1}^n p(\vec{x}_i) \cdot \chi_\phi(\vec{x}_i) \cdot
h(t(\vec{x}_i)),
\end{equation}
faces a number of difficulties.
This section describes how these are addressed in PIP.

{\bf Basic sampling from $p$}.
It may be hard to sample from $p$.
Pip implements MCMC (in the form of the Metropolis algorithm),
a very powerful and general technique for computing samples, essentially based
on random walks in a Markov Chain biased by $p$. Of course, this technique
has a considerable overhead and is avoided where possible.

{\bf Independent random variables}.
In many special cases, there are better techniques, such as
when the random variables are mutually independent (i.e.,
$p_{X_1 \dots X_k}(x_1, \dots, x_k) = \prod_{i=1}^k p_{x_i}(x_i)$).
In that case, we sample independently from the various constituent
distributions $p_{X_i}$.  It should also be noted that constraint atoms may 
be treated as derived boolean random variables.  When the value of the constraint is
fixed (eg, when sampling under a selective condition) these atoms introduce
dependencies between their component random variables.

{\bf Special sampling techniques}.
Efficient direct sampling (not using MCMC) is possible for a constituent
distribution $p_{X_i}$ if we have  an inverse CDF $P^{-1}$. This allows
us to draw a sample from $p_{X_i}$ by drawing a sample $x$ uniformly from [0,1]
and computing $P^{-1}(x)$.
We can precompute and materialize (inverse) CDFs for later use using Monte
Carlo integration itself.

Special and very efficient sampling methods are also known
for certain  well-studied distributions.
For example, for the normal distribution, samples can be efficiently drawn
using the Box-Muller transform (implemented in PIP), the Ziggurat algorithm, etc.

{\bf Sparseness of samples for selective conditions}.
Samples for which $\chi_{\phi}$ is zero do not contribute
to an expectation.
If $\phi$ is a very selective condition, most samples do not
contribute to the summation computation of the
approximate expectation.
While we do not strictly employ rejection sampling here --
samples for with $\chi_{\phi}$ is zero count towards the number or samples
$n$ by which we average -- information can get very sparse and the approximate
expectations have a high relative error.
(This is closely related to the most prominent problem in online
aggregation systems \cite{OnlineAggregation,DBO}, and also in MCDB).

In the context of discrete probabilistic databases, the Karp-Luby algorithm
addresses exactly this problem. It is a fully polynomial-time randomized
approximation scheme for the probability of a DNF condition. As argued
before, for positive relational algebra, assuming DNF conditions means no loss
of generality or efficiency.





%%%%%%%%%%%%


\medskip

We now study some of these aspects in more detail.
In particular, we develop a Karp-Luby style algorithm for the continuous case.
The strength of this algorithm is that it only requires us to be able to approximate the probabilities of {\em conjunctive}\/ conditions and sample from the
joint distribution to efficiently approximate the probability of a DNF
(which is important when we want to compute probabilities or
expectations of aggregates after {\em duplicate elimination}\/).

%ck: actually that's not really right...
%It may seem that if MCMC is used for integration, it does not matter much
%whether the condition is conjunctive or a DNF because conditions are only
%checked once a sample has been created. However, MCMC is inefficient and
%to be avoided where possible. If random variables are independent and
%CDFs are available, we can efficiently sample from


\subsection{Constrained Sampling}
\label{subsec:csampling}


Though  PIP   requires  that   all  distribution  classes   provide  a
general-purpose  sampling  routine, it  is  periodically necessary  to
sample a variable from a subset of its range.  For example, consider a
row  containing  the variable  $Y \sim Normal(5,10)$  and the  condition
atoms $(Y >  -3)$ and $(Y < 2)$.  The expectation  of the variable $Y$
in the  context of  this row  is not $5$.   Rather the  expectation is
taken only over values of $Y$ that fall in the range $(-3,2)$.

More generally,  PIP requires  the ability to  sample values  that are
constrained   by   boolean   formulas   of  condition   atoms.    This
functionality is used both when  computing expectations and as part of
the Karp-Luby  estimator.  For conjunctions of atoms,  this problem is
equivalent to sampling  from a contiguous subset of  the sample space.

\nop{
The decision process used for this purpose is shown in Figure
\ref{fig:sample_decision}. 


\begin{figure}
\begin{center}
\resizebox{2in}{!}{\includegraphics{graphics/sampling_decision.pdf}}
\caption{The constrained sampling decision process}
\label{fig:sample_decision}
\end{center}
\end{figure}
} % end nop

The  most  straightforward approach  to  this  problem  is to  perform
rejection sampling; sample sets  are repeatedly generated until one is
found  that  satisfies  the   constraint  formula.   However,  as  the
probability  of satisfying  all  of  the atoms  drops,  the number  of
rejected  samples  grows.  Thus  the  cost  of  rejection sampling  is
inversely proportional to the probability of satisfying all the atoms.

If the inverse  CDF is available for a given variable,  it may be used
to rapidly generate samples  within specified bounds.  The inverse CDF
is effectively a mapping from  the domain $(0,1)$ to the corresponding
value  in  the given  distribution.   Furthermore,  the CDF  increases
monotonically.
$$[x > y] \rightarrow \left[CDF^{-1}(x) > CDF^{-1}(y)\right]$$

Thus, to obtain samples constrained  to  $(lower, upper)$, we
sample  $CDF^{-1}(X)$  where X  is  chosen  uniformly  from the  range
$(CDF(lower), CDF(upper))$.   In the  unlikely event that  the inverse
CDF is  available, but  the CDF  is not, this  technique may  still be
used.  Instead of sampling  from the range $(CDF(lower), CDF(upper))$,
we instead sample  $x \in (L,H)$ where $L$ is  initialized to $0$, and
$H$ is initialized to $1$.  If $CDF^{-1}(x) \leq lower$ we set $L = x$
and try again.  Similarly, if $CDF^{-1}(x)  \geq upper$ we set $H = x$
and  try again.   In  this way,  we  effectively learn  the values  of
$CDF(lower)$ and $CDF(upper)$.

When the inverse CDF is not available, naive sampling is typically the
most  efficient  approach.  The  probability  of  a  set of  variables
satisfying  a set  of atoms  is increased  as the  number of  atoms is
reduced.  As in conjunctive integration, it is possible to improve the
success  rate  by  separately  generating  samples  for  each  minimal
independent  subset.  Because  the  subsets are  smaller,  it is  less
likely that a sample will  need to be discarded.  Furthermore, because
fewer variables are being  sampled for each subset, less computational
effort is wasted generating invalid samples.

\subsection{MCMC Sampling}

A final alternative available to PIP, albeit a rarely used one, is the
Metropolis  algorithm \cite{metropolis}.   Starting from  an arbitrary
point within the sample space,  this algorithm performs a random walk.
Steps  are  sampled  from  a  multivariate  normal  distribution,  and
rejection sampling  is used  to weight the  walk towards  regions with
higher  probability  densities.  Samples  taken  at regular  intervals
during the random walk may be used as samples of the distribution.

Because the Metropolis  algorithm uses relative probability densities,
it is possible to use a density function that has not been normalized.
Thus, regions of  space that do not satisfy the  clause are assigned 0
density; the random walk never enters these regions.

The Metropolis algorithm has an  expensive startup cost, as there is a
lengthy  `burn-in' period  while  it generates  a sufficiently  random
initial  value.  Despite  this startup  cost, the  algorithm typically
requires only a relatively small  number of steps between each sample.
Consequently,   the  Metropolis  algorithm   is  ideally   suited  for
generating large numbers of samples  when the CDF is not available and
the probability of sampling a given value is small.

It necessary to parametrize the normal distribution used to select the
next step in the random walk.  Specifically, the standard deviation is
highly dependent on the distribution being sampled from;  If the value selected is
too small,  steps taken  by the  algorithm will be  too small  and the
number  of  steps required  to  generate  independent samples  becomes
large.   If  the value  selected  is  too  large, the  algorithm  will
frequently attempt to leave the  constraint bounds and thus many steps
will be rejected.

PIP uses the burn-in phase to select an appropriate standard deviation
for  each  dimension.   Prior  to  sampling, PIP  performs  a  dynamic
standard deviation  computation.  Steps are limited  to movement along
one axis;  thus only one  variable contributes to the  step direction.
If  the step  is rejected,  the standard  deviation for  that  axis is
lowered by a  constant factor.  If the step  is accepted, the standard
deviation is raised by a  constant factor.  These factors are selected
such  that the  standard  deviation  converges to  a  point where  the
acceptance-rejection ratio falls in  the commonly accepted ideal range
of  $0.1-0.4$\cite{numericalrecipes}.  The  dynamic standard  deviation computation  does not
replace the burn-in  phase, but instead reduces the  number of burn-in
steps required.



\subsection{Computing Confidences}
\label{subsec:cint}


Computing the confidence of a row in the C-Table; ie, computing the 
probability that the conjunction $\phi$ of the row's condition atoms, 
is equivalent to computing
\[
\sum_{\vec{x}} \int_{\vec{y}} p(\vec{x}, \vec{y}) \cdot \chi_\phi(\vec{x},\vec{y}) \; d\vec{y}.
\]
This integral may be naively estimated via Monte Carlo sampling as
described in Section \ref{sec:montecarlo}.
The difficulty is twofold: First, it must be possible to sample from $p$.
Second, enforcing $\chi_\phi$ requires rejection sampling, which can be
very inefficient if $\phi$ is selective.




%ck: where is the optimization? which choice do we have?
\nop{
All relevant conditions are  known prior to integration.  This advance
knowledge  may be  used to  produce more  accurate estimates  at lower
costs.

The first  optimization stems  from the observation  that the  cost of
Monte Carlo integration is inversely  proportional to the scale of the
value being  computed; in order  to achieve estimates  with equivalent
precision\footnote{It   is  important   to  distinguish   between  two
different precision metrics: the  number of significant figures in the
result as  opposed to the number  of decimal places.   This paper uses
the former  metric, referring to  the latter as  the \textit{absolute}
precision.} fewer iterations are required if the value being estimated
is large.   When estimating the  expectation $E[g]$, the scale  of the
computed  probability drops  as  the number  of constrained  variables
grows.  Thus, fewer  total samples will be needed  to achieve a target
precision  if  it is  possible  to  sample  subsets of  the  variables
independently.
} % end nop


{\bf Exploiting independence.}
To minimize the number of variables being integrated at one time, PIP first subdivides constraint atoms into minimal independent subsets.  Two constraint subsets are independent if their member atoms have no variables in common.  When determining subset independence, composite random variables (for instance, defined by artithmetic expressions over random variables) are treated as the set of all of their component variables.  By definition, atoms in each subset are independent.  Thus, the probability of each subset may be computed independently as well; the overall probability is the product of the independent probabilities.  For example, consider the one row c-table 
\[
\begin{tabular}{c|c}
R & $\phi_2$ \\
\hline
& $(X > 4) \wedge ([X\cdot Z] > Y) \wedge (A < 6)$ \\
\end{tabular}
\]
In this case, the atoms $(X > 4)$ and $([X\cdot Z] > Y)$ form one minimal independent subset, while $(A < 6)$ forms another.

Because condition atoms describing discrete variables are all of the form $Var = Val$, discrete variables are handled trivially.  Inconsistent values have already been removed, so the probability for the entire subgroup is the probability of the listed variable assignment.

The simplest subset of continuous atoms is one that references only one variable.  In this case, the atoms in the set provide constant upper or lower bounds, and the integration problem may be solved by evaluating the variable's CDF at the tightest upper and lower bounds.  If the CDF is not available or if it is not possible to derive tight bounds on the CDF, PIP can still integrate via Monte Carlo sampling.  In the one-variable case, numerical integration of the variable's PDF could also be used where an extremely precise answer is required.

{\bf Sampling using inverse CDFs}.
With more than two variables in the independent subset, Monte Carlo sampling becomes the most effective way of estimating the subset's probability.  As has already been noted, Monte Carlo techniques perform poorly if the value being computed is small.  However, in some cases PIP may be able to use a variable's CDF to reduce the sampling area.  

As discussed in Section \ref{subsec:csampling}, inverted CDFs allow efficient sampling of bounded variables.  For each variable in the subset where both a CDF and an inverted CDF are available, PIP computes the variable's upper and lower bounds from the atoms in the subset.  This includes both direct constraints of the form $X > C$, where $X$ is bounded on the bottom by $C$, and pairwise constraints of the form $X + Y < C_1$ and $X - Y < C_2$, where $X$ is bounded on the top by $\frac{C_1+C_2}{2}$.  

Applied to all the variables in the subset that have both a CDF and an inverted CDF, this process creates a hyper-rectangular bounding box in the sample space.  Because rectangular bounds are independent, the probability of a sample falling within the bounds can be computed independently for each variable as above.  Finally, Monte Carlo integration is performed, but only within the subspace.
%
%ck: i think not
%The result of this integration is normalized by the probability of a sample
%falling within the rectangular bounds.
%
Because  sampling is  constrained  to the  bounded  area, Monte  Carlo
integration avoid rejection sampling, and requires fewer samples for a
higher precision.

This  process is  only possible  if rectangular  bounds on  the sample
space exist.  However, it is  most useful when the probability density
contained within the atoms is  small.  Though it is possible to define
non-convex   constraint   atoms,   all   linear  atoms   are   convex.
Furthermore, the space  defined by the conjunction of  a set of convex
atoms is itself  convex.  Thus, anecdotally it is possible to
define a bounding box containing no less than half of the area defined
by the constraints.


There are  cases where bounds are insufficient.   For example, concave
atoms  are   not  likely   to  admit  effective   rectangular  bounds.
Similarly, even though a bounding box  covers no less than half of the
volume of a contiguous convex constraint area, the  bulk of the  probability mass may
still lie  outside of the constrained  sample area.  In  such cases, a
recursive technique may be applied.

The  bounding box  is first  subdivided into  smaller  regions.  Monte
Carlo integration  is performed on the  region twice, but  with only a
small number of iterations apiece.  If the two results agree to within
the  desired  precision, integration  stops  and  the  average of  the
results is multiplied by the  probability of a sample falling into the
sampling region.  If the  two results differ significantly, the region
is further subdivided and the algorithm recurses on each sub-region.

Because the  recursive cutoff is determined by  the estimated accuracy
of the  result, this  algorithm will not  recurse on regions  that are
entirely   within  or   outside  of   the  constrained   sample  area.
Consequently, the majority of  samples generated by the algorithm will
be  put towards estimating  relatively high  values where  Monte Carlo
integration is most effective.


\subsection{Karp-Luby Algorithm in the Continuous}


The number  of condition atoms in a  given row  is linear in  the query
complexity;  every renaming operator  adds a  fixed number  of condition
atom  columns to the  table, while  a join  simply merges  the condition
columns  of both  input tables.   Thus, the  size of  each conjunctive
clause is fixed with respect to the amount of input data and typically
relatively small.   However, there is no  such bound on  the number of
conjunctive  clauses  in a DNF.   Thus,  while  it  is
reasonable for  PIP to perform conjunctive integration  in memory, PIP
must use the disk to store intermediate state when computing integrals
of DNF formulas.

Two techniques may  be used to estimate the general  integral of a set
of conjunctive clauses of atoms.  Under naive Monte Carlo integration,
PIP first generates a set of samples and performs a linear scan of the
conjunctive clauses to determine how many samples are true in at least
one conjunctive clause.  The  number of samples required is determined
both by  the scale  of the answer  and the desired  precision.  Though
straightforward,  this  approach is  inefficient  if  the value  being
computed is small.

An  alternative approach  is a  Karp-Luby style  estimator, as
used  in the discrete case in \cite{RDS07, KO2008}.   This  approach first  computes  the
independent probability of each  conjunctive clause.  The sum of these
probabilities, termed the bag sum, is computed and stored.  Though the
bag sum is  related to the integral, the two are  not equal unless the
conjunctive clauses  are all mutually exclusive.  If  there is overlap
between clauses,  the bag sum will  exceed the value  of the integral.
The Karp-Luby estimator computes  this overlap, generating an estimate
of the ratio of the integral to the bag sum.

After computing the  bag probabilities and the bag  sum, the Karp-Luby
estimator  generates  samples  for  each clause,  constrained  to  the
subspace  it defines.   Each clause  is used  to produce  a  number of
samples  proportional  to  its  contribution  to  the  bag  sum.   The
generated samples  are compared agaainst  all clauses.  The  number of
clauses that  satisfy a given sample  are counted, and  the average of
the inverse of the counts is the ratio of the integral to the bag sum.
This value is multiplied by the  bag sum to produce an estimate of the
integral.  This algorithm is summarized in Figure
\ref{fig:klestimator}


\begin{figure}
\begin{center}
\begin{enumerate}
\item Perform a linear scan of the clauses $C$ of the disjunction, computing $P[\vec{\phi_C}]$.  As part of the same scan, compute $bag\_sum = \sum_C P[\vec{\phi_C}]$.  
\item Perform another linear scan, this time generating $total\_samples \cdot \frac{P[clause]}{bag\_sum}$ samples $\vec{S}$. 
\item Perform a final linear scan over $\vec{C} \bowtie \vec{S}$.  Compute the array $Sat[S] = \sum_C \chi_{\vec{\phi_C}}(S)$
\item The average value $\frac{1}{|\vec{S}|} \sum_{S}\frac{1}{Sat[S]}$ is an estimator for the ratio of the integral to the bag sum. 
\end{enumerate}
\caption{The Karp-Luby Estimator}
\label{fig:klestimator}
\end{center}
\end{figure}


There  is an  evident  tradeoff between  these  two processes.   naive
sampling is  faster when constrained  sampling cannot  be efficiently
performed.   However,  even  in  this case,  the  Karp-Luby  estimator
provides a more consistent precision; it generates a consistent number
of ``useful'' samples, even though  doing so is more expensive.  These
processes are summarized in Figure \ref{fig:integration}.


\begin{figure}
\begin{center}
\includegraphics[width=3.2in]{graphics/sampling_flowchart.pdf}
\end{center}

\vspace{-5mm}

\caption{\textbf{General Integration}}
\label{fig:integration}
\end{figure}




\section{Design of the PIP System}
\label{sec:design}
The goal of PIP is to evaluate queries on variables sampled from both discrete and continuous distributions as well as to provide tools to aid in the statistical analysis of those results.  Uncertainty in PIP is represented via the pVar datatype.  Every instance of this datatype represents a random variable.  When instantiating a pVar, users specify both a distribution for the variable to be sampled from, and a parameter set for that distribution.  

In addition to representing uncertainty in values for individual cells in a table, PIP represents per-tuple uncertainty with c-tables.  Each tuple is tagged with a condition that must hold for the variable to be present in the table.  C-table conditions are expressed as a boolean equation of \textit{atoms}, arbitrary inequalities of pVars.  The independent probability, or \textit{confidence} of the tuple is the probability of the condition being satisfied.  

\subsection{Random Variables}
  A pVar, or probabilistic variable, represents a nondeterministic value in the database.  Every pVar is created from by specifying a distribution and a set of parameters for that distribution.  For example, we write $[X=>Normal(\mu,\sigma^2)]$ to represent a pVar named X that follows a Normal distribution with a mean of $\mu$ and a standard deviation of $\sigma^2$.  In this way, PIP is agnostic to the distribution which a variable is sampled from; arbitrary problem-specific distributions may be created and seamlessly integrated into PIP's infrastructure.  

When defining a distribution, programmers need only include a mechanism for sampling from that distribution.  However, if it is possible to efficiently compute or estimate the distribution's probability density function ($PDF$), cumulative distribution function ($CDF$), or inverse cumulative distribution function ($CDF^{-1}$), these may be included to increase PIP's efficiency.  The process of defining a variable distribution is described further in Section \ref{sec:implementation}.  

Though PIP abstracts the details of the distribution from its query evaluation, it distinguishes between discrete and continuous distributions.  As described in Section \ref{sec:background}, existing research into c-tables has demonstrated efficient ways of querying variables sampled from discrete distributions.  PIP employs similar techniques when it is possible to do so.

Multiple pVars may be combined into an algebraic formula to create a composite pVar.  Composite pVars may be used to define joint distributions, to build more complex distributions, or may be created as a side effect of query evaluation.  Though composite pVars are interchangeable with regular pVars in general, there are several instances where they receive special treatment.  In particular, it is necessary to ensure that within a given sample all instances of a given pVar assume the same value; the value of a composite pVar is obtained only after its component pVars have been sampled.

Finally, constraint atoms are conditional formulas of pVars.  Such formulas may be equalities if all variables in the fomula are discrete, otherwise they must be an inequality; the probability of a continuous variable adopting an exact value is zero.  Given a constraint atom $C$, its probability $P[C]$ is the probability of selecting variable assignments that satisfy the formula.  Constraint atoms are used to express a tuple's condition.  By adding constraint atom columns to a table, a tuple's presence in the table becomes dependent on the constraint atom being true.  

Multiple atoms may be joined by boolean operators to create a constraint formula.  PIP expresses constraint formulas as DNF equations.  Multiple atoms in a tuple are evaluated conjunctively; the tuple is present if and only if all the atoms are true.  Multiple tuples sharing the same key value or values are used to express disjunctions; the key value is present in the output if any one tuple with that key is present.

\subsection{Variable Definition}
Before evaluating a query on random variables, the variables must first be defined.  This process takes one of two forms.  PIP uses a repair-key operator similar to that used in \cite{KochMayBMS2008} to define discrete distributions.  Conceptually, this operator identifies tuples that share key values and ensures that only one of the two tuples is present in the database at any given moment.  

Mutual exclusion is ensured by adding a constraint atom of the form $Var = Val$, where each key is assigned its own discrete variable, and all tuples sharing a key have distinct values.  Repair-key may be parametrized with an initial probability value for each tuple.  The sum of these probabilities may not exceed one for a given key.  If the sum is less than one, the remainder indicates the probability that none of the tuples are in the table.

Continuous variables may be created inline with the create-variable operation.  This function takes a distribution class and parameters, and outputs a new variable.  For example, the following query outputs a variable delivery time for a given order.

\begin{verbatim}
select orders.order_id, orders.item_id,
       create variable (`Normal', params.mean,
         params.std_dev) AS delivery_time
from   orders, params
where  orders.item_id = params.item_id;
\end{verbatim}

\subsection{Query Evaluation}
Query evaluation proceeds in two phases: Query and Sampling.  During the query phase, PIP evaluates a query rewritten to employ the c-tables relational algebra extensions described in Section \ref{sec:background}.  Selection clauses not involving pVars are handled traditionally, while those involving one or more pVars instead tag the output tuple with an equivalent condition clause.  For example, the query:

\begin{verbatim}
select *
from   input_table
where  fixed_column = 3
 and   4 > variable_column;
\end{verbatim}
%
is rewritten to
%
\begin{verbatim}
select *, constraint(`4 > variable_column')
from   input_table
where  fixed_column = 3;
\end{verbatim}

Queries are also modified to ensure that these newly created constraint columns are not projected away.

All selections that generate composite columns including at least one pVar are replaced by composite pVars; evaluation of these columns is effectively delayed until the sampling phase.  For example, the query:

\begin{verbatim}
select fixed_column + variable_column
from   input_table
\end{verbatim}
%
is rewritten to
%
\begin{verbatim}
select composite(+, fixed_column, variable_column)
from   input_table
\end{verbatim}

The rewritten query is evaluated by a deterministic database engine and produces a \textit{nondeterministic table}; the query's output contains constant values, pVars and condition atoms.   Note that discrete variable assignments are expressed entirely through constraint atoms; discrete variables are stored in the database as constants.  Because of this, the complexity of evaluating queries over discrete values is pushed entirely into the underlying deterministic database engine.  Thus, in the query phase deterministic variables are treated as constants with respect to continuous variables.

It is also important to note that the nondeterministic table is a \textbf{lossless} encoding of the relationship between the variables used in the query and the output table.  There is no bias to be perpetuated in further queries on the output table.  Consequently, the output of one query phase may be saved as a materialized view and used for arbitrary further computation.


\section{Implementation}
\label{sec:implementation}
In order to evaluate the viability of PIP's c-tables approach to continuous variables, we have implemented an initial version of PIP as an extension to the PostgreSQL DBMS as shown in Figure \ref{fig:blockdiag}.  %PIP's extended functionality is provided by a set of user-defined functions written in C.  

\begin{figure}
\begin{center}
\resizebox{1.8in}{!}{\includegraphics{graphics/blockdiag.pdf}}
\caption{The PIP Postgres plugin architecture}
\label{fig:blockdiag}
\end{center}
\vspace*{-0.3in}
\end{figure}

\subsection{Query Rewriting}
Much of this added functionality takes advantage of PostgreSQL's extensibility features, and can be used ``out-of-the-box".  For example, we define the function \vspace*{-0.1in}
\[
\mbox{\textbf{\footnotesize CREATE\_VARIABLE($distribution$[,$params$])}}\vspace*{-0.1in}
\]
which is used to create continuous variables\footnote{For discrete distributions, PIP uses a repair-key operator similar to that used in \cite{KochMayBMS2008}}.  Each call allocates a new variable, or a set of jointly distributed variables and initializes it with the specified parameters.  When defining selection targets, operator overloading is used to make random variables appear as normal variables; arbitrary equations may be constructed in this way.  

%\begin{example}\em
%Continuous variables may be created inline with the create-variable operation.  
%
%\begin{verbatim}
%select o.order_id, o.item_id,
%       CREATE_VARIABLE (`Normal', p.mean,
%         p.std_dev) AS delivery_time
%from   orders o, params p
%where  o.item_id = p.item_id;
%\end{verbatim}
%
%\end{example}

%Angle brackets around a random variable are shorthand for the variable's expectation.  All instances of this are replaced by a call to pip's expectation sampling function.  

To complete the illusion of working with static data, we have modified PostgreSQL itself to add support for C-Table constructs.  Under the modified PostgreSQL when defining a datatype, it is possible to declare it as a CTYPE; doing so has the following three effects:
\begin{itemize}
\item CTYPE columns (and conjunctions of CTYPE columns) may appear in the WHERE and HAVING clauses of a SELECT statement.  When found, the CTYPE components of clause are moved to the SELECT's target clause.
{\small\begin{verbatim}
  select *
  from   inputs
  where  X>Y and Z like '%foo'
\end{verbatim}}
is rewritten to
{\small\begin{verbatim}
  select *, X>Y
  from   inputs
  where  Z like '%foo'
\end{verbatim}}

\item SELECT target clauses are rewritten to ensure that all CTYPE columns in input tables are passed through.   The exception to this is in the case of special probability-removing functions.  If the select statement contains one or more such functions (typically aggregates, or the conf operator), CTYPE columns are not passed through.  
{\small\begin{verbatim}
  select X,Y
  from   inputs
\end{verbatim}}
is rewritten to
{\small\begin{verbatim}
  select X,Y,inputs.phi1,inputs.phi2,...
  from   inputs
\end{verbatim}}

\item In the case of aggregates, the mechanism by which CTYPE columns may be passed through is unclear.  Thus If the select statement contains an aggregate and one or more input tables have CTYPE columns, the query causes an error unless the aggregate is labeled as a probability-removing function.

\item UNION operations are rewritten to ensure that the number of CTYPE columns in their inputs is consistent.  If one input table has more CTYPE columns of a given type than the other, the latter is padded with NULL constraints.
{\small\begin{verbatim}
  -- left(X,phi1), right(X,phi1,phi2)
  select *
  from   left UNION right
\end{verbatim}}
is rewritten to
{\small\begin{verbatim}
  select *
  from   (select *,NULL AS phi2 FROM left
         ) UNION right
\end{verbatim}}

\end{itemize}

\begin{figure}
\begin{center}
\footnotesize
\begin{tabular}{c|ccc}
$R_{ctable}$ & $A$ & $B$ & $\phi$ \\ \hline
& $X*3$ & $5$ & $X > Y \wedge Y > 3$ \\
& $Y$ & $3$ & $Y < 3 \vee X < Y$ \\
\end{tabular}
\begin{center}
$\Downarrow \Downarrow \Downarrow$
\end{center}
\begin{tabular}{c|cccc}
$R_{int}$ & $A$ \begin{footnotesize}(VarExp)\end{footnotesize} & $B$ \begin{footnotesize}(integer)\end{footnotesize} & $\phi_1$ \begin{footnotesize}(CTYPE)\end{footnotesize}& $\phi_2$  \begin{footnotesize}(CTYPE)\end{footnotesize} \\ \hline
& $X*3$ & $5$ & $X > Y$ & $Y > 3$ \\
& $Y$ & $3$ & $Y < 3$ & NULL \\
& $Y$ & $3$ & $X < Y$ & NULL \\
\end{tabular}
\caption{Internal representation of C-Tables}
\label{fig:intrep}
\end{center}
\vspace*{-0.3in}
\end{figure}



Note that these extensions are not required to access PIP's core functionality; they exist to allow users to seamlessly use deterministic queries on probabilistic data.

PIP takes advantage of this by encoding constraint atoms in a CTYPE datatype; Overloaded $>$ and $<$ operators return a constraint atom instead of a boolean if a random variable is involved in the inequality, and the user can ignore the distinction between random variable and constant value (until the final statistical analysis).


\subsection{Defining Distributions}
PIP's primary benefit over other c-tables implementations is its ability to admit variables chosen from arbitrary continuous distributions.  These distributions are specified in terms of general distribution classes, a set of C functions that describes the distribution.  In addition to a small number of functions used to parse and encode parameter strings, each PIP distribution class defines one or more of the following functions.
\begin{itemize}
\footnotesize
\item \texttt{Generate(Parameters, Seed)} uses a pseudorandom number generator to generate a value sampled from the distribution.  The seed value allows PIP to limit the amount of state it needs to maintain; multiple calls to Generate with the same seed value produce the same sample, so only the seed value need be stored.
\item \texttt{PDF(Parameters, x)} evaluates the probability density function of the distribution at the specified point.  
\item \texttt{CDF(Parameters, x)} evaluates the cumulative distribution function at the specified point.
\item \texttt{InverseCDF(Parameters, Value)} evaluates the inverse of the cumulative distribution function at the specified point.
\end{itemize}

PIP requires that all distribution classes define a Generate function.  All other functions are optional, but can be used to improve PIP's performance if provided; The supplemental functions need only be included when known methods exist for evaluating them efficiently.

%Future implementations could conceivably generalize the sampling process.  A sample may be generated using any of the four functions: The Metropolis-Hastings algorithm can sample from an arbitrary PDF, the inverse CDF evaluated on a uniform random value produces a sample, and a binary search may be used to evaluate the inverse CDF given the CDF.

\subsection{Sampling Functionality}
PIP provides several functions for analyzing the uncertainty encoded in a c-table.  The two core analysis functions are conf() and expectation().

\begin{itemize}
\footnotesize
\item \texttt{conf()} performs a conjunctive integration to estimate the probability of a specific row's condition being true.  For tables of purely conjunctive conditions, conf() can be used to compute each row's confidence. 

\item \texttt{aconf()}, a variant of conf(), is used to perform general integration.  This function is an aggregate that computes the joint probability of all equivalent rows in the table, a necessity if disjunctions are in use.  

\item \texttt{expectation()} computes the expectation of a variable by repeated sampling.  If a row is specified when the function is called, the sampling process is constrained by the constraint atoms present in the row.

\item \texttt{expected\_sum()}, \texttt{expected\_max()} are aggregate variants of expectation.  As with expectation() they can be parametrized by a row to specify constraints.

\item \texttt{expected\_sum\_hist()}, \texttt{expected\_max\_hist()} are similar to the above aggregates in that they perform sampling.  However, instead of outputting the average of the results, it instead outputs an array of all the generated samples.  This array may be used to generate histograms and similar visualizations.
\end{itemize}

Aggregates pose a challenge for the query phase of the PIP evaluation process.  Though it is theoretically possible to create composite variables that represent aggregates of their inputs, in practice it is infeasible to do so.  The size of such a composite is not just unbounded, but linear in the size of the input table.  A variable symbolically representing an aggregate's output could easily grow to an unmanageable level.  Instead, PIP limits random variable aggregation to the sampling phase.  





\section{Evaluation}
\label{sec:evaluation}
As a comparison point for Pip's ability to manage continuous random variables, we have constructed a sample-first probabilistic extension to Postgres that loosely emulates MCDB's tuple-bundle concept using ordinary Postgres rows.  A sampled variable is represented using an array of floats, while the tuple bundle's presence in each sampled world is represented using a densely packed array of booleans.  In lieu of an optimizer, test queries were constructed by hand so as to minimize the lifespan of either array type.

We evaluated both the Pip C-Tables and the Sample-First infrastructure against three related queries.  Tests were run over a single connection to a modified instance of PostgreSQL 8.3.4 with default settings running on a 2x4 core 2.0 GHz Intel Xeon with a 4MB cache.  All queries were evaluated over a 1 GB database generated by the TPC-H benchmark.  Unless otherwise specified, all sampling processes generate 1000 samples apiece.  Unless otherwise specified, results shown are the average of 10 measurements run sequentially.  

The first set of tests evaluate Pip's performance on queries similar to those used in \cite{MCDB}.  The results of these tests are shown in Figure \ref{fig:querytimings}.  Performance times for Pip are divided into two components: query and sample, to distinguish between time spent evaluating the deterministic components of a query and building the result c-table, and time spent computing expectations and confidences of the results.

\begin{figure}
\begin{center}
\resizebox{3in}{!}{\includegraphics{graphics/query_timings.pdf}}
\caption{Query evaluation times in Pip and Sample-First}
\label{fig:querytimings}
\end{center}
\end{figure}

The first query computes the rate at which customer purchases have increased over the past two years.  The percent increase parametrizes a Poisson distribution that is used to predict how much more each customer will purchase in the coming year.  Given this predicted increase in purchasing, the query estimates the company's increased revenue for the coming year.

In the second and third queries, past orders are used to compute the mean and standard deviation of manufacturing and shipping times.  These values parametrize a pair of Normal distributions that combine to predict delivery dates for each part ordered today from a japanese supplier.  The second query computes the maximum of these dates, while the third compares the times against arbitrary customer satisfaction thresholds to generate a list of potential ``dissatisfied'' customers.

As expected, Pip's performance on these queries is comparable to the sample-first approach.  In the general case, Pip performs effectively the same computations as Sample-First, save that Pip delays sampling slightly longer.  In particular, queries 2 and 3 demonstrate how little time sampling can take for some queries; additional samples can be computed without incurring the nearly 1 minute query time.

The final timing test combines a simplified version of the prior two queries to simulate a risk management query that might be evaluated daily.  Given per-customer revenue forecasts, estimates for product manufacturing and shipping times, customer satisfaction thresholds, compute the expected revenue lost to customers dissatisfied with today's service.  For added realism, this query uses a precomputed table of part production and shipping times (which changes rarely enough that it can be updated monthly).  This query is shown in Figure \ref{fig:timingq4}.

This query includes two distinct, independent sampling components: the expectation of revenue from a customer, and the probability that a customer will be dissatisfied.  A studious user may note this fact and hand optimize the query to compute these values independently.  However, without this optimization, a sample-first approach will generate one pair of values for each customer for each world.  Because worlds where a customer is satisfied are not relevant to the query, an arbitrarilly large number of customer revenue values will be discarded.  Pip separates the computation of expectations and confidences and does not suffer from this problem.  

For this comparison, customer satisfaction thresholds were set such that an average of 10\% of customers were dissatisfied.  Consequently sample-first discarded an average of 10\% of its values.  To maintain comparable accuracies, the sample-first query was evaluated with 10,000 samples while the Pip query remained at 1000 samples.  

\begin{figure}
\begin{center}
\resizebox{3in}{!}{\includegraphics{graphics/iterative_refinement.pdf}}
\caption{Variance as a function of samples in Pip and Sample-First.  Each data point is an estimate generated by a single run at the indicated number of samples}
\label{fig:iterativerefinement}
\end{center}
\end{figure}


Figure \ref{fig:iterativerefinement} demonstrates one extreme case of this in a comparison between Karp-Luby estimates and Sample-First estimates.  The results shown are for repeated executions of a query similar to query 4, save with a filter that removes all but approximately 10 clients.  In queries that do not involve a large linear aggregate, the sample-first approach disqualifies a sufficient number of possible worlds that subsequent expectation computations falter.  Conversely the Karp-Luby estimator has sufficient information that it can employ a precomputed CDF lookup table to compute each row's bag probabilities.  Because of this and the fact that it generates more ``useful'' samples, its results have a much lower variance with far fewer samples required.

We have shown that it is possible to apply the c-tables approach to probabilistic databases with only minimal overhead, even when it is used to represent arbitrary variable distributions.  The additional information provided by the c-table adds a degree of flexibility that allows Pip to outperform Sample-First approaches in a number of instances.

\begin{figure}[t!]
%\footnotesize
\begin{footnotesize}
\begin{verbatim}
create table `shipping_params' as
  select 
    avg   (l_shipdate    - o_orderdate) as ship_mu,
    avg   (l_receiptdate - l_shipdate ) as arrv_mu,
    stddev(l_shipdate    - o_orderdate) as ship_sigma,
    stddev(l_receiptdate - l_shipdate ) as arrv_sigma,
    l_partkey as p_partkey
  from orders,lineitem
  where o_orderkey = l_orderkey
  group by partkey;
alter table params add constraint "p_partkey_pkey" 
  primary key (p_partkey);
-- BEGIN TIMING QUERY --
create temporary table q4_shipping as
  select o_orderkey AS orderkey, o_custkey AS custkey,
    CREATE VARIABLE(`Normal',ship_mu,ship_sigma)
    CREATE VARIABLE(`Normal',arrv_mu,arrv_sigma)
  from  orders,lineitem,shipping_params
  where p_partkey = l_partkey;
   and  o_orderdate = today()
   and  o_orderkey  = l_orderkey;
create temporary table q4_annoyed as
  select custkey from q4_shipping where ship > 120
union all
  select custkey from q4_shipping where arrv > 90;
create temporary table q4_order_increase as
  select o_orderkey, o_custkey,
     CREATE VARIABLE(`Poisson', increase) *
       l_extended_price * (1.0 - l_discount) as rev
  from (select newc / oldc as increase, custkey 
    from (select o_custkey as custkey, 
             sum(o_orderdate.year-1996.0) AS newc,
             sum(1997.0-o_orderdate.year) AS oldc
      where o_orderdate.year = 1997 
        or  o_orderdate.year = 1996
      group by custkey
     ) as counts
   ) as increase_per_cust,
    orders
  where custkey = o_custkey
 ) as var_increase_per_customer,
 (select lineitem.*,
  from  nation,supplier, lineitem, partsupp
  where n_name = 'japan' and n_nationkey = s_nationkey
   and  s_suppkey = ps_suppkey
   and  ps_partkey = l_partkey
   and  ps_suppkey = l_suppkey
 ) as items_from_japan;
-- BEGIN TIMING SAMPLE --
select avg(confidence),
       expected_sum_naive(rev, q4_annoyed)
from   q4_annoyed,
       (select o_custkey as custkey, rev
        from   q4_revenue_gains
       ) as revenues
where  revenues.custkey = q4_annoyed.custkey;
\end{verbatim}
\end{footnotesize}

\vspace{-5mm}

\caption{Timing Query 4}
\label{fig:timingq4}
\end{figure}


\section{Conclusion}

We have shown that it is possible to apply the c-tables approach to probabilistic databases with only minimal overhead, even when it is used to represent arbitrary variable distributions.  The additional information provided by the c-table adds a degree of flexibility that allows PIP to outperform Sample-First approaches in a number of instances.

\begin{small}
\bibliographystyle{abbrv}
\bibliography{bibtex}
\end{small}


\end{document}

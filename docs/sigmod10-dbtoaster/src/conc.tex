\section{Conclusion}
We presented \compiler, a query compilation framework that aggressively compiles
SQL aggregate queries to C++ code through the use of a novel recursive
compilation algorithm. We consider continuous aggregate queries in the context
of stream processing, and in particlar support query evaluation over an
arbitrary database subject to any pattern of updates. \compiler\ represents
queries in a map calculus where formulae correspond to map data structures and
recursively compiles formulae via a set of transformations and simplifications.
The key features of our algorithm include applying deltas to query input
relations, to construct incremental map maintenance terms, and that these terms
get progressively simpler as combinations of deltas are applied. The code
resulting from compilation performs tuple-at-a-time processing, and consists
solely of very simple map lookup and update statements.

This code is extremely lightweight and efficient, and in our experimental
evaluation, shows 1-4 orders of magnitude performance improvement for update
stream processing over compiled versions of both classical view maintenance
algorithms and naive recomputation. Indeed the comparison to actual database
systems shows an even larger gap.
\comment{
In fact both the view maintenance and standard processing styles
can be seen as special cases of recursive compilation, suggesting that
recursively compiled queries can span a range of incremental computation needs,
and may present an answer on how to bridge stream and ad-hoc query processing
engines.
}
There are many potential topics in this relatively unexplored area of
query compilation, embedded query processing, and database engine generation for
specialized domains. Some of the topics we are looking at include extracting
parallelism for multi-core operation, and distributing query processing and map
maintenance to cloud computing environments.
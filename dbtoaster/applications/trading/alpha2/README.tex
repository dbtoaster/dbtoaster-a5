\documentclass{letter}

%%%%%%%%%% Start TeXmacs macros
\newcommand{\section}[1]{\medskip\bigskip

\noindent\textbf{\LARGE #1}}
\newcommand{\tmtexttt}[1]{{\ttfamily{#1}}}
%%%%%%%%%% End TeXmacs macros

\begin{document}

\section{Running the project}

The different components of the project and what they can do as of now have
been explained in the Documentation.pdf. This document assumes that points
mentioned in that document are clear before proceeding with this one. There
are many scenarios one may want to test, and how to test may not have been
clear from the documentation. This document is to clear that. The following
are the scenarios that have been tested, and the method to do the same has
been mentioned.
\begin{description}
  \item[Simple Match Testing] One basic scenario is to check if the matcher
  runs correctly. i.e., the market parses input orders correctly, matches them
  according to specified rules, and produces an update stream. To run the
  virtual market, all that the coder needs to do is create a StockMarketServer
  object. This object essentially does the task of hosting a server at the
  address specified in its constructor. Then one needs to run a client class,
  that can connect to the server and exchange messages. One way of doing so is
  shown in test.TestClient.java. This class has a main, and a handler (which
  can be adjusted to print relevant details from the market's update stream)
  which can read in orders from the console and send them to the market.
  
  \item[Market Maker Testing] There is also a case where one may want to test
  a market maker. The example of how to do so is shown in
  test.MarketMakerTest.java. Once again the market is simulated by creating
  the StockMarketServer object. This hosts the market server. Then a historic
  stream is started, which replays data from the history into the market.
  Currently the code for syncing historic timestamp with current timestamp has
  been commented to allow heavier load of historic data. Once the historic
  stream is begun, the market maker object is created, and the execute()
  method of the market maker object is called to begin its function. Code for
  the testing is as simple as shown below:\tmtexttt{}
  
  \tmtexttt{StockMarketServer s = new StockMarketServer();
  
  HistoricTrader ht = new HistoricTrader(); ht.run();
  
  MarketMaker mm = new MarketMaker(0.1, 1000, 1000, 10101); mm.execute();}
  
  However it should be noted that the mm.execute() call is a blocking call,
  and should be made in the end, or through a thread if more than one Market
  maker is to be added to the market. (It can also be done through a different
  main(), or by adding a main() to the MarketMaker class, enabling it to be a
  stand alone process).
  
  \item[Algo Trader testing] Here there is not as much uniformity as there
  should be. Up till now all executions have been integrated into one, where
  all components have a method which allows them to start executing. However,
  for algo traders, they have a separate main, which essentially makes them a
  separate process. This same thing can be done for market makers as well, and
  indeed it would be cleaner for the codebase to handle both market makers and
  algo trader executions similarly. However for now, they are handled
  differently, and the coding sequence to test algo traders would be: Create
  the StockMarketServer object, and run the main of the algo trader you want
  to run. (you can also start the historic stream as done in the previous
  example). 
\end{description}


\end{document}



%\section{Introduction}


\longversion{
This paper calls for a new breed of systems which we call {\em dynamic data
management systems (DDMS)}. In a nutshell, think of a DDMS as (very large) {\em
dynamic data structures}\/ with {\em agile, frequently fresh views}\/, and {\em
a facility for monitoring these views and triggering applica\-tion-level
events}\/. DDMS could be relatively lightweight systems, or could be
functionality that eventually finds its way into more traditional DBMS or data
stream processors.
}


\section{Introduction}

\comment{
Often, the current state of a data point is not as interesting as how the data
changes over time.  Changes force reevaluation of prior questions and answers
and allow new conclusions to be drawn.
}
Dynamic, continuously evolving sets of records are a staple in a wide variety of
today's data management applications. Such applications range from large,
social, content-driven Internet applications, to highly focused data processing
verticals like data intensive science, telecommunications and intelligence
applications. There is no one brush with which we can paint a picture of all
dynamic data applications -- they face a broad spectrum of update volumes, of
update impact on the body of data present, and data freshness requirements.
However, modern data management systems either treat updates and their impact on
datasets and queries as an afterthought, by extending DBMS with triggers and
heavyweight views~\cite{griffin-sigmod:95, bello-vldb:98, zhou-vldb:07,
zhou-icde:07}, or only handle small, recent sets of records in data stream
processing~\cite{abadi-vldbj:03, demers-sigmod:07, motwani-cidr:03,
chandrasekaran-cidr:03}.






We propose dynamic data management systems that are capable of handling
arbitrary, high-frequency or high-impact updates on a general dataset,
and any standing queries (views) posed on the dataset by long-running
applications and services. We base the design of DDMS around four criteria:

\begin{enumerate}
\item
The stored dataset is large and changes frequently.

\item 
The
% computation of standing queries through the %incremental
maintenance of materialized views dominates ad-hoc querying.

\item
Access to the data is primarily by monitoring the views and performing
simple computations on top of them.

%
%; i.e., by reading out the views or by performing very simple queries
%on top of the views which can be evaluated in a small fraction of the
%time it would take to evaluate the view from scratch.
%
\comment{
Views may be structured (tables) or booleans (flags, events). Thus some updates
cause events, observable in the views, that trigger subsequent computations, but
it is rare that the data store is accessed asynchronously by humans or
applications.
}

Some updates cause events, observable in the views, that trigger subsequent
computations, but it is rare that the data store is accessed asynchronously by
humans or applications.

\item
Updates happen primarily through an {\em update stream}\/. Computations
triggered by view events usually do not cause updates: there is usually no
feedback loop.

\end{enumerate}

\noindent A DDMS is a lightweight system that provides large dynamic data
structures to support declarative views of data. A DDMS is \textit{agile},
keeping internally maintained views fresh in the face of dynamic data.  Client
applications primarily interact with a DDMS by registering callbacks for view
changes, rather than by accessing views directly. A DDMS does not necessarily
provide additional DBMS functionality such as persistency, transactions, or
recoverability.



\comment{
DDMS are lightweight systems that are optimized to {\em keep views fresh}\/ at
the greatest rates feasible given base data volumes, which may range from
moderate to extremely large, depending on application. In this section, we
discuss two quite different fields of application that justify DDMS.
}


Compared to a classical DBMS, a DDMS differs in its reaction to updates.  To
minimize response times, updates must be performed immediately upon arrival,
precluding bulk processing. This determines the programming model: compared to
a DBMS, control flow is reversed, and the DDMS invokes application code, not
vice versa.



An active DBMS~\cite{ceri-vldb:00} could simulate a DDMS through triggers, but
is not optimized for such workloads, and even if support for state-of-the-art
incremental view maintenance is present, performs very poorly. Thus, DDMS differ
from active databases in their being optimized for different workloads; DDMS are
optimized for event processing and monitoring tasks, while active database
systems are optimized to support traditional DBMS functionality not present in
DDMS, such as transactions.




Compared to a data stream processing system and particularly an event processing
system (such as Cayuga~\cite{demers-sigmod:07}, SASE+~\cite{agrawal-sigmod:08}),
DDMS have much larger states, which will usually have to be maintained in
secondary storage, and require drastically different query processing
techniques. In a stream processor, the queries reside in the system while the
data streams by. In a DDMS on the other hand, the data state is maintained in
the system while a stream of updates passes through (much more like an OLTP
system).

%
Moreover, event and stream processors~\cite{abadi-vldbj:03, motwani-cidr:03,
chandrasekaran-cidr:03} support drastically different query languages which are
designed to ensure that only very small state has to be maintained, using
windows or constructs from formal language theory~\cite{white-pods:07}. DDMS
views are often rather complex and expensive, including large non-windowed joins
and aggregation. In general, we expect DDMS to support standard SQL.
\comment{
A DDMS can be thought of as an {\em update stream}\/ processing system in which
query workloads look more like those of classical DBMS (i.e., no constructs such
as windows).}
The query processing techniques most suitable for such workloads come from DBMS
research -- incremental view maintenance in particular -- and update stream
research~\cite{ghanem-tods:10} but do not scale to high-frequency view
maintenance.


% After that we will flesh out DDMS in more detail.

\medskip

We present two examples of application classes motivating the desiderata and
design choices of a DDMS.

{\bf Large-scale data analytics, but not as a batch job}\/.
% with an interactive or soft real-time aspect
%
\comment{
Large-scale data analytics in the cloud -- using systems such as map/reduce --
make it painfully obvious that the data management research community has almost
missed a great opportunity for impact. While some aspects of such systems have
certainly been previously explored by the data management community, these
systems are not databases, as some strata of the systems, scientific computing,
and large-scale Web applications communities find important to emphasize.
Nevertheless, our research community can certainly make important contributions
towards making such systems more useful and effective. Clearly, the last word on
supporting {\em queries}\/ in such systems has not been said.

}
Large-scale data analytics in the cloud are mostly performed on massively
parallel processing engines such as map/reduce. These systems are not databases,
as some strata of the systems, scientific computing, and large-scale Web
applications communities find important to emphasize. Nevertheless, our research
community can play an important role in making such systems more useful and
effective. Clearly, the last word on posing {\em queries}\/ in such systems
has not been said.




Map/reduce-like systems achieve scalability at the cost of response time and
interactivity.
%
%For various reasons having to do with their programming model,
%the method of parallelizing, and of dealing with failures,
%such systems tend to be sluggish in producing their results. 
%
However, there is an increasing number of important applications of large-scale
analytics that call for more interactivity or better response times that allow
for online use. Among large Web applications, examples include (social or other)
network monitoring and statistics~\cite{olston-cidr:09}, search with interactive
feedback~\cite{bast-cidr:07}, interactive recommendations, keeping personalized
Web pages at social networking sites up to date~\cite{fu-sigmod:10}, and so
forth.

%
\longversion{
Many of these applications are not yet mission-critical to Web applications
companies, but they are increasingly of competitive advantage.

}

\comment{
The DDMS take on supporting large-scale data analytics is to provide large {\em
dynamic data structures}\/ and to support declaratively defined views on them
that, compared to key-value stores, increase programmer productivity. DDMS views
are {\em agile}\/ in that they are continually kept fresh.
}

Large-scale data analytics is equally present in more classical business
applications such as data warehousing and scientific applications. Take the
case of data warehousing with real-time updates: as data warehouses become
increasingly mission-critical to commercial and scientific enterprises, the
importance of up-to-date analyses increases. Traditionally, OLAP systems are not
optimized for frequent updating, and may be considerably out-of-date. A DDMS 
could dramatically improve the freshness of warehoused data.

A DDMS is well-suited to large-scale data analytics through its provision of
large dynamic data structures as views, instead of forcing programmers to
re-implement view computations manually on top of key-value stores, and its
emphasis on simple lightweight systems as opposed to the use of monolithic DBMS.
Continually fresh DDMS views at first seem at odds with the bulk update
processing dogma of large scale analytics systems, but enable important
applications that require interactivity or event processing.


\medskip


{\bf Algorithmic trading with order books.}\/
In recent years, algorithmic trading systems have come to account for a majority
of volume traded at the major US and European financial markets (for instance,
for 73\% of all US equity trading volume in the first quarter of 2009
\cite{Iati2009}). The success of automated trading systems depends critically on
strategy processing speeds: trading systems that react faster to market events
tend to make money at the cost of slower systems. Unsurprisingly, algorithmic
trading has become a substantial source of business for the IT industry; for
instance, it is the leading vertical among the customer bases for high-speed
switch manufacturers (e.g., Arista \cite{Becht2010}) and data stream processing.




A typical algorithmic trading system is run by mathematicians who develop
trading strategies and by programmers and systems experts who implement these
strategies to perform fast enough, using mainly low-level programming languages
such as C. Developing trading strategies requires a feedback loop of simulation,
back-testing with historical data, and strategy refinement based on the insights
gained. This loop, and the considerable amount of low-level programming that it
causes, is the root of a very costly {\em productivity bottleneck}\/: in fact,
the number of programmers often exceeds the number of strategy designers by
an order of magnitude.


\comment{
Trading algorithms often perform a considerable amount of data crunching
and statistical processing that could in principle be implemented using SQL
views, coupled with some relatively straightforward control and trading logic.


%
\longversion{
Differently from other areas of finance such as technical analysis,
where stream processing engines
\cite{abadi-vldbj:03,motwani-cidr:03} can be applied,
}
Data processing in trading algorithms using views cannot be performed by DBMS or
data stream processing systems today: the former are not able to (1) {\em update
their views at the required rates}\/ (for popular stocks, hundreds of orders per
second may be executed, even outside burst times) and the latter are not able to
(2) {\em maintain large enough data state}\/ and support suitable query
languages (non-windowed SQL aggregates) on this state.

%
A data management system that could handle these two requirements would yield a
very substantial productivity increase that can be directly monetized -- the
holy grail of algorithmic trading.
}

Trading algorithms often perform a considerable amount of data crunching that
could in principle be implemented as SQL views, but cannot be achieved by DBMS
or data stream processing systems today: DBMS are not able to (1) {\em update
their views at the required rates}\/ (for popular stocks, hundreds of orders per
second may be executed, even outside burst times) and stream engines are not
able to (2) {\em maintain large enough data state}\/ and support suitable query
languages (non-windowed SQL aggregates) on this state.
A data management system fulfilling these two requirements would yield a very
substantial productivity increase that can be directly monetized -- the holy
grail of algorithmic trading.



To understand the need to maintain and query a large data state, note that
many stock exchanges provide a detailed view of the market microstructure
through complete bid and ask {\em limit order books}. The bid order book is a
table of purchase offers with their prices and volumes, and correspondingly the
ask order book indicates investors' selling orders. Exchanges execute trades by
matching bids and asks by price and favoring earlier timestamps. Investors
continually add, modify or withdraw limit orders, thus one may view order books
as relational tables subject to high update volumes. The availability of order
book data has provided substantial opportunities for automatic algorithmic
trading.





To illustrate this, we describe the Static Order Book Imbalance (SOBI) trading
strategy. SOBI computes a volume-weighted average price (VWAP) over those orders
whose volume makes up a fixed upper $k$-fraction of the total stock volume in
both bid and ask order books. SOBI then compares the two VWAPs and, based on
this, predicts a future price drift (for example a bid VWAP larger than an ask
VWAP indicates demand exceeds supply, and prices may rise). For simplicity, we
present the VWAP for the bids only:



\begin{verbatim}
select avg(b2.price * b2.volume) as bid_vwap
from   bids b2
where  k * (select sum(volume) from bids)
         > (select sum(volume) from bids b1
            where b1.price > b2.price);
\end{verbatim}
\comment{
Focusing on the $k$-fraction of the order book closest to the current price
makes the SOBI strategy less prone to attacks known as {\em axes}\/ (large
tactical orders far from the current price that will thus not be executed but
may confuse competing algorithms).
}


Coming back to our two desiderata, for trading algorithms to be successful, (1)
views such as VWAP need to be maintained and monitored by the algorithms at or
close to the trading rate. However, (2) the views cannot be expressed through
time-, row- or punctuation-based window semantics.
\comment{
This calls for a new breed of {\em large-state update stream processing
systems}.
}
This lends weight to the need for large, long-lived state in a DDMS.




\comment{
\section{Key Characteristics of DDMS}


\noindent DDMS are optimized for these four criteria:

\begin{enumerate}
\item
The stored dataset is large and changes frequently.

\item 
The
% computation of standing queries through the %incremental
maintenance of materialized views dominates ad-hoc querying.

\item
Access to the data is primarily by monitoring the views and performing
simple computations on top of them.

%
%; i.e., by reading out the views or by performing very simple queries %on top of the views which can be evaluated in a small fraction of the %time it would take to evaluate the view from scratch.
%
Views may be structured (tables) or booleans (flags, events). Thus some updates
cause events, observable in the views, that trigger subsequent computations, but
it is rare that the data store is accessed asynchronously by humans or
applications. A DDMS primarily interacts by triggering application code, rather
than by invocations from the applications.



\item
Updates happen primarily through an {\em update stream}\/. Computations
triggered by view events usually do not cause updates: there is usually no
feedback loop.

\end{enumerate}




\noindent We compare DDMS to existing data management systems.

{\em Compared to a classical DBMS}\/, a DDMS differs in its reaction to updates,
which will frequently have to be performed immediately when they arrive to
minimize response time, precluding bulk processing. This determines the
programming model: compared to DBMS, control flow is reversed, and the DDMS
invokes application code, not vice versa. 


An (active) DBMS~\cite{ceri-vldb:00} could simulate a DDMS through triggers, but
is not optimized for such workloads, and even if support for state-of-the-art
incremental view maintenance is present, performs very poorly. Thus, DDMS differ
from active databases in their being optimized for different workloads; DDMS are
optimized for event processing and monitoring tasks, while active database
systems are optimized for more traditional DBMS workloads. DDMS will not
necessarily support typical functionality such as transactions.



{\em Compared to a data stream processing system}\/
and particularly an event processing system (such as
Cayuga~\cite{demers-sigmod:07}, SASE+~\cite{agrawal-sigmod:08}),
%
DDMS have much larger states, which will usually have to be maintained in
secondary storage, and require drastically different query processing
techniques. In a stream processor, the queries reside in the system while the
data streams by. In a DDMS on the other hand, the data state is maintained in
the system while a stream of updates passes through (much more like an OLTP
system).

%
Moreover, event and stream processors~\cite{abadi-vldbj:03, motwani-cidr:03,
chandrasekaran-cidr:03} support drastically different query languages which are
designed to ensure that only very small state has to be maintained, using
windows or constructs from formal language theory~\cite{white-pods:07}. DDMS
views are often rather complex and expensive, including large non-windowed joins
and aggregation. In general, we expect DDMS to support standard SQL.


A DDMS can be thought of as an {\em update stream}\/ processing system in which
query workloads look more like those of classical DBMS (i.e., no constructs such
as windows). The query processing techniques that seem most readily suitable for
such workloads come from DBMS research -- incremental view maintenance in
particular -- but do not scale to high-frequency view maintenance.
}

\comment{
\smallskip

The sweet spots of DBMS, data stream processing systems, and DDMS with respect
to data state size and programming model (event processing vs. ad-hoc querying
and updating) are illustrated in Table~\ref{tab:quad_chart}.



\begin{table}
\begin{verbatim}
          state size |     small          large
                     |
event processing     |
---------------------+--------------------------
                     |
    no               |                     DBMS
                     |
    yes              |  stream proc.      *DDMS*
\end{verbatim}
\caption{Quad chart.}
\label{tab:quad_chart}
\end{table}
}

\medskip

{\bf Structure of the paper}\/.
There are many technical challenges in making a DDMS a reality. This paper
initiates a study of DDMS and presents DBToaster, a prototype developed by the
authors. The contributions of this paper are as follows: In Section
\ref{sec:overview}, we further characterize the notion of Dynamic Data
Management Systems, discuss the state machine abstraction, and describe schema
and view definition formalisms. Section \ref{sec:dbtoaster} presents the
DBToaster view maintenance technique. It demonstrates how the state machine
abstraction, which calls for the minimization or compilation of the state
transition function, leads to new algorithms. Section \ref{sec:storage}
discusses storage management in DBToaster. Finally, Section
\ref{sec:distribution} discusses scaling up through parallelization.



\documentclass{sig-alternate}


\usepackage{latexsym}
\usepackage{amsmath}
\usepackage{color}
\usepackage{graphicx}
\usepackage{amssymb}
\usepackage{epstopdf}
\usepackage{algorithmic}
\DeclareGraphicsRule{.tif}{png}{.png}{`convert #1 `dirname #1`/`basename #1 .tif`.png}



\def\punto{$\hspace*{\fill}\Box$}
\newcommand{\nop}[1]{}
\newcommand{\tuple}[1]{{\langle#1\rangle}}
\def\lBrack{\lbrack\!\lbrack}
\def\rBrack{\rbrack\!\rbrack}
\newcommand{\Bracks}[1]{\lBrack#1\rBrack}


\newtheorem{theorem}{Theorem}[section]
\newtheorem{proposition}[theorem]{Proposition}


\nop{
\newtheorem{metatheorem}{Metatheorem}[section]
\newtheorem{example}[theorem]{Example}
\newtheorem{algorithm}[theorem]{Algorithm}
\newtheorem{definition}[theorem]{Definition}
\newtheorem{property}[theorem]{Property}
\newtheorem{corollary}[theorem]{Corollary}
\newtheorem{lemma}[theorem]{Lemma}
\newtheorem{remark}[theorem]{Remark}
\newtheorem{conjecture}[theorem]{Conjecture}
\newtheorem{proviso}[theorem]{Proviso}
\newtheorem{todo}[theorem]{ToDo}
} % end nop



\conferenceinfo{SIGMOD}{'10 Indianapolis, IN}


\title{The Cumulus System for Exact
%, Low-Latency
Online Aggregation in the Cloud}


%\title{Streamlining Data Warehousing through Compilation}
\numberofauthors{3}
\author{}%Oliver Kennedy, Yanif Ahmad, Christoph Koch}
%\date{}                                           % Activate to display a given date or no date

\toappear{}

\newtheorem{example}{Example}

\begin{document}

\maketitle
%\section{}
%\subsection{}

\abstract{
In this paper we present Cumulus, a compiler for producing customized data warehouses.  Cumulus uses a novel set of techniques to compile the execution of complex queries down to simple message-passing operations between a set of map datastructures.  By expressing the query result in terms of a recursively defined view-maintenence problem, the data warehouse is able to maintain a set of intermediate results that make it possible to incrementally update the data warehouse in realtime.  The intermediate results produced by these incremental updates can also be used to simplify processing of OLAP-style queries.  Furthermore, the ease of partitioning and distributing Cumulus' map datastructures makes it feasible to deploy the entire data warehouse in-memory.  The result is a power-efficient data warehousing infrastructure that can maintain synchronization with a relational database, allowing online processing of high-volume aggregate queries.
}


\section{Introduction}


Recent years have seen the beginning of a paradigm shift in data management
research from incrementally
improving decades-old database technology
%
% (particularly, OLTP) -- System R and
% the long line of systems that have followed it --
%
to questioning established
architectures and creating fundamentally different, more lightweight systems
that are often domain-specific
(cf.\ e.g. \cite{DBLP:conf/vldb/StonebrakerMAHHH07,DBLP:journals/pvldb/KallmanKNPRZJMSZHA08}).
Part of the impetus for this change was given by 
potential users such as scientists and the builders of
large-scale Web sites and services such as Google, Amazon, and Ebay,
who have a need for data management systems but have found current databases
not to scale to their needs.
One can observe a trend to disregard database contributions
in these communities \cite{dbcolumn, DBLP:conf/sigmod/PavloPRADMS09}, and to build lightweight systems based on
robust technologies mostly pioneered by the operating systems and distributed
systems communities, such as large scale file systems, key-value stores, and
map-reduce
\cite{DBLP:journals/cacm/DeanG08, DBLP:journals/tocs/ChangDGHWBCFG08}.
Further impetus has resulted from the current need to develop data management
technology for multicore and cloud computing.
%
%, and by the example given by a
%number of innovative recent database startup companies that develop
%databases based on new architectures.
%\footnote{Examples are Vertica and Paraccel, who build column stores,
%and Greenplum, Asterdata, and Netezza, who take databases into the Cloud.}

There is a recent tendency among pundits outside the database community to
contest the need for powerful queries, and to
think of key-value stores -- with only the power to look up data by
keys -- as (much more efficient) database query engines.
%
%It shall not be denied that,
%with clever engineering, a surprising range of problems can be solved
%using key-value stores.
%
However, expressive query languages such as SQL do not cease to have
important applications and a substantial user base.
Alas, we do not know how to process SQL queries on updateable data
using a system as lightweight as a key-value store.

This paper contributes a fundamental and versatile building block for
enabling new, more lightweight and nimble data processing systems based
on SQL aggregation que\-ries. We believe that our contribution
constitutes an important
step towards achieving the contradiction in terms mentioned
above: executing complex aggregation queries on updateable data
using little more than a key-value store.

At the heart of our approach is a new aggressive recursive incremental
view maintenance mechanism.
In most traditional database query processors, the  basic building blocks of
queries are large-grained operators such as joins.
Our approach is based on compilation, reducing
queries to programs that are not based on classical query operators.
A large class of
SQL aggregation queries can be compiled down to very simple message
passing programs that incrementally maintain materialized views of
the queries. These message passing programs keep a hierarchy of map data
structures (which may be served out of a key-value store) up to date
and can share computation in the case that multiple aggregation queries (e.g.,
a data cube) need to be maintained.  Most importantly, though, these message
passing programs can be massively parallelized to the degree that the updating
of each single result aggregate value can be done in constant time on normal
off-the-shelf computers.\footnote{That is,
we assume that addition and multiplication
of {\em two numbers} can be performed in constant time, which is true for
%
%bounded precision numbers such as 
%
standard base types such as int and float,
but we assume no unrealistic models such
as aggregators with unbounded fan-in, as used in some theoretical models of
parallel computation. Our implementation indeed incrementally maintains
individual aggregate values in constant time. Note that since there are
usually many more aggregate values to maintain than there are processors,
this does not mean that each update is processed in constant time.
``Constant time'' is with respect to the size of the data, not the compiled
query.}
To the best of our knowledge, it was not known before
that this is possible.

In comparison, no such constant-time parallel processing technique
is known for nonincremental query
evaluation: Indeed, it is unlikely to exist.\footnote{Constant-time
bounded fan-in nonincremental
parallel processing is known not to be possible for
the class of queries we address,
unless the complexity class TC0 collapses into NC0, which it is not known
to do \cite{Joh90}.} Classical incremental view maintenance approaches, which
express the delta (=change) to a query result given an update again
as a (slightly simpler) query, fare no better: Generally,
given a query, there is another query whose delta is the first query.
Thus, classical incremental view maintenance has the same limits to
parallelization as nonincremental evaluation.


\subsection{Message passing programs}


We compile SQL aggregation queries to {\em map maintenance
message}\/ (M3) programs. An M3 program consists of a set of triggers of
the form
\[
\mbox{{\tt on insert into $R$($\vec{x}$) \{
($\vec{y}$:$D_{\vec{y}}$) $m[\vec{x}, \vec{y}]$ += $s$
\}}}
\]
or
\[
\mbox{{\tt on delete from $R$($\vec{x}$) \{
foreach $\vec{y}$ do $m[\vec{x}, \vec{y}]$ -= $s$
\}}}
\]
where $R$ is a relation name,
$\vec{x}$ and $\vec{y}$ are distinct tuples of variables and
$s$ is either a term or of the form
{\tt if $\phi$ then $t$ else 0}, where $t$ is a term.
Terms are built from addition, multiplication,
constants, variables from $\vec{x}$, external function calls $f(\vec{z})$,
and map accesses $m_1[\vec{z}_1], \dots, m_k[\vec{z}_k]$ where
$m$, $m_1$, $\dots$, $m_k$ are pairwise distinct
and the variables in $\vec{z}_1, \dots, \vec{z}_k$ are a nonoverlapping
subsets of the variables in $\vec{x}, \vec{y}$.
Conditions $\phi$ are conjunctions of comparisons $t'' \;\theta\; t'''$,
where $t'',t'''$ are terms without map accesses and
$\theta \in \{ =,<,\le,\neq \}$.
If $\vec{y}$ consists of zero variables, we omit
{\tt foreach $\vec{y}$ do}.
For each relation name, there may by multiple insert and delete triggers.

M3 programs can be read as straightforward pseudocode.
There are subtle issues to be discussed later about the domains of 
variable tuples
to be iterated over by foreach loops. Let us for now assume that
{\tt foreach $\vec{x}$ do $(\dots)$} iterates over all distinct tuples of
values currently in the
database, and that map values $m[\vec{x}]$ for $\vec{x}$ containing newly
inserted values are initially zero.


\begin{example}\em
\label{ex:TPCH-Q12}
Consider the following query on a TPC-H like schema,
which counts the number of LineItems per customer id.
\begin{verbatim}
SELECT   C.cid, SUM(1)
FROM     Customer C, Order O, LineItem L
WHERE    C.cid=O.cid AND O.oid=L.oid
GROUP BY C.cid;
\end{verbatim}
Here, cid is a key for the Customer relation and oid is a key for the
Order relation, but oid is not a key for LineItem.
Our compiler translates this query to the M3 program
\begin{verbatim}
on insert into Customer (cid, ...) { qO1[cid] += 1 }
on insert into Order (oid, cid, ...) {
  qL[cid, oid] += qO1[cid]
}
on insert into LineItem (oid, ...) {
  foreach cid do q[cid] += qL[cid, oid]
}
\end{verbatim}

In this and the following example, the delete-triggers are precisely
like the insert-triggers, but with {\tt +=} replaced by {\tt -=}.
Thus, to save space, the deletion triggers are omitted.

Let us ignore parallelization first.
It is not hard to see that this trigger program correct maintains the
query result, for each distinct {\tt cid} in Customer.cid, as {\tt q[cid]}.
(The maps {\tt qO1} and {\tt qL} are auxiliary.)
We assume that
there are no cascading deletes and, for instance, before we can delete an
Order, we have to delete all associated lineitems.
\punto
\end{example}


{\em Parallelization}.
The syntax of statements
\begin{equation}
\mbox{{\tt foreach $\vec{y}$ do $m[\vec{x}, \vec{y}]$ $\pm$= $s$}}
\label{eq:foreach}
\end{equation}
is misleading in that it suggests a loop --
that a nonconstant amount of work is needed to bring aggregate
values up to date. Of course, polynomial amounts of work are in fact need,
but only because in general there are many aggregate values -- in a
map representing the result of a group-by query or in an auxiliary map --
to be maintained. In fact, each statement of form (\ref{eq:foreach})
writes each value $m[\vec{x}, \vec{y}]$ only once and admits
{\em embarassing parallelism}: $m$ can be partitioned across many machines
that share the work.

Assume that the storage of individual maps
is partitioned across several machines. To execute a statement of form
(\ref{eq:foreach})
in a trigger invocation with arguments $\vec{x} = \vec{a}$,
where $s$ uses map lookups $m_i[\vec{x}_i, \vec{y}_i]$,
each node storing a value $m_i[\vec{a}_i, \vec{y}_i] = v$,
for $\vec{y}_i$ arbitrary, sends the message
$m_i[\vec{a}_i, \vec{y}_i] = v$ to the node managing value
$m[\vec{a}, \vec{y}]$. This way, that node receives all the values it
needs to update all $m$ values it represents.
Of course this requires a suitable protocol to ensure overall consistency
and that the right versions of map values are read and written in the right
order.



\begin{example}[star-join decomposition]\em
\label{ex:ssb}
Con\-sider \\ a simplified version of the star schema
benchmark (SSB) schema with relations Date(\underline{datekey}, year),
Part(\underline{partkey}, partcat), where partcat stands for a part category,
and LineOrder(datekey, partkey, revenue), which may contain duplicate tuples.
The query asks for the total revenues grouped by year and part category.

\begin{verbatim}
SELECT   P.partcat, D.year, SUM(revenue)
FROM     Date D, Part P, LineOrder L
WHERE    D.datekey=L.datekey
AND      P.partkey=L.partkey
GROUP BY P.partcat, D.year;

on insert into Date (datekey, year) {
  mPL[datekey, year] += 1
}
on insert into Part (partkey, partcat) {
  mDL[partkey, partcat] += 1
}
on insert into LineOrder (datekey, partkey, revenue) {
  foreach (partcat, year) do
  m[partcat, year] += revenue
                    * mDL[partkey, partcat]
                    * mPL[datekey, year]
}
\end{verbatim}

Observe how, on insertion into LineOrder, the code for incrementally
maintaining the query result {\tt m} decomposes into
two parts with disjoint variables, 
{\tt mDL[partkey, partcat]} and {\tt mPL[datekey, year]}.

The maps mPL and mDL have value at most
one at each position because datekey and partkey are keys for Date and Part,
respectively.

On insert into LineOrder, given values for
datekey and partkey, we instruct nodes
to send their {\tt mDL[partkey, x]} and {\tt mPL[datekey, y]} values,
for any {\tt x} and {\tt y},
to nodes maintaining {\tt m[x, *]} and {\tt m[*, y]}, respectively.
A node managing {\tt m[u, v]} receives, possibly from distinct nodes,
{\tt mDL[partkey, u]} and {\tt mPL[datekey, v]}
and can increment {\tt m[u, v]} by
{\tt revenue*mDL[partkey, u]*mPL[datekey, v]}.
%
%We only send nonzero {\tt mDL[partkey, x]} and {\tt mPL[datekey, y]} values,
%but at least an empty message so that the node managing m knows that it does
%not have to wait for anything more.
\punto
\end{example}


\begin{example}[self-join]\em
\label{ex:self-join}
We now ask, for each customer id (cid),
for the number of customers of the same nation (including the customer
identified by cid in the count).
\begin{verbatim}
SELECT   C1.cid, SUM(1)
FROM     Customer C1, Supplier C2
WHERE    C1.nation = C2.nation
GROUP BY C1.cid;
\end{verbatim}
The compiler produces the following on-insert trigger:
\begin{verbatim}
on insert into Customer (cid, nation) {
  q[cid] += qC1[nation];
  foreach cid2 do q[cid2] += qC2[cid2, nation];
  q[cid] += 1;
  qC1[nation] += 1;
  qC2[cid, nation] += 1
}
\end{verbatim}
The on-delete trigger is just like the on-insert trigger with {\tt +=}
changed to {\tt -=} everywhere other than in the third statement
({\tt q[cid] += 1}), which remains unchanged.
We will establish later that this M3 program is indeed correct.

For now, we challenge the reader to find a 
fundamentally different (ideally, simpler) way to
perform the incremental maintenance of {\tt q}
which has the M3 property of embarassing parallelism, with each value
to be updated only requiring a constant amount of work.
Examples~\ref{ex:TPCH-Q12} and \ref{ex:ssb} were chosen for simplicity,
but we believe that this example shows that creating M3 programs in general
is nontrivial.
\punto
\end{example}


It shall be emphasized that for each of the examples of this section,
and the paper as a whole, our compilation approach produces exactly
the M3 programs shown.


The fragment of SQL queries that we can compile to M3 essentially comprises
SUM-agg\-regation queries with group-by.
COUNT and AVG queries can be defined by arithmetic expressions over these.
We exclude MIN and MAX queries, aggregation
nested into FROM or WHERE clauses, the DISTINCT and HAVING keywords,
outerjoins,
and the relational difference operation. At the end of this paper, we will
discuss which of these features can be added without fundamental difficulties.


\subsection{The Cumulus System}


We have developed a system, Cumulus\footnote{A Cumulus cloud is an
aggregation cloud, thus the name.}, that parallelizes the
execution of M3 programs in a cluster or computing cloud.
Cumulus performs incremental maintenance of exact aggregation views online, and
executes an efficient
protocol to ensure consistency of map data and query results,

While it is no fundamental requirement of our compilation
approach, we have chosen to use the resources of the cloud to maintain
the data in main memory, allowing for very low latency updating and querying.
At the time of writing this,
Terabyte-sized memory chips (DIMMs and flash) have already been announced by
manufacturers, and already now, large data warehouses
can be run in main memory in the cloud, where additional hardware costs
(main memory is more expensive per TB than hard disks) are
offset by greater robustness of the system, lower maintenance
costs, lower heat production \cite{1154557}, and of course by of orders
of magnitude better speed and latency characteristics.

The Cumulus protoype aims at demonstrating our results in the context of
pushing OLAP into the cloud.
Cumulus automates  the process of  creating, loading,
and  maintaining  in-memory  data  warehouses.
(Optional logging of updates to secondary storage
for persistency is supported.)
Cumulus targets OLAP applications  that perform real-time analytics of
relational data.  By feeding it an SQL query, Cumulus's infrastructure
becomes linked to  a set of OLTP databases.  Cumulus  keeps the
data warehouse synchronized with the source databases via an
update stream. It achieves synchronization {\em in realtime}
through parallelization, keeping data in main memory, and our approach of
query processing by message passing.


\subsection{Contributions and Structure of the Paper}


Our main technical contributions are as follows.
\begin{itemize}
\item
We present M3, a massively parallelizable language
for message passing programs that can be used to incrementally maintain
SQL aggregation queries.

\item
We describe our compiler for translating SQL aggregation queries to M3
programs. Our compilation technique is based on a novel, aggressive, recursive
form of incremental view maintenance.

\item
We present Cumulus, our system for exact online aggregation in realtime.
We describe the Cumulus message passing protocol, which assures
consistency of the maps using only few messages, and
infrastructure, and show how it can be used to efficiently distribute the
processing and storage requirements of query processing and
the incremental maintenance of large aggregate views and datacubes.

\item We show evidence for the scalability of our approach by examining the
performance of Cumulus on examples drawn from the TPC-H\cite{tpch2008}
benchmark. 
\end{itemize}


The remainder of this paper is organized as follows.
Section \ref{sec:compiler} describes our SQL to M3 compiler.
In Section \ref{sec:architecture}, we provide an overview of Cumulus's online
infrastructure and discuss how data is managed within that infrastructure.
Section \ref{sec:experiments} presents
experimental results that demonstrate the viability and scalability of
Cumulus.
Section \ref{sec:relatedwork} discusses related work.
The paper concludes with Section \ref{sec:conclusions}









\section{Compilation to M3}
\label{sec:compiler}


\subsection{Query Calculus}


\def\safe{\mbox{safe}}
\def\AggSum{\mbox{Sum}}

Our calculus consists of
{\em formulae} of positive quantifier-free relational domain calculus
(i.e., formulae constructed from conjunctions ``and'',
disjunctions ``or'', and atoms) and of {\em terms}.
%
The atomic formulae are {\em true}, {\em false}, relational atoms $R(\vec{x})$
where $\vec{x}$ is a tuple of variables,
and atomic constraints of the form $t_1 \;\theta\; t_2$ comparing two terms
$t_1$ and $t_2$ using comparison operations $\theta$ of $=$, $\neq$, $<$,
and $\leq$.
%
Terms are built from variables, constants, built-in function calls
$f(\vec{t})$, where $\vec{t}$ is a tuple of terms,
and aggregrate sums ($\AggSum$) using addition and multiplication.
Built-in functions compute their result entirely based on their input
terms, not accessing the database (e.g., mod or string concatenation).
In short, the grammar for formulae $\phi$ and terms $t$
(given variables $x$, constants $c$, relation names $R$,
comparison operators $\theta$,
and builtin functions $f$) is
\begin{eqnarray*}
  \phi &\mbox{::-}& \phi \land \phi
               \mid \phi \lor \phi \mid (\phi)
               \mid \mbox{true} \mid \mbox{false} \mid R([x(,x)^*])
               \mid t \;\theta\; t
\\
  t &\mbox{::-}& t * t \mid t + t \mid (t) \mid c \mid x \mid f([t(,t)^*]) \mid
                 \AggSum(t, \phi)
\end{eqnarray*}

An aggregate term $\AggSum(t, \phi)$
is {\em constraints-only} if $\phi$ does not
contain relational atoms $R(\vec{x})$ (but atomic constraints $s \theta t$).
We can also think of such an aggregate term as a (functional)
if-statement (if $\phi$ then t else 0)
or, using C syntax, ($\phi$ ? $t$ : 0).

We will make one important syntactic restriction, namely that
no $\AggSum$ terms may occur in atomic constraints.

Formulas and terms are evaluated relative to a given set of
{\em bound variables}.
The bound variables of a subformula are the bound variables of the formula.

Given a set of bound variables $B$,
the {\em safe variables} of a formula are defined bottom-up
as usual in relational
calculus (see e.g. \cite{DBLP:books/aw/AbiteboulHV95}). In particular,
\begin{eqnarray*}
\safe_B(R(\vec{x})) &:=& \{x_i\} \cup B \\
\safe_B(\phi \land \psi) &:=& \safe_B(\phi) \cup \safe_B(\psi) \\
\safe_B(\phi \lor \psi)  &:=& \safe_B(\phi) \cap \safe_B(\psi) \\
\safe_B(\phi \land x = y) &:=&
\left\{\begin{array}{ll}
\safe_B(\phi) \cup \{ x, y \} &\dots
\mbox{$x$ or $y$ is} \\
&\;\;\; \mbox{in $\safe_B(\phi)$} \\[.5ex]
\safe_B(\phi) &\dots \mbox{otherwise}.
\end{array} \right.
\end{eqnarray*}
Here $\{x_i\}$ drops order and turns the tuple $\vec{x}$ into a set.

Given a term $\AggSum(t, \phi)$ with bound variables $B$,
the bound variables of $\phi$ are $B$ and the bound variables of $t$ are
the safe variables of $\phi$, $\safe_B(\phi)$.
The bound variables of a subterm are the bound variables of the term.
Variables occuring as terms must be bound.

\begin{example}\em
Given singleton bound variable set $\{ y \}$,
\[ \AggSum(u * f(z), \underbrace{(\underbrace{(\underbrace{R(x, z)}_{x,y,z} \lor \underbrace{y=z}_{y,z})}_{y,z} \land z = w)}_{y,z,w}) \]
is invalid: The safe variables of the formula are
$\{y,z,w\}$, so $u$ is not bound in the term $u * f(z)$. The overall term
becomes valid for bound variables $\{u,y\}$.
\punto
\end{example}


\def\db{{\cal{A}}}

The semantics of formulas and terms is given by a (polymorphic) function
$\Bracks{\cdot}(\cdot, \cdot)$ that takes a database and values for the
bound variables as arguments. Given database $\db$ and values $\vec{b}$ for
the bound variables,
$\Bracks{\phi}(\db, \vec{b})$ evaluates to a relation and
$\Bracks{t}(\db, \vec{b})$ evaluates to a value of the type of
terms.\footnote{In practice,
we have several types such as integers and floats, but
here we will not talk about types and will assume that all terms evaluate to,
say, floating point numbers. However, our implementation supports the
main data types of SQL, and no noteworthy observations were made achieving
this.}
We assume a multiset semantics for relations, which is important to note
since we focus on computing aggregates. The multiset semantics of formulas
is defined by their well-known translation to relational algebra
(Codd's theorem), and the standard multi-set semantics of
(in our case, positive) relational algebra. 
$\AggSum$ terms are new, but otherwise the
semantics of terms is obvious.
A term $\AggSum(t, \phi)$
computes the sum of the values $t[\vec{x}]$
over the distinct valuations (with duplicates)
of the safe variables $\vec{x}$ of $\phi$, i.e.,
given a database $\db$ and values $\vec{b}$ for the bound variables of $\phi$,
\[
\Bracks{\AggSum(t, \phi)}(\db, \vec{b}) =
\sum_{\vec{v} \;\mathrm{in}\; \Bracks{\phi}(\db, \vec{b})} \Bracks{t}(\db, \vec{v}).
\]


{\bf From SQL to the Calculus}.
Given our semantics definition, the translation from SQL to our calculus is
straightforward.
%
%We focus on aggregation queries, specifically
%sum aggregation queries (count and avg aggregation queries can be encoded
%using sum). Aggregates can be nested in the SELECT clause, but not in
%the FROM, WHERE, or HAVING clause. We support GROUP by, although in a way
%that may at first seem nonstandard. We do not support DISTINCT.
%
A SQL aggregate query
\begin{verbatim}
SELECT groupcols, SUM(t)
FROM   R1 r11, R1 r12, ..., R2 r21, ...
WHERE  cond
GROUP BY groupcols
\end{verbatim}
is expressed in the calculus as
\[
\AggSum(t, R_1(\vec{x}_{11}) \land R_1(\vec{x}_{12}) \land \dots
\land R_2(\vec{x}_{21}) \land \dots \land \mbox{cond})
\]
with {\em bound variables} groupcols.


\begin{example}\em
\label{ex:self-join-calc}
The query of Example~\ref{ex:self-join} translates to
$\AggSum(1, \mbox{Customer}(c_1,n_1) \land \mbox{Customer}(c_2, n_2) \land
n_1=n_2)$ in the calculus, with bound variable $c_1$.
\punto
\end{example}

\subsection{Normalization and Simplification}


A semiring is an algebra with two associative operations,
$+$ and $*$, that
have neutral elements (called 0 and 1, respectively), which satisfy
distributivity ($a*(b+c)= a*b + a*c$), and where $+$ is commutative.
Semirings with variables
have polynomials, that is, each expression of the
semiring can be mapped to an equivalent expression that is
a sum of flat products (the products are also known as {\em monomials}\/).
Turning semiring expressions into polynomials just means to apply
distributivity repeatedly until we end up with a polynomial.
This can be combined with simplification operations based on the 1 and
0-elements, i.e., $\alpha * 1$ maps to $\alpha$, $\alpha*0$ maps to $0$, and
$\alpha+0$ maps to $\alpha$. Polynomials can be conveniently implemented
as lists of lists of atoms, where an empty top-level list (i.e., polynomial)
has value 0 and an empty monomial list has value 1.

Both our formulae and our terms are semirings; in particular,
in the semiring of formulae, $\land$ is the product operation,
$\lor$ is addition,
and $\textit{false}$ and $\textit{true}$ are 0 and 1, respectively.

For arbitrary terms $s$ and $t$ and formulas $\phi$ and $\psi$,
$\AggSum$ terms can be simplified using the following equations (to be
applied by replacing a left by a right hand side expression)
\begin{eqnarray*}
\AggSum(t, \textit{true}) &=& t \\
\AggSum(t, \textit{false}) &=& 0 \\
\AggSum(0, \phi) &=& 0 \\
\AggSum(s+t, \phi) &=& \AggSum(s, \phi) + \AggSum(t, \phi) \\
\AggSum(t, \phi \lor \psi) &=& \AggSum(t, \phi) + \AggSum(t, \psi)
\end{eqnarray*}

All these algebraic laws can be applied, and the calculus expression
be maximally simplified, in a single bottom-up pass of the expression.
A term ($\AggSum$ or other) maximally simplified in this way
is a sum of terms that contain neither $+$ nor $\lor$; we call such 
normalized terms {\em recursively monomial}.
Define function RecMonomials to compute the list of recursively monomials
of a term.

\def\vars{\mbox{vars}}

{\bf Factorization of monomial aggregate terms}.
For $e$ either a formula or a term, let $\vars(e)$
be the set of all variables occurring in $e$.
Factorization employs the equivalence
\[
\AggSum(s*t, \phi \land \psi) = \AggSum(s, \phi) * \AggSum(t, \psi)
\]
which is true if
$(\vars(s) \cup \vars(\phi)) \cap (\vars(t) \cup \vars(\psi)) = \emptyset$.

Consider an aggregate term $\AggSum(t, \phi)$ where both $t$ and $\phi$ are
monomials, consisting of the sets $T$ and $F$ of atomic terms and formulae,
respectively (that is, $t = \prod T$ and $\phi = \bigwedge F$).
We can think of the
elements of the two-sorted set $T \cup F$ as the hyperedges of a
{\em hypergraph},
where the function $\vars(e)$ maps hyperedge $e$ to the nodes that are
part of it.
The set of {\em connected components} ${\cal C}$ of this hypergraph is the
maximum cardinality set of subsets of $T \cup F$ such that for any
two components $C_1, C_2 \in {\cal C}$ with $C_1 \neq C_2$,
$\vars(C_1) \cap \vars(C_2) = \emptyset$. Asking for the maximum number of
nonoverlapping components is of course the same as asking for components of
minimum size, and the set of connected components is unique and can be computed
in linear time using Tarjan's algorithm. Given ${\cal C}$, each component
$C \in {\cal C}$ can again be partitioned by sort into a monomial term $t_C$
and a monomial formula $\phi_C$.
%
% By convention, if $C$ does not contain atomic terms, $t_C = 1$ and if $C$
% does not contain atomic formulae, $\phi_C = \textit{true}$. It is not
% difficult to verify that
%
$\AggSum(t, \phi)$ is equivalent to
\[
\prod_{C \in {\cal C}} \AggSum(t_C, \phi_C).
\]

{\em Recursive factorization}, given term $\AggSum(t, \phi)$, first recursively
factorizes the aggregate terms in $t$ before applying factorization as
just described on the top level.


\begin{example}\em
The connected components of term
\[
\AggSum(5 * x * \AggSum(1, R(y, z)) * w, R(x,y) \land S(z) \land R(v, w))
\]
are
$\{ \{5\}, \{ x, R(x,y), \AggSum(1, R(y, z)), S(z) \}$,
$\{ w, R(v, w)) \} \}$
and thus the term factorizes as
$\AggSum(5, \textit{true}) *
\AggSum(x * \AggSum(1$, $R(y, z)), R(x, y) \land S(z)) *
\AggSum(w, R(v, w)))$.
$\AggSum(5, \textit{true})$ simplifies to $5$.
\punto
\end{example}


{\bf Variable elimination}.
Given an abitrary monomial formula $\phi = \bigwedge (E, O)$,
where $E$ are the equality
atoms $x=y$ and $O$ are the remaining atoms (either set may be empty) and
a set of bound variables $B$.
We eliminate redundant variables as follows.

Consider the equivalence classes of the equivalence relation $E$.
For each equivalence class $C$ of $E$, distinguish an element
(i.e., variable) as $x_C$ such that,
if $B \cap C \neq \emptyset$, $x_C$ is an arbitrary element of $B \cap C$;
otherwise, it is an arbitrary element of $C$.
Create a unification mapping
$\Theta$ that maps each unbound variable $y$ of $E$ to $x_{[y]}$ (where $[y]$ is the
equivalence class of $y$) and is the identity on the bound variables.
Now substitute all variables in $O$
using $\Theta$, obtaining $O'$. Let
$E' =  \bigcup \{ y = x_{[y]} | y \in ((B \cap [y]) -  x_{[y]}) \}$.
Then we replace $\phi$ by $(\bigwedge O') \land \bigwedge E'$.

Given a term $\AggSum(t,\phi)$ where $\phi$ is a monomial,
and bound variables, we eliminate variables by 
first eliminating variables in $\phi$, creating $\phi'$ and $\Theta$.
Then we substitute all variables of $t$ that are in the domain of $\Theta$
using $\Theta$, obtaining $t'$. The result, $\AggSum(t', \phi')$, is equivalent
to $\AggSum(t,\phi)$.


\begin{example} \em
Given term $\AggSum(y*v*r, R(z, v) \land v<q
\land x=y \land x=z \land u=v \land v=w \land q=r)$
and bound variables $\{x,y,z,r\}$.
The variable equivalence classes are
$\{ \{x,y,z\}, \{u,v,w\}, \{q,r\} \}$. We chose the rightmost variable in
each class as the variable to substitute by. In the first class, we can choose
freely because all members are bound. In the second we can choose freely
because none are bound. In the third, we must choose $r$ because it is bound
and $q$ is not.
The mapping is
$\Theta = \{ x \mapsto x, y \mapsto y, z \mapsto z, u \mapsto w, v \mapsto w,
w \mapsto w, q \mapsto r, r \mapsto r \}$.
We apply $\Theta$ to $y*v*r$ and $R(z, v) \land v < q$ and obtain
$z*w*r$ and $R(z,w) \land w<r$, respectively. The simplified
overall term is
$\AggSum(z*w*r, R(z,w) \land w<r \land x=z \land y=z)$.
\punto
\end{example}


{\bf Extraction of aggregates}.
For a term $t$ and its set $B$ of bound variables,
the function ExtractAggregates($t$, $B$)
replaces each maximal subterm $s$ of $t$
that is of the form $\AggSum(\cdot, \cdot)$ but is not constraints-only
by a ``map access''  $m[\vec{x}]$, where
$m$ is a new name and $\vec{x}$ is the set of variables
both bound at $s$ and used in $s$ ordered arbitrarily.
The result of ExtractAggregates thus is a pair $(t', \Theta)$ of the remainder
term $t'$ and a mapping $\Theta$ from map accesses $m[\vec{x}]$ to extracted
subterms $s$ (which could be used to undo the extraction).

\begin{example}\em
Let $t$ be the term
\begin{multline*}
\AggSum(x*\AggSum(w, R(v, w), R(w, z)), x<y \land y=z) \\
*\; 5 * y * \AggSum(u, R(u, x)).
\end{multline*}
Then ExtractAggregates($t$, $\{x,y\}$) returns the pair consisting of term
$\AggSum(x*m_1[z], x<y \land y=z) * 5 * y * m_2[x]$
and the mapping
$\{
m_1[z] \mapsto \AggSum(w, R(v, w), R(w, z));
m_2[x] \mapsto \AggSum(u, R(u, x))
\}$.
\punto
\end{example}


{\bf Lifting ifs}. Observe that if $\phi$ is a constraints-only term in which
all variables are bound, then
\begin{eqnarray*}
\AggSum(t, \phi \land \psi) &=& \AggSum(\AggSum(t, \psi), \phi) \\
t * \AggSum(t', \phi) &=& \AggSum(t * t', \phi)
\end{eqnarray*}
Thus we can lift $\phi$ to the top of a recursively monomial term.
Let function LiftIfs do exactly this.

\begin{example}\em
This will be used in Example~\ref{ex:self-join-compile}:
\begin{multline*}
\mbox{LiftIfs}((-1) * \AggSum(1, C(c_2, n) \land c_1=c), \{c_1, c, n\}) = \\
\AggSum((-1) * \AggSum(1, C(c_2, n)), c_1=c).
\end{multline*}
\end{example}


Given a recursively monomial term $t$ and a set of bound variables $B$,
let the function Simplify($t$, $B$) recursively factorize $t$, then perform
variable elimination, and finally if-lifting.


\subsection{Delta Computation}


\def\dt{\Delta_{\pm R(\vec{t})}}


Given a term or formula $\alpha$
of our calculus, and an insertion or deletion of a
single tuple $\vec{t}$ to/from a relation $R$ of the database.
We denote the database obtained from database $\db$ by this update by
$\db \pm R(\vec{t})$.
We can express a delta $\Delta_{\pm R(\vec{t})} \alpha$
(which is a term if $\alpha$ is a term and a formula if $\alpha$ is a formula)
such that,
given current database $\db$ and values $\vec{b}$ for the bound variables,
\begin{eqnarray}
\Bracks{\alpha}(\db \pm R(\vec{t}), \vec{b}) &=&
\,\Bracks{\alpha}(\db, \vec{b}) \pm \,\Bracks{\dt \alpha}(\db, \vec{b}).
\label{eq:delta}
\end{eqnarray}

The delta rules for semirings are as follows. We use $+$ and $*$ for the
addition and multiplication operations; for formulae, these are of course
$\lor$ and $\land$, respectively.
\[\begin{array}{lllcr}
\dt (\alpha + \beta) &:=& ((\dt \alpha)    &+& (\dt \beta))
\\[1ex]
\dt (\alpha \,*\, \beta)
   &:=& ((\dt \alpha)               &*& \beta\;\,) \\
   &+ & (\quad\quad\quad\;\, \alpha &*& (\dt \beta)) \\
   &+ & ((\dt \alpha)               &*& (\dt \beta))
\end{array}\]

For atomic formulae and terms,
\[\begin{array}{lllr}
\dt \AggSum(t, \phi)
   &:=& \AggSum((\dt t), & \phi\;\,) \\
   &+ & \AggSum(\quad\quad\quad\; \,t,  & (\dt \phi)) \\
   &+ & \AggSum((\dt t), & (\dt \phi))
\\[1ex]
\dt R(x_1, \dots, x_k) &:=& \big( \bigwedge_{i=1}^k (x_i = t_i) \big)^\pm
\end{array}\]
and, for $S$ a relation different from $R$,
\[
\dt S(x_1, \dots, x_l) := \textit{false}.
\]
For all other atomic terms and formulae, $\dt$ is the zero-element
of their respective semirings (that is, $0$ and $\textit{false}$,
respectively.)

Here $(\cdot)^\pm$ is an annotation that we do not give a formal semantics
to for space limitations, but explain how to eliminate. $\phi^+$ is just
$\phi$ and $\phi^-$ intuitively defines a relation of ``negative'' tuples.
We define $\phi^- \land \psi = \phi \land \psi^- = (\phi \land \psi)^-$,
$\phi^- \land \psi^- = \phi \land \psi$, and
$\AggSum(t, \phi^-) = -\AggSum(t, \phi)$. To
push $(.)^-$ up beyond $\lor$, we first compute
a DNF, push $\lor$ out of the formula across $\AggSum$, and then apply
the above rules to the remaining monomial formulae.
For example,
$\AggSum(t, \phi \land (\psi \lor \pi^-)) =$
$\AggSum(t, \phi \land \psi) + \AggSum(t, \phi \land \pi^-) =$
$\AggSum(t, \phi \land \psi) - \AggSum(t, \phi \land \pi)$.


\begin{proposition}
\label{prop:delta-correct}
This definition of $\dt$ satisfies Equation \ref{eq:delta}.
\end{proposition}


\def\duv{\Delta_{\pm R(u,v)}}
\def\dc{\Delta_{\pm C(c,n)}}


\begin{example}\em
\label{ex:self-join-delta}
Consider the query of Example~\ref{ex:self-join}, which translates to the
calculus as
\[
q[c_1] = \AggSum(1, C(c_1,n_1) \land C(c_2, n_2) \land n_1=n_2)
\]
where $C$ is short for Customer (see Example~\ref{ex:self-join-calc}).
Now, since $(\dc n_1=n_2) = 0$ and thus
\[
(\dc C(c_2, n_2) \land n_1=n_2) = (c_2=c \land n_2=n)^\pm \land n_1=n_2,
\]
\begin{multline*}
\dc \AggSum(1, C(c_1,n_1) \land C(c_2, n_2) \land n_1=n_2) = \\
\AggSum(1, \dc (C(c_1,n_1) \land (C(c_2, n_2) \land n_1=n_2))) = \\
\AggSum(1, 
((c_1=c \land n_1=n)^\pm \land (C(c_2, n_2) \land n_1=n_2)) \lor \\
(C(c_1,n_1) \land (c_2=c \land n_2=n)^\pm \land n_1=n_2) \lor \\
((c_1=c \land n_1=n)^\pm \land (c_2=c \land n_2=n)^\pm \land n_1=n_2) =
\end{multline*}

\vspace{-6mm}

\begin{eqnarray*}
&\pm& \AggSum(1, c_1=c \land n_1=n \land C(c_2, n_2)       \land n_1=n_2) \\
&\pm& \AggSum(1, C(c_1,n_1)        \land c_2=c \land n_2=n \land n_1=n_2) \\
&+&   \AggSum(1, c_1=c \land n_1=n \land c_2=c \land n_2=n \land n_1=n_2)
\end{eqnarray*}

Simplifying this with bound variable $c_1$ yields
%\begin{eqnarray*}
%&\pm& \AggSum(1, c_1=c \land C(c_2, n)) \\
%&\pm& \AggSum(1, C(c_1,n)) \\
%&+&   \AggSum(1, c_1=c)
%\end{eqnarray*}
\[
\pm \AggSum(1, c_1=c \land C(c_2, n))
\pm \AggSum(1, C(c_1,n))
+   \AggSum(1, c_1=c)
\]

Now suppose $C$ currently stores tuples (Joe, USA) and (Bill, USA).
Then $q$[Joe] = $q$[Bill]=2 (and $q$[Dan] would be 0 by default).
Now insert (Dan, USA).
The changes to the query map are
\begin{eqnarray*}
q[\mbox{Joe}]  &\mbox{+=}& 0 + 1 + 0 = 3 \\
q[\mbox{Bill}] &\mbox{+=}& 0 + 1 + 0 = 3 \\
q[\mbox{Dan}]  &\mbox{+=}& 2 + 0 + 1 = 3.
\end{eqnarray*}
On deletion of Bill we get
$q$[Dan] := $q$[Joe] += 0 - 1 + 0 = 2 and
$q$[Bill] += -3 - 1 + 1 = 0.

Let us for a moment consider the same query except that we do not group by
$c_1$: That is, the aggregate term for the query is the same, and the delta does not change, but $c_1$ now is not bound. Then simplifying $\dc q$ yields
\[
\pm \AggSum(1, C(c_2, n)) \pm \AggSum(1, C(c_1,n)) + 1.
\]
Suppose we again start with the two tuple database (Joe, USA) and (Bill, USA):
Then $q[\,] = 4$. We add (Dan, USA) and
$\Delta_{+C(\mathrm{Dan}, \mathrm{USA})} q = 2 + 2 + 1 = 5$, changing
$q[\,]$ to 9. We remove Bill and get
$\Delta_{-C(\mathrm{Bill}, \mathrm{USA})} q = -3 - 3 + 1 = -5$, changing
$q[\,]$ back to 4.
\punto
\end{example}




\subsection{Compilation Algorithm}
\label{sec:compilation-alg}


An M3 program consists of a set of triggers of the form
\[
\mbox{{\tt on <action> $R$($\vec{x}$) \{ $s_1$; $\dots$; $s_k$ \}}}
\]
where {\tt <action>} is either {\tt insert into} or {\tt delete from},
$R$ is a relation name,
and the $s_i$ are statements of the form
\begin{equation}
\mbox{{\tt foreach $\vec{y}$ do $m[\vec{x}, \vec{y}]$ $\pm$= $t$}}
%\label{eq:foreach}
\end{equation}
where $\vec{x}$ and $\vec{y}$ are distinct tuples of variables and
$t$ is a term in which all aggregates are constraints-only (or in other
words, functional if-statements).
If $\vec{y}$ consists of zero variables, we omit
{\tt foreach $\vec{y}$ do}.
Let $m_1[\vec{z}_1], \dots, m_k[\vec{z}_k]$ be the map accesses in $t$.
Then $m$, $m_1$, $\dots$, $m_k$ must be pairwise distinct
and the variables in $\vec{z}_1, \dots, \vec{z}_k$ must be a nonoverlapping
subsets of the variables in $\vec{x}, \vec{y}$.
%
For each relation name, there may by multiple insert and delete triggers.


The compilation algorithm employs {\em recursive incremental view
maintenance}. The algorithm Compile($n$, $\vec{b}$, $t$) takes a name $n$ that
represents the query result map, a tuple of bound variables $\vec{b}$ (the map
arguments), and a term $t$ representing the query to be compiled.
For each trigger to be created (that is, for each relation name $R$ of
the schema, there is an insert and a delete trigger),
we apply $\Delta$, compute the monomials of the delta, and then simplify.
Then we extract the non-constraints-only aggregate
subterms of each monomial obtained.
The remainder terms of the extraction
do not contain aggregates, just variables, constants, *,
if-then-else constructs, and map lookups, and we can turn them into
M3 map update statements. (These terms can be e.g.\ read as C or Java rvalue
expressions.)
%
We union together the mappings produced by the calls to ExtractAggregates,
and eliminate duplicates. 
These are the definitions of the auxiliary maps we are using.
We recursively call Compile to create update triggers for the
auxiliary maps as well.


The Compile function is given in Figure~\ref{fig:compilation-algo}. 
SimplifyArgs is a function that takes bound variables that contain
$\vec{b}$ and a statement of the form
{\tt foreach $\vec{x}\vec{y}$ do $q[\vec{x}\vec{y}]$ +=}
$\AggSum(t, \vec{x}=\vec{b})$
and simplifies it to the equivalent statement
{\tt foreach $\vec{y}$ do $q[\vec{b}\vec{y}]$ +=} $t$.
Note: We lift ifs to be able to apply this optimization.



\begin{figure}
\begin{tabbing}
{\bf algorithm} Compile(\=map\_name: string, \\
                  \>map\_args: var list, \\
                  \>t: term) \\
outputs an M3 program \\
{\bf begin} \\
{\bf for each} relation $R$ in the schema,
               pm in $\{+,-\}$ {\bf do} \\
~~\=
  trigger\_args =: \=turn columns names of $R$ into list \\
\>               \>of new argument variable names; \\
\>{\bf for each} $t_i$ in
        RecMonomials($\Delta_{pm R(\mathrm{trigger\_args})} t$) {\bf do} \\
\>~~\=bound\_vars =: trigger\_args $\cup$ map\_args; \\
\>\>($t'_i$, $\Theta_i$) := ExtractAggregates( \\
\>\>~~      Simplify($t'$, bound\_vars), bound\_vars); \\
\>\>s := SimplifyArgs({\tt foreach} map\_args {\tt do} \\
\>\>~~       map\_name[map\_args] pm= $t'_i$), trigger\_args); \\[1ex]
\>\>{\bf if} pm=`+' {\bf then} \\
\>\>~~{\bf output} {\tt on insert into} $R$(trigger\_args) $\{s\}$; \\
\>\>{\bf else} \\
\>\>~~{\bf output} {\tt on delete from} $R$(trigger\_args) $\{s\}$; \\[1ex]
\>$\Theta$ := $\bigcup_i \Theta_i$; /* eliminates duplicates */ \\
\>{\bf for each} $(m[\vec{x}] \mapsto t')$ in $\Theta$ {\bf do}
                 Compile($m, \vec{x}, t'$); \\
{\bf end}
\end{tabbing}

\vspace{-6mm}

\caption{The compilation algorithm.}
\label{fig:compilation-algo}
\end{figure}


Note that the use of Simplify, SimplifyArgs and duplicate elimination
is not necessary for correctness of the compiled M3 programs, but is important
to create small and efficient programs.


\begin{theorem}
Given a query term $t$ and bound variables $\vec{x}$ by which results
are to be grouped,
the output of Compile($m$, $\vec{x}$, $t$) is an M3 program that
correctly maintains the query in map $m[\vec{x}]$ under inserts and deletes.
\end{theorem}



\begin{example}\em
\label{ex:self-join-compile}
Consider the query $q[c_1]$ of
Example~\ref{ex:self-join}, for which we already know
the simplified $\dc q$ from Example~\ref{ex:self-join-delta}.
After simplification and extraction of non-constraints-only aggregates,
the RecMonomials of the insertion/deletion triggers\footnote{Here,
$(\pm 1) * t$ is a shortcut for $t$ in the insertion case and
$(-1)*t$ in the deletion case.} are
$t'_1 = \AggSum((\pm 1) * \mbox{q1}[n], c_1=c)$,
$t'_2 = (\pm 1) * \mbox{q2}[c_1, n]$, and
$t'_3 = \AggSum(1, c_1=c)$
where
\begin{eqnarray*}
\mbox{q1}[n] &\mapsto& \AggSum(1, C(c_2, n)) \\
\mbox{q2}[c_1, n] &\mapsto& \AggSum(1, C(c_1,n)).
\end{eqnarray*}
Using SimplifyArgs, we get the three trigger statements
\[
q[c] \mbox{ {\tt $\pm$=} q1}[n]; \quad\quad
\mbox{{\tt foreach $c_1$ do}} \; q[c_1] \mbox{ {\tt $\pm$=} q2}[c_1, n];
\]

\vspace{-6mm}

\[
q[c] \mbox{ {\tt +=} } 1.
\]
We further have to compile q1 and q2. Since
\[
\dc \mbox{q1}[n] = \dc \mbox{q2}[c,n] = \pm 1,
\]
the compiled M3 program is exactly as shown in Example~\ref{ex:self-join}.
\punto
\end{example}


\nop{
{\em Implementation Advice}.
There have been, in total, five prototypes of our compiler, and
in each subsequent generation, our picture of the problem has become clearer.
The compiler precisely as described in the section has been implemented
in OCAML in less than 1500 lines of code. We caution the reader against
abandoning our choice of using a calculus perspective
in favor of relational algebra with aggregates, or of misunderstanding
the exact role of safe and bound variables as used in this section.
We have made these mistakes in the past, resulting in 
a very difficult, and by about an order of magnitude larger, piece of code.
It may not be easy to see, but we strongly believe that
our choices here are not pedantism,
but key to allowing for concise presentation
and painless implementation.
} % end nop


\subsection{Domains}


We have addressed domains of variables to loop over in foreach statements
in a handwavy fashion in the introduction. Let us now be precise about this
issue.
We can assign a domain $R.A$, where $R$ is a relation name and $A$ is
a column name to each variable.
For instance, given atomic formula Customer$(x,y)$, where
the first column of Customer is cid, variable $x$ has domain Customer.cid.
This annotation or typing of variables can be propagated bottom-up.
Each variable being looped over can be typed in this way.

The M3 programs that our compiler creates are correct if we assume
an a priori given domain of each variable
that we will never have to add to later (while inserts are processed).
A {\tt foreach $\vec{x}$} statement loops over all the distinct tuples
$\vec{x}$ that we can construct from these variable domains.

Unfortunately, it is not a realistic assumption that these variable domains
are known in advance. For practical reasons, we have to start with empty
domains, to which we add when tuples are inserted. However, this means that,
in general, we need to do further work to create initial map values for new
domain elements.


\begin{example}\em
We modify the query of Example~\ref{ex:self-join}. We now ask, for each cid,
for the number of customers from {\em different}\/ nations.
The insert trigger for this query is
\begin{verbatim}
on insert into Customer (cid, nation) {
  q[cid] += q1[nation];
  foreach cid2 do q[cid2] += q2[cid2, nation];
  foreach nation1 do q1[nation1] +=
    if nation <> nation1 then 1 else 0;
  foreach nation2 do q2[cid, nation2] +=
    if nation <> nation2 then 1 else 0
}
\end{verbatim}
which is correct if the inserted cid and nation values are
already in the domains of variables cid2, nation1, and nation2.
But suppose cid is new. 
TODO: FINISH.
\end{example}


\subsection{Key-Foreign Key Join Optimization}


\begin{figure}
\begin{center}
\includegraphics[width=2in]{images/q12_graph.pdf}
\caption{Data-flow graph for the example query.
Light and dashed edges can be optimized out, given foreign key constraints.}
\label{fig:dataflow}
\end{center}
\end{figure}


When relations are joined along a key-foreign key relationship,
then we may assume that the transactional databases providing the updates
enforce their consistency. In particular, this means that we cannot use
a foreign key value until it has been inserted as a key; for instance, in
TPC-H, we cannot insert an order with a customer id for which there is
no customer; we first have to insert a suitable customer tuple.
Conversely, we cannot delete a customer until all its dependent orders
have been deleted.\footnote{Suitable code to implement ON DELETE CASCADE
semantics in M3 can be generated by compilation as well, but this is not
covered here because of lack of space.}
Because of lack of space, we only give an illustrative example which
is suggestive of the general algorithm.


\nop{
Thus, given key-foreign key joins, certain M3 statements that our compilation
algorithm produces are superfluous.
Consider a statement
{\tt foreach $\vec{y}$ do $m_2[\vec{x}\vec{y}]$ $\pm$= $t$}
in an on-insert trigger for relation $R$, .

These can be eliminated -- as an
optimization -- by analyzing the data flow graph of the program.
The nodes of this graph are the
map names of the M3 program and where there is an edge labeled $R$ from
node $m_1$ to $m_2$ if there is an M3 statement
{\tt foreach $\cdot$ do $m_2[\cdot]$ $\pm$= $t$} where
$m_1$ appears in $t$ in an insert or delete trigger for relation $R$.
Note that our compilation approach guarantees that this graph is always
acyclic (even if there are self-joins).
} % nop


\begin{example}\em
\label{ex:TPCH-Q12}
Consider the following query on a TPC-H like schema,
which counts the number of LineItems per customer id.
\begin{verbatim}
SELECT   C.cid, SUM(1)
FROM     Customer C, Order O, LineItem L
WHERE    C.cid=O.cid AND O.oid=L.oid
GROUP BY C.cid;
\end{verbatim}
Here, cid is a key for the Customer relation and oid is a key for the
Order relation, but oid is not a key for LineItem.
The compilation algorithm of Section~\ref{sec:compilation-alg} yields the
following insert triggers:
\begin{verbatim}
on insert into Customer(cid, nation) {
  q[cid] += qC[cid];
  foreach oid do qL[cid, oid] += qLC[oid, cid];
  qO1[cid] += 1
}
on insert into Order(oid, cid, ...) {
  q[cid] += qO1[cid]*qO2[oid];
  qC[cid] += qO2[oid];
  qL[cid, oid] += qO1[cid];
  qLC[oid, cid] += 1
}
on insert into LineItem(oid, ...) {
  foreach cid do  q[cid] += qL[cid, oid];
  foreach cid do qC[cid] += qLC[oid, cid];
  qO2[oid] += 1
}
\end{verbatim}

Consider the data flow graph for this program, which is shown in
Figure~\ref{fig:dataflow} and which illustrates the dependencies between
maps through updates to certain relations (the edge labels). All M3 statements
other than those contributing only bold edges can be removed. For example,
the statement {\tt q[cid] += qC[cid]} in {\tt on insert into Customer}
can be removed because {\tt qC} represents the number of line items for this
customer, and the customer is new. The solid thin lines represent
feasible updates to maps that have become disconnected from the query result
map, and can be eliminated.
The simplification yields the M3 program
\begin{verbatim}
on insert into Customer (cid, ...) { qO1[cid] += 1 }
on insert into Order (oid, cid, ...) {
  qL[cid, oid] += qO1[cid]
}
on insert into LineItem (oid, ...) {
  foreach cid do q[cid] += qL[cid, oid]
}
\end{verbatim}

The delete-triggers are precisely
like the insert-triggers, but with {\tt +=} replaced by {\tt -=}.
\punto
\end{example}





\section{Architecture}
\label{sec:architecture}
\begin{figure}
\begin{center}
\includegraphics[width=3in]{images/Architecture.pdf}
\caption{Cumulus' architecture.}
\label{fig:arch}
\end{center}
\end{figure}
Cumulus consists of three components: a runtime, a \textit{Query Layer}, and a compiler.  A diagram of this architecture is presented in Figure \ref{fig:arch}.  The warehousing runtime accepts relation updates at a coordinator node called the \textit{Switch}, and distributes their data throughout an array of data warehouse storage nodes, or \textit{DW Nodes}.  Though this paper focuses on a runtime with a centralized switch, we discuss a distributed switch implementation in Section \ref{sec:distswitch}.

Adjacent to the runtime is the \textit{Query Layer}, a component that acts as an intermediary between the end user and the dw nodes.  The query layer accepts roll-up and drill-down queries, translates them into the corresponding set of data warehouse lookups, and executes those queries on the warehouse.

Cumulus' final component is a compiler that guides the behavior of the other three components.  The Cumulus compiler takes schemas for one or more input relations and an arbitrary SQL query as input.  The query is decomposed into a set of subqueries, each representing a join over some subset of the input relations.  Each subquery is materialized into a \textit{map}, that is partitioned across the DW nodes.  

Based on these subqueries, the compiler generates a series of \textit{update rules} that translate changes in the input relations to map deltas.  These update rules take the form:
$$+R(x_1, x_2, x_3, \ldots)\ :\ Map\ K[\ldots]\ += Expression$$
That is, when the tuple $(x_1, x_2, x_3, \ldots)$ is inserted into input relation $R$, adjust the target map $K$ by the indicated arithmetic expression over constants, variables ($x_n$), or the contents of other maps.  
 
It is possible for an update rule to modify an an entire cross section of a map.  Such updates are expressed via variables in the update rule not bound to one of the input parameters.  We refer to these as loop variables.  A loop variable occurs exactly twice in an update rule, once as a key in the target map, and once as an key for a map in the expression.  Thus an update rule with loop variables behaves as if it were a set of parallel updates; every update corresponds to one possible valuation of all loop variables selected from the domains of the map they index into.

When a tuple is added to or removed from an input relation, the Switch composes a set of messages parametrized on the tuple's contents and dispatches them into the data warehouse as a sequence of map reads and writes.  Nodes being read from accept and process the requests and forward the responses to the appropriate writers.  We refer to this entire sequence of operations as an update.

\begin{example}
One of the rules the compiler generates for updates to the lineitems table in the example datacube is
{\footnotesize
\begin{eqnarray*}
+lineitems(l\_oid,lateship,latedeliver,shipmode) : \\
Map\ 1[shipmode,latecommit,lateship,spriority,\\ opriority,l\_oid,cid,nation] \\
 += Map\ 5[cid,nation,l\_oid,opriority,spriority]
\end{eqnarray*}}
\noindent Here, Map 1 represents the query output $count(*)$, while Map 5 represents the natural join $customers \bowtie orders$\footnote{Map 5 actually represents the $count(*)$ of the join, grouped by all columns.  However, because $cid$ and $oid$ are keys for each input table, the count is binary; Each entry in Map 5 is either 0 or 1.  It is also possible to treat Map 5 as a map from $(cid,oid)$ to $\{(nation, opriority, spriority)\}$. }.  Note that the variables $cid$, $nation$, $opriority$, and $spriority$ do not appear in the input tuple.  These variables are treated as loop variables; an entire cross-section of Map 1 will be updated by the values stored in the corresponding portion of Map 5.  

The update corresponding to this rule is coordinated by a subset of nodes that store partitions of Map 1.  Each node coordinates updates only for those partitions it stores locally.  Simultaneously, read requests are dispatched by the switch to a subset of the nodes managing partitions of Map 5, and the responses are forwarded to the appropriate Map 1 nodes.  
\end{example}

\subsection{Anatomy of an Update}

\begin{figure}
\begin{center}
\includegraphics[width=1.5in]{images/UpdateStep.pdf}
\caption{Information flow during one map update.}
\label{fig:updatestep}
\end{center}
\end{figure}

A map update begins at the Switch when an input table is modified.  Each input table is associated with one or more triggers, each requiring a write to some range of map values and zero or more reads from different maps.  For each trigger, the Switch identifies and sends a PUT message to each DW Node managing a map being written to.  Additionally, the Switch identifies all DW Nodes managing maps that will be read from and sends a FETCH message to them, requesting the desired map values.  Finally, if required (i.e., if Cumulus is being used standalone, without an underlying database), the switch can log the update to disk for persistence.

The nodes receiving a FETCH request perform the appropriate reads and send their responses to the PUT nodes in a PUSH message.  Upon receipt of all necessary PUSH messages, the PUT nodes compute the update expression and modify the affected maps.  An illustration of the complete message-passing process is presented in Figure \ref{fig:updatestep}.  

\subsubsection{Trigger Dispatch}
Maps are partitioned along dimensional axes when they grow beyond the capacity of a single node, akin to the partitioning done in grid files\cite{318586}.  To streamline the dispatch of messages to component nodes, the switch pre-generates a spatial index for each template, similar to the grid file directory.  Every entry in the spatial index contains a set of PUT and FETCH messages.  When the template is triggered, the tuple's values are used to index into the spatial index and the corresponding messages are parametrized and sent.  The algorithm for generating the spatial index is presented in Figure \ref{alg:dispatch}.

\begin{figure}
\begin{algorithmic}[1]
\STATE $Msgs \leftarrow \emptyset$
\FORALL{Update $U \in$ Trigger}
	\STATE Region $Reg = \pi_{U.target.loop\_vars} \left(index(U.target\_map)\right)$
	\FORALL{Partition $\{P | P\in U.target\_map \wedge (P \cap Reg \neq \emptyset)\}$}
		\STATE $Reads = \emptyset$
		\FORALL{Ref $R \in get\_map\_refs(U.expression)$}
			\STATE $RReg \leftarrow \pi_{R.loop\_vars} P$
			\STATE $RPart \leftarrow \{RP | RP\in R.map \wedge (RP \cap RReg \neq \emptyset)\}$
			\STATE $Reads \leftarrow Reads \cup \{(ReadP.node, R)\}$
		\ENDFOR
		\STATE $Reads \leftarrow group\_by\_node(Reads)$
		\STATE $Msgs \leftarrow put(U, P, Reads.size)$
		\FORALL{$($Node $N, \{$Ref $R\}) \in Reads$}
		  \STATE // PUSH results to $P.node$
			\STATE $Msgs \leftarrow fetch(N, \{R\}, P.node)$
		\ENDFOR
	\ENDFOR
\ENDFOR
\end{algorithmic}
\caption{The Switch's trigger dispatch algorithm}
\label{alg:dispatch}
\end{figure}

Looping updates require the Switch to match corresponding read and write partitions.  The correspondence is obtained by identifying intersections between partitions of the target map that are affected by the update, and those of each map in the update expression.  This is equivalent to a join over components of the spatial index stored at the Switch.  Loop-free updates are a special case of this, where only one partition is required from each map.  The trigger dispatch algorithm is shown in Figure \ref{alg:dispatch}.

\subsubsection{Get Collation}

FETCH responses, or PUSH messages for an update are sent to the node managing the partition being updated.  Having received the number of FETCH messages sent by the switch with the PUT message, the destination node can buffer PUSH messages until all have arrived.  At this point, if the update is loop-free, the destination node simply uses the contents of the PUSH messages to evaluate the update expression.

\begin{figure}
\begin{algorithmic}[1]
\STATE Given: Update $U$
\FORALL{Ref $R \in U.expression$}
	\STATE $Inputs \leftarrow Inputs \cup (R, \emptyset)$
\ENDFOR
\FORALL{$($Ref $InR,$ Value $V) \in \cup(msg_{PUSH})$}
	\FORALL{$(R, Table) \in Inputs$}
		\IF{$check\_match(InR, R)$}
			\STATE $Table \leftarrow Table \cup (InR.keys, V)$
		\ENDIF
	\ENDFOR
\ENDFOR
\FORALL{$(K,\{V\}) \in $ JOIN $ (Keys, Val) \in Inputs$}
	\STATE $Target = bind\_vars(U.target \leftarrow Keys)$
	\STATE $apply(Target, bind\_vars(U.expression \leftarrow \{V\}))$
\ENDFOR
\end{algorithmic}
\caption{The DW Node's collation algorithm}
\label{alg:collation}
\end{figure}

If the update requires a loop, the destination node must do some processing.  The node first generates a set of tables, one for each map reference in the update expression.  Arriving map values populate tables that correspond to any map reference matching the value's keys, where loop variables act as wildcards.  When all values have been received, the destination node computes the natural join of all of the generated tables, effectively producing one update value for every assigned value in the domain of all of the loop variables.  The loop-free update is a special case of this where each generated table contains only one row.  The collation algorithm is shown in Figure \ref{alg:collation}

\subsection{Update Consistency and Garbage Collection}

Cumulus' delta-encoding approach allows it to be agnostic to the order in which it processes input tuples.  However, this flexibility requires that an update's effect on the warehouse be logically atomic.  Atomicity is of particular importance, since the set of update rules triggered by a given update switch typically span the breadth of the data-dependency graph; Updates to any input relation typically have at least one read to a map updated by every other relation.

As the clearinghouse for updates, the Switch presents an ideal point for generating a total ordering over all tuples to be inserted into the warehouse.  The Switch maintains a version number for each node, incremented on every PUT dispatched to the node hosting it.  FETCHes sent to that node are tagged with the version number of the last PUT sent to the node, while PUTs are dependent on the completion of the prior PUT.

Data warehouse nodes ensure consistent evaluation of updates by buffering PUT requests received out of order, or prior to the completion of an earlier PUT request.  Similarly, FETCH requests for a particular version are buffered until the corresponding PUT request has completed, if it hasn't already done so.

Periodically, the switch queries each node for the completion status of pending PUT requests.  For each node, it computes the lowest FETCH version number for a read associated with an incomplete PUT request.  The Switch then disseminates these version numbers throughout the warehouse with a COMMIT message.  When a node receives a COMMIT, it snapshots its maps by discarding all updates for values that were overwritten before the commit version.

\section{Compilation}
\label{sec:compilation}

\subsection{Update Rules}
- Translating SQL into Update Rules

- Update DAGs/Data flow graphs

- Exploiting Foreign Keys

- Cascading Map Rejection

\begin{figure}
\begin{center}
\includegraphics[width=2.5in]{images/q12_graph.pdf}
\caption{Data-flow graph for the example query.  Light and dashed edges can be optimized out, given cascading delete foreign key constraints.}
\label{fig:dataflow}
\end{center}
\end{figure}

\subsection{Layout}
How do we partition data (the layout)?  Where do various bits of code get executed live?  What kind of runtime analytics do we need to collect to manage the data layout on the fly? 


\section{Experiments}
\label{sec:experiments}

\section{Optimizations}
\label{sec:optimizations}

\subsection{Distributing the Switch}
\label{sec:distswitch}
The role of the switch is to provide a synchronization mechanism for the updates.  Specifically, the delta encoding approach used by the update rules requires that all update rules be applied to a consistent snapshot of the maps.  The current implementation of Cumulus performs this synchronization at a single node.  However, limiting the switch to only one node creates a scaling bottleneck.  

Despite the cloud computing mantra of rejecting consistency, this application requires it.  Complex locking protocols have poor scaling performance, so a simpler, lock-free protocol is required.  We achieve this goal by introducing the notion of \textit{pipeline scheduling}.  Pipeline scheduling exploits the acyclicity of compiler-generated data-flow graphs to allow nodes to correctly interleave and process update rules with only limited network overhead and processing latency.

\subsection{Pipeline Scheduling}
Loose clock synchronization between nodes is assumed, and allows time to be partitioned into a sequence of numbered ticks, each lasting on the order of seconds.  When a set of updates is triggered, a switch node sends tentative put requests to all participating nodes.  These nodes respond with their current tick counter, and the switch forwards the maximum returned tick to all nodes.

The updates are considered to have been posted at the maximum returned tick.  However processing is deferred for a number of ticks equal to the depth of the map being updated.  The result is a data-flow process resembling a parallelized CPU pipeline.

During a given tick, all updates scheduled for processing are evaluated.  Once all updates scheduled for the tick have completed, the node responds to FETCH requests for the tick with PUSH messages.  Recall that each edge in the data flow graph represents a FETCH request for a specific update.  The difference in depth between the edge's ends is the number of ticks in advance of the PUT that the response is sent.  

For example, in Figure \ref{fig:dataflow} without removing any edges, $Depth(M1) = 2$, $Depth(M2) = 1$, and $Depth(M4) = 0$.  Updates to map $M1$ scheduled for processing during tick 4 would receive data from map $M2$ during tick 3, and from map $M1$ during tick 2.

%\subsection{Synchronizing Ticks}
%Ensuring that nodes are operating on the same tick poses two challenges.  First, the nodes must 
%
%Clock tick synchronization requires two pieces of functionality: Asserting 
%
%Loose clock tick synchronization requires two stages.  
%
%is achieved through a cascading gossip protocol.
%
%
%\subsection{Distributed Queries}
%

\section{Conclusions}
\label{sec:conclusions}

\bibliographystyle{abbrv}
\bibliography{main}




%\begin{figure}
%\begin{center}
%\begin{itemize}
%\item Middleware
%\begin{itemize}
%\item Query(Query) 
%\end{itemize}
%
%\item Warehouse Nodes
%\begin{itemize}
%\item Get(\{Entry\}, Version)
%\item Fetch(\{Entry\}, Version, Destination)
%\item Push(\{(Entry, Value)\}, Version)
%\item Put(Template, \{Params\}, Version [, Fetches])
%\end{itemize}
%
%\item Switch
%\begin{itemize}
%\item update(Relation, \{Params\})
%\end{itemize}
%
%\end{itemize}
%\caption{Cumulus' API}
%\label{fig:api}
%\end{center}
%\end{figure}

\end{document}  

\documentclass{sig-alternate}

\usepackage{amsmath,amssymb}
\usepackage{stmaryrd}
\usepackage{graphicx}
\usepackage{color}

\newcommand{\comment}[1]{}
\newcommand{\tinysection}[1]{\noindent{\bf #1.}}
\newcommand{\tuple}[1]{{\langle#1\rangle}}
\newcommand{\todo}[1]{\textcolor{red}{#1}}
\newcommand{\note}[1]{\textcolor{blue}{#1}}


\begin{document}
%\title{DBToaster: A Dynamic Data Management System}
\title{Dynamic Data Management Systems}
\numberofauthors{3}
\author{
\alignauthor
Oliver Kennedy\thanks{Also Dept.\ of Computer Science, Cornell University.}\\
%    \affaddr{\'Ecole Polytechnique F\'ed\'erale de Lausanne} \\
    \affaddr{EPFL} \\
    \affaddr{Lausanne, Switzerland}
    \email{oliver.kennedy@epfl.ch}
\alignauthor
Yanif Ahmad\\
    \affaddr{Johns Hopkins University}
    \affaddr{Baltimore, MD}
    \email{yanif@cs.jhu.edu}
\alignauthor
Christoph Koch\\
    \affaddr{EPFL} \\
    % \affaddr{\'Ecole Polytechnique F\'ed\'erale de Lausanne} \\
    \affaddr{Lausanne, Switzerland}
    \email{christoph.koch@epfl.ch}
}
\maketitle

\begin{abstract}
This paper calls for a new breed of systems which
we call dynamic data management systems (DDMS). In a nutshell,
a DDMS manages (very large) dynamic data structures with 
agile, frequently fresh views, and a facility for monitoring
these views and triggering application-level events.
%
We motivate DDMS with applications in large-scale data analytics
including data warehousing and high-frequency algorithmic trading
and compare them to more traditional systems such as DBMS and
data stream processors.
%
We describe the DBToaster project, which is an ongoing effort to
develop a prototype DDMS system. We describe the main architecture
design, techniques for high-frequency incremental view maintenance,
query optimization, storage, scaling up by parallelization, and
the various key challenges to overcome to make DDMS a reality.
\end{abstract}

\section{Introduction}


Recent years have seen the beginning of a paradigm shift in data management
research from incrementally
improving decades-old database technology
%
% (particularly, OLTP) -- System R and
% the long line of systems that have followed it --
%
to questioning established
architectures and creating fundamentally different, more lightweight systems
that are often domain-specific
(cf.\ e.g. \cite{DBLP:conf/vldb/StonebrakerMAHHH07,DBLP:journals/pvldb/KallmanKNPRZJMSZHA08}).
Part of the impetus for this change was given by 
potential users such as scientists and the builders of
large-scale Web sites and services such as Google, Amazon, and Ebay,
who have a need for data management systems but have found current databases
not to scale to their needs.
One can observe a trend to disregard database contributions
in these communities \cite{dbcolumn, DBLP:conf/sigmod/PavloPRADMS09}, and to build lightweight systems based on
robust technologies mostly pioneered by the operating systems and distributed
systems communities, such as large scale file systems, key-value stores, and
map-reduce
\cite{DBLP:journals/cacm/DeanG08, DBLP:journals/tocs/ChangDGHWBCFG08}.
Further impetus has resulted from the current need to develop data management
technology for multicore and cloud computing.
%
%, and by the example given by a
%number of innovative recent database startup companies that develop
%databases based on new architectures.
%\footnote{Examples are Vertica and Paraccel, who build column stores,
%and Greenplum, Asterdata, and Netezza, who take databases into the Cloud.}

There is a recent tendency among pundits outside the database community to
contest the need for powerful queries, and to
think of key-value stores -- with only the power to look up data by
keys -- as (much more efficient) database query engines.
%
%It shall not be denied that,
%with clever engineering, a surprising range of problems can be solved
%using key-value stores.
%
However, expressive query languages such as SQL do not cease to have
important applications and a substantial user base.
Alas, we do not know how to process SQL queries on updateable data
using a system as lightweight as a key-value store.

This paper contributes a fundamental and versatile building block for
enabling new, more lightweight and nimble data processing systems based
on SQL aggregation que\-ries. We believe that our contribution
constitutes an important
step towards achieving the contradiction in terms mentioned
above: executing complex aggregation queries on updateable data
using little more than a key-value store.

At the heart of our approach is a new aggressive recursive incremental
view maintenance mechanism.
In most traditional database query processors, the  basic building blocks of
queries are large-grained operators such as joins.
Our approach is based on compilation, reducing
queries to programs that are not based on classical query operators.
A large class of
SQL aggregation queries can be compiled down to very simple message
passing programs that incrementally maintain materialized views of
the queries. These message passing programs keep a hierarchy of map data
structures (which may be served out of a key-value store) up to date
and can share computation in the case that multiple aggregation queries (e.g.,
a data cube) need to be maintained.  Most importantly, though, these message
passing programs can be massively parallelized to the degree that the updating
of each single result aggregate value can be done in constant time on normal
off-the-shelf computers.\footnote{That is,
we assume that addition and multiplication
of {\em two numbers} can be performed in constant time, which is true for
%
%bounded precision numbers such as 
%
standard base types such as int and float,
but we assume no unrealistic models such
as aggregators with unbounded fan-in, as used in some theoretical models of
parallel computation. Our implementation indeed incrementally maintains
individual aggregate values in constant time. Note that since there are
usually many more aggregate values to maintain than there are processors,
this does not mean that each update is processed in constant time.
``Constant time'' is with respect to the size of the data, not the compiled
query.}
To the best of our knowledge, it was not known before
that this is possible.

In comparison, no such constant-time parallel processing technique
is known for nonincremental query
evaluation: Indeed, it is unlikely to exist.\footnote{Constant-time
bounded fan-in nonincremental
parallel processing is known not to be possible for
the class of queries we address,
unless the complexity class TC0 collapses into NC0, which it is not known
to do \cite{Joh90}.} Classical incremental view maintenance approaches, which
express the delta (=change) to a query result given an update again
as a (slightly simpler) query, fare no better: Generally,
given a query, there is another query whose delta is the first query.
Thus, classical incremental view maintenance has the same limits to
parallelization as nonincremental evaluation.


\subsection{Message passing programs}


We compile SQL aggregation queries to {\em map maintenance
message}\/ (M3) programs. An M3 program consists of a set of triggers of
the form
\[
\mbox{{\tt on insert into $R$($\vec{x}$) \{
($\vec{y}$:$D_{\vec{y}}$) $m[\vec{x}, \vec{y}]$ += $s$
\}}}
\]
or
\[
\mbox{{\tt on delete from $R$($\vec{x}$) \{
foreach $\vec{y}$ do $m[\vec{x}, \vec{y}]$ -= $s$
\}}}
\]
where $R$ is a relation name,
$\vec{x}$ and $\vec{y}$ are distinct tuples of variables and
$s$ is either a term or of the form
{\tt if $\phi$ then $t$ else 0}, where $t$ is a term.
Terms are built from addition, multiplication,
constants, variables from $\vec{x}$, external function calls $f(\vec{z})$,
and map accesses $m_1[\vec{z}_1], \dots, m_k[\vec{z}_k]$ where
$m$, $m_1$, $\dots$, $m_k$ are pairwise distinct
and the variables in $\vec{z}_1, \dots, \vec{z}_k$ are a nonoverlapping
subsets of the variables in $\vec{x}, \vec{y}$.
Conditions $\phi$ are conjunctions of comparisons $t'' \;\theta\; t'''$,
where $t'',t'''$ are terms without map accesses and
$\theta \in \{ =,<,\le,\neq \}$.
If $\vec{y}$ consists of zero variables, we omit
{\tt foreach $\vec{y}$ do}.
For each relation name, there may by multiple insert and delete triggers.

M3 programs can be read as straightforward pseudocode.
There are subtle issues to be discussed later about the domains of 
variable tuples
to be iterated over by foreach loops. Let us for now assume that
{\tt foreach $\vec{x}$ do $(\dots)$} iterates over all distinct tuples of
values currently in the
database, and that map values $m[\vec{x}]$ for $\vec{x}$ containing newly
inserted values are initially zero.


\begin{example}\em
\label{ex:TPCH-Q12}
Consider the following query on a TPC-H like schema,
which counts the number of LineItems per customer id.
\begin{verbatim}
SELECT   C.cid, SUM(1)
FROM     Customer C, Order O, LineItem L
WHERE    C.cid=O.cid AND O.oid=L.oid
GROUP BY C.cid;
\end{verbatim}
Here, cid is a key for the Customer relation and oid is a key for the
Order relation, but oid is not a key for LineItem.
Our compiler translates this query to the M3 program
\begin{verbatim}
on insert into Customer (cid, ...) { qO1[cid] += 1 }
on insert into Order (oid, cid, ...) {
  qL[cid, oid] += qO1[cid]
}
on insert into LineItem (oid, ...) {
  foreach cid do q[cid] += qL[cid, oid]
}
\end{verbatim}

In this and the following example, the delete-triggers are precisely
like the insert-triggers, but with {\tt +=} replaced by {\tt -=}.
Thus, to save space, the deletion triggers are omitted.

Let us ignore parallelization first.
It is not hard to see that this trigger program correct maintains the
query result, for each distinct {\tt cid} in Customer.cid, as {\tt q[cid]}.
(The maps {\tt qO1} and {\tt qL} are auxiliary.)
We assume that
there are no cascading deletes and, for instance, before we can delete an
Order, we have to delete all associated lineitems.
\punto
\end{example}


{\em Parallelization}.
The syntax of statements
\begin{equation}
\mbox{{\tt foreach $\vec{y}$ do $m[\vec{x}, \vec{y}]$ $\pm$= $s$}}
\label{eq:foreach}
\end{equation}
is misleading in that it suggests a loop --
that a nonconstant amount of work is needed to bring aggregate
values up to date. Of course, polynomial amounts of work are in fact need,
but only because in general there are many aggregate values -- in a
map representing the result of a group-by query or in an auxiliary map --
to be maintained. In fact, each statement of form (\ref{eq:foreach})
writes each value $m[\vec{x}, \vec{y}]$ only once and admits
{\em embarassing parallelism}: $m$ can be partitioned across many machines
that share the work.

Assume that the storage of individual maps
is partitioned across several machines. To execute a statement of form
(\ref{eq:foreach})
in a trigger invocation with arguments $\vec{x} = \vec{a}$,
where $s$ uses map lookups $m_i[\vec{x}_i, \vec{y}_i]$,
each node storing a value $m_i[\vec{a}_i, \vec{y}_i] = v$,
for $\vec{y}_i$ arbitrary, sends the message
$m_i[\vec{a}_i, \vec{y}_i] = v$ to the node managing value
$m[\vec{a}, \vec{y}]$. This way, that node receives all the values it
needs to update all $m$ values it represents.
Of course this requires a suitable protocol to ensure overall consistency
and that the right versions of map values are read and written in the right
order.



\begin{example}[star-join decomposition]\em
\label{ex:ssb}
Con\-sider \\ a simplified version of the star schema
benchmark (SSB) schema with relations Date(\underline{datekey}, year),
Part(\underline{partkey}, partcat), where partcat stands for a part category,
and LineOrder(datekey, partkey, revenue), which may contain duplicate tuples.
The query asks for the total revenues grouped by year and part category.

\begin{verbatim}
SELECT   P.partcat, D.year, SUM(revenue)
FROM     Date D, Part P, LineOrder L
WHERE    D.datekey=L.datekey
AND      P.partkey=L.partkey
GROUP BY P.partcat, D.year;

on insert into Date (datekey, year) {
  mPL[datekey, year] += 1
}
on insert into Part (partkey, partcat) {
  mDL[partkey, partcat] += 1
}
on insert into LineOrder (datekey, partkey, revenue) {
  foreach (partcat, year) do
  m[partcat, year] += revenue
                    * mDL[partkey, partcat]
                    * mPL[datekey, year]
}
\end{verbatim}

Observe how, on insertion into LineOrder, the code for incrementally
maintaining the query result {\tt m} decomposes into
two parts with disjoint variables, 
{\tt mDL[partkey, partcat]} and {\tt mPL[datekey, year]}.

The maps mPL and mDL have value at most
one at each position because datekey and partkey are keys for Date and Part,
respectively.

On insert into LineOrder, given values for
datekey and partkey, we instruct nodes
to send their {\tt mDL[partkey, x]} and {\tt mPL[datekey, y]} values,
for any {\tt x} and {\tt y},
to nodes maintaining {\tt m[x, *]} and {\tt m[*, y]}, respectively.
A node managing {\tt m[u, v]} receives, possibly from distinct nodes,
{\tt mDL[partkey, u]} and {\tt mPL[datekey, v]}
and can increment {\tt m[u, v]} by
{\tt revenue*mDL[partkey, u]*mPL[datekey, v]}.
%
%We only send nonzero {\tt mDL[partkey, x]} and {\tt mPL[datekey, y]} values,
%but at least an empty message so that the node managing m knows that it does
%not have to wait for anything more.
\punto
\end{example}


\begin{example}[self-join]\em
\label{ex:self-join}
We now ask, for each customer id (cid),
for the number of customers of the same nation (including the customer
identified by cid in the count).
\begin{verbatim}
SELECT   C1.cid, SUM(1)
FROM     Customer C1, Supplier C2
WHERE    C1.nation = C2.nation
GROUP BY C1.cid;
\end{verbatim}
The compiler produces the following on-insert trigger:
\begin{verbatim}
on insert into Customer (cid, nation) {
  q[cid] += qC1[nation];
  foreach cid2 do q[cid2] += qC2[cid2, nation];
  q[cid] += 1;
  qC1[nation] += 1;
  qC2[cid, nation] += 1
}
\end{verbatim}
The on-delete trigger is just like the on-insert trigger with {\tt +=}
changed to {\tt -=} everywhere other than in the third statement
({\tt q[cid] += 1}), which remains unchanged.
We will establish later that this M3 program is indeed correct.

For now, we challenge the reader to find a 
fundamentally different (ideally, simpler) way to
perform the incremental maintenance of {\tt q}
which has the M3 property of embarassing parallelism, with each value
to be updated only requiring a constant amount of work.
Examples~\ref{ex:TPCH-Q12} and \ref{ex:ssb} were chosen for simplicity,
but we believe that this example shows that creating M3 programs in general
is nontrivial.
\punto
\end{example}


It shall be emphasized that for each of the examples of this section,
and the paper as a whole, our compilation approach produces exactly
the M3 programs shown.


The fragment of SQL queries that we can compile to M3 essentially comprises
SUM-agg\-regation queries with group-by.
COUNT and AVG queries can be defined by arithmetic expressions over these.
We exclude MIN and MAX queries, aggregation
nested into FROM or WHERE clauses, the DISTINCT and HAVING keywords,
outerjoins,
and the relational difference operation. At the end of this paper, we will
discuss which of these features can be added without fundamental difficulties.


\subsection{The Cumulus System}


We have developed a system, Cumulus\footnote{A Cumulus cloud is an
aggregation cloud, thus the name.}, that parallelizes the
execution of M3 programs in a cluster or computing cloud.
Cumulus performs incremental maintenance of exact aggregation views online, and
executes an efficient
protocol to ensure consistency of map data and query results,

While it is no fundamental requirement of our compilation
approach, we have chosen to use the resources of the cloud to maintain
the data in main memory, allowing for very low latency updating and querying.
At the time of writing this,
Terabyte-sized memory chips (DIMMs and flash) have already been announced by
manufacturers, and already now, large data warehouses
can be run in main memory in the cloud, where additional hardware costs
(main memory is more expensive per TB than hard disks) are
offset by greater robustness of the system, lower maintenance
costs, lower heat production \cite{1154557}, and of course by of orders
of magnitude better speed and latency characteristics.

The Cumulus protoype aims at demonstrating our results in the context of
pushing OLAP into the cloud.
Cumulus automates  the process of  creating, loading,
and  maintaining  in-memory  data  warehouses.
(Optional logging of updates to secondary storage
for persistency is supported.)
Cumulus targets OLAP applications  that perform real-time analytics of
relational data.  By feeding it an SQL query, Cumulus's infrastructure
becomes linked to  a set of OLTP databases.  Cumulus  keeps the
data warehouse synchronized with the source databases via an
update stream. It achieves synchronization {\em in realtime}
through parallelization, keeping data in main memory, and our approach of
query processing by message passing.


\subsection{Contributions and Structure of the Paper}


Our main technical contributions are as follows.
\begin{itemize}
\item
We present M3, a massively parallelizable language
for message passing programs that can be used to incrementally maintain
SQL aggregation queries.

\item
We describe our compiler for translating SQL aggregation queries to M3
programs. Our compilation technique is based on a novel, aggressive, recursive
form of incremental view maintenance.

\item
We present Cumulus, our system for exact online aggregation in realtime.
We describe the Cumulus message passing protocol, which assures
consistency of the maps using only few messages, and
infrastructure, and show how it can be used to efficiently distribute the
processing and storage requirements of query processing and
the incremental maintenance of large aggregate views and datacubes.

\item We show evidence for the scalability of our approach by examining the
performance of Cumulus on examples drawn from the TPC-H\cite{tpch2008}
benchmark. 
\end{itemize}


The remainder of this paper is organized as follows.
Section \ref{sec:compiler} describes our SQL to M3 compiler.
In Section \ref{sec:architecture}, we provide an overview of Cumulus's online
infrastructure and discuss how data is managed within that infrastructure.
Section \ref{sec:experiments} presents
experimental results that demonstrate the viability and scalability of
Cumulus.
Section \ref{sec:relatedwork} discusses related work.
The paper concludes with Section \ref{sec:conclusions}







\section{Data Management By State Machines}


\begin{figure}
(a) DDMS: show state machine, with databases as states and updates as transitions.

(b) DBMS: database sits and waits for queries to arrive; answers them.

(c) Data stream processor: Set of sitting queries; a stream of data passes by.

\caption{Data management systems architectures: DDMS vs. DBMS vs. data stream processors.}
\end{figure}



A DDMS can be abstractly modeled as a(n infinite) state machine in which the current repository is the current state of the state machine that undergoes a transition whenever an update is applied to it.

Definitions.
\begin{itemize}
\item
states = the database at different points in time. A state is a relational database.

\item
transitions/updates = single-tuple or bulk updates to base relations, but not single CPU instructions or individual writes to memory. A single-tuple update (to a base relation) may well require many changes to the visible relations/data objects of the database state

\item
schema: visible schema (e.g. materialized views of interest) plus auxiliary schema (e.g. base relations that we do not want to monitor plus auxiliary data structures such as auxiliary materialized views/lower levels of a DBToaster hierarchy and indexes)
\end{itemize}


The state machine abstraction leads to new algorithmic ideas. DBToaster is an example. Also, when the idea is to precompute the state transition function of the state machine, to make it as efficient as possible at runtime, compilation is a natural way to go that will not look like a disconnected idea.


Structure of this section:
\begin{itemize}
\item
State machine abstraction

\item
Programming model: Boolean views are events, which trigger application code

\item
Architecture diagram:
Compiler/Optimizer: produces low-level view maintenance code.
Update stream.
Event notification facility.
Event notification by invocation by the view maintenance code?
Ad-hoc querying in client-side library?

\item
System description.
This really cannot be understood if taken out of context and should be moved to the following sections.
\begin{itemize}
\item
Query optimization: The next section describes a method of incremental view maintenance that relies on materializing multiple layers of auxiliary views. This trades off view maintenance time cost against space cost. The optimizer will exploit the potential to save space by  deciding which auxiliary view layers to materialize and which to leave implicit. It will also perform
multi-view optimization, deciding which auxiliary views from different visible views can be merged.

What do we say about the structural recursion optimization, and where do we say it?

\item
Low-level data structures: we will describe the multi-level hash table data structure in section 4. Work on parallelization will be required. Our data structures are a bit unusual since they represent exclusively aggregates and their values are exclusively numerical. It is a conseqence of our approach that loops in query processing are always over a set of complete dimensions of the multi-dimensional table data structures we use; thus all our loops are naturally implemented as full scans over these dimensions. However, many fields in these tables will be zero and indexing or compression could be employed to omit scanning over all-zero areas.
\end{itemize}
\end{itemize}






\section{Compiling the state transition function; Incremental View Maintenance}
\label{sec:dbtoaster}

The \compiler\ project investigates processing techniques and architectural
design for dynamic data management systems. Its two core research themes are
that of incremental query processing, and program synthesis where we can tailor
the design of system internals to yield lightweight, efficient data management
tools. Algorithmic trading, where specialized engines abound, is a natural use
case for \compiler. High-frequency data ingestion and the competitive advantage
of low latencies has led to widespread use of low-level code and processing
hardware such as FPGAs. We describe the computational and software development
aspects of the \compiler\ platform and toolchain below.

\tinysection{Dynamic Data Management}
\compiler\ constructs dynamic data management tools that process database
updates as incrementally as possible. Its query model is identical to that of
view maintenance~\cite{griffin-sigmod:95}, where given a SQL query on a
database, it generates a program to maintain query results as the database
changes. Its novelty lies in the nature of this maintenance program, and its use
of \textit{agile} views and \textit{multi-level} view maintenance.

Like view maintenance, \compiler\ materializes a query as a view, and maintains
the view using a (first-level) \textit{delta} query. Going beyond view
maintenance, we observe that a delta query is simply a parameterized SQL query.
This first-level delta query can also be materialized, and subsequently
maintained by a second-level delta query. There are a bounded number
of higher-level delta queries we can construct. Thus query processing works by a
sequence of maintenance statements, where each statement maintains a
\textit{k}-level view based on some computation applied to a
\textit{k+1}-level view. Further details on the compilation process and
structure of incremental programs can be found in~\cite{KAK2011,koch-pods:10}.
\comment{
As an example, \compiler\ turns the following query:

\begin{verbatim}
select sum(a.vol-*b.vol)
from bids b, asks a where b.price=a.price;
\end{verbatim}

\noindent into an imperative style program as shown below.

\begin{verbatim}
double q;
hash_map<double, double> c_b, sv_b, c_a, sv_a;
// c_b,c_a = count per price for bids, and asks.
// sv_b,sv_a = total vol per price for bids, and asks. 

on_insert_asks(double price, double vol): 
 q += vol * c_a[price] - sv_a[price];
 cb[price] += 1; sv_b[price] += vol;

on_insert_bids(double price, double vol):
 q += sv_b[price] - vol * c_b[price];
 sv_a[price] += vol; c_a[price] += 1;
\end{verbatim}
}

\comment{
\tinysection{Query Engine Compilation and Synthesis}
The \compiler\ compilation framework uses two internal representations to
synthesize stream engines, the first is a query calculus used to compute
higher-level delta queries as mentioned above and for a variety of
simplifications relating to constant folding, unification, variable elimination
and factorization. The second representation is a collection-oriented functional
language, extended with constructs such as persistence and resource annotations
to better utilise the hardware for computations performed by queries.

\todo{SQL queries, preaggregation, construction of incremental plan 
(polynomialization, delta transformation, simplification, repeat), low-level
compilation}

\todo{Diagram of platform: query, compiler, code generator backends, use of
generated code in runtimes}
}

\comment{
In its current incarnation, \compiler\ acts as a one-shot compiler that
transforms standard SQL queries into incremental programs as described above.
}
\compiler\ provides an extensible, retargettable code generator that currently
has support for running incremental programs in a variety of imperative,
functional and query languages (C++, Java, OCaml, pgplsql, PLSQL), as well as on
massively parallel processing infrastructures (Hadoop, and DBToaster Cumulus, a
fine-grained shared nothing stream engine under development). In this
demonstration, we will compile SQL queries to Java for our coarse-grained stream
engine, Jasper.


\tinysection{Programming with DBToaster} 
The benefits of type-safety, analysis and optimization opportunities offered by
language integrated queries and embedded domain specific query languages has
greatly improved the development process for applications interacting with data
management systems. To this day however, the focus has been on using databases
either as a persistence layer~\cite{wiedermann-popl:07}, or as
coprocessors~\cite{grust-sigmod:09}. The \compiler\ project faces outwards from
data management techniques, viewing such trends as an opportunity to export and
embed query processing, optimization, indexing and storage, transaction
management, and concurrency directly into application code.

\comment{
\note{Comment on the advantages of native code querying, esp. in terms
of UDFs (aka abstraction in the PL literature).}
}


Our current focus is on main-memory view maintenance. \compiler\ uses an
embedded SQL DSL called \dsl\ (\dslurl) to integrate with application code
written in \targetlang. As an example snippet, assuming we have a
\texttt{SOBI} method as described in Section~\ref{sec:intro} defined using the
\texttt{query.from(\ldots).where(\ldots).list(\ldots)} style syntax of \dsl
\footnote{\dsl\ supports a larger subset of SQL than SPJAG queries,
we use this form to generically illustrate the comprehension composition
syntax of language integrated queries.}\ , we route orders based on connection
state:

\begin{verbatim}
HashMap<String, ConnectionState> conns =
  openConnections({"NASDAQ"; "NYSE"});
OrderAction action = SOBI(exposure, funds, threshold);
if ( conns.get("NASDAQ").latency >
       conns.get("NYSE").latency )
  action.execute(conns.get("NASDAQ"));
else action.execute(conns.get("NYSE"));
\end{verbatim}

\dsl\ generates SQL query strings in a type-safe manner and executes them
over a \driver\ connection. We are developing a bare-bones \driver\ driver to
support the execution of SQL queries, which involves (just-in-time) query
compilation via the \compiler\ compiler into \targetlang\ code, dynamic loading
of the \targetlang\ classes corresponding to the newly created query engine, and
finally engine execution.


Applications can access
result sets which are internally backed by the data structures maintained by
\compiler\ engines. In addition to the standard pull-based interfaces provided
by the JDBC API, we provide an API for push-based notifications and application
code execution via callbacks. 
\comment{
Other event-driven mechanisms that we will develop
in the future include proxy datastructures and futures \todo{[CIDR]}.
}
To summarize, in this demonstration, we will show how \compiler\ can be used to
create a pure Java application that embeds an in-process relational stream
engine.


\tinysection{The DBToaster Runtime}
In addition to supporting direct application use of queries, \compiler\ provides
a query runtime to facilitate managed execution of queries, in particular to
provide a coarse-grained approach to scaling query workloads by performing
resource allocation and distributed execution. The \compiler\ runtime is a
distributed stream processing engine called \spe\ which has been inspired by
earlier efforts on the Borealis project~\cite{borealis-design:05}. \spe\ is
implemented in Java making it well suited to JVM-based target
languages for \compiler. Jasper internally invokes \compiler\ to generate
operators. Our operators are arbitrarily complex queries that may
perform nesting, multiway joins, essentially the full fragment of SQL
supported by \compiler, enabling the specification of operators or
\textit{modules} in a declarative language. \spe\ can provide window query
functionality around modules (which simply invoke insert or delete triggers in
\compiler\ code), as well core runtime and distribution mechanisms such as
checkpointing, or migrating operators (all \compiler\ code includes
automatic serialization of internal state).

\comment{
\todo{What kind of coupling do we provide for application and queries with this
managed runtime? Are apps also compiled into migratable modules, or do we
provide a layer of indirection at their interaction to let us do whatever we
want under the hood?}
}



\section{Managing Storage in DBToaster}
A DDMS is created with knowledge of the workload it will be used for; though the set of transition functions may be infinite, there is a finite set of classes of transition function that we can analyze, not only to simplify incremental computation, but also to streamline the storage mechanisms underpinning the database.

As in a traditional DBMS, the highest level storage abstraction in a DDMS is the table, containing zero or more rows.  Note however, that the storage implementation need not be related to the table structure; rows from different tables may be interleaved, co-clustered, stored in hierarchical index structures, or kept in whichever form is most efficient for the DDMS' specific workload.  

For example, consider a DDMS constructed to maintain a view of the base relations as well as the query\texttt{\\
SELECT prof, COUNT(DISTINCT e.student)\\
FROM professors p, classes c, enrollment e\\
WHERE p.id = c.teacher AND c.id = e.class\\
GROUP BY prof
}
Here, \texttt{c.teacher} and \texttt{e.class} are foreign keys mapped to \texttt{p.id} and \texttt{c.id} respectively.  We can implement this query in the DDMS by using two tables: one of students for each professor, and another containing the count of students for each professor.  Though the data is drawn from two distinct logical tables, the most efficient storage mechanism would place the student list on the same page as the student count.

In keeping with the DDMS' update-centric view of the world, we view each transition function as a series of write operations, each of which may require one or more read operations.  Both read and write operations may operate on a single row, or iterate over a range of values keys from one or more columns.  We can construct a directed hypergraph out of the write operations, with each node representing a table, each out-edge representing a read, and each in-edge representing a write.  We refer to this hypergraph as the transition function's data-flow graph.

This data-flow graph provides a useful mechanism for analyzing the storage, processing, and IO requirements of a query.  In particular it simplifies the analysis of schemes that partition data across multiple physical units, be they disk blocks, disks, or storage servers.

\begin{itemize}
\item Overview of read, write, and message costs

\item Partitioning across multiple axis, expanding the dataflow graph into a messaging graph.  

\item Messaging and computational (memory) resources, separability of subgraphs.

\item Basic instantiation - 1 disk or cluster

\item Extensions to the multi-disk case
\end{itemize}

\section{Discussion and Conclusions}

%\footnotesize{
\bibliographystyle{abbrv}
\bibliography{ref}
%}



\end{document}
